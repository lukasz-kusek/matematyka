\chapter{Rachunek r�niczkowy funkcji wielu zmiennych}

\section{Pochodna funkcji}

\subsection{Pochodna cz�stkowa}

\subsection{Pochodna kierunkowa}

\subsection{Pochodne cz�stkowe wy�szych rz�d�w}

\section{R�niczka funkcji}

\subsection{Twierdzenie o przedstawieniu przyrostu funkcji}

\subsection{Funkcja r�niczkowalna}

\subsection{R�niczka funkcji}

\subsection{R�niczka zupe�na}

\cite[Paragraf 11.5, Paragraf 11.6]{krysicki2}

\medskip

Niech
\begin{itemize}
\item funkcje $P(x,y)$ i~$Q(x,y)$ s� klasy $C^0$ w~obszarze $D$
\item funkcja $F(x,y)$ jest klasy $C^1$ w~obszarze $D$
\end{itemize}

\medskip

\begin{definicja}
Wyra�enie

\[
P(x,y) \: \textnormal{d}x + Q(x,y) \: \textnormal{d}y
\]

\medskip

\noindent jest \textbf{\emph{r�niczk� zupe�n�}}\label{def:rozniczka_zupelna} funkcji $F(x,y)$ je�eli zachodz� zwi�zki

\[
\frac{\partial F}{\partial x} = P(x,y), \qquad \frac{\partial F}{\partial y} = Q(x,y)
\]

\medskip

\noindent w~ka�dym punkcie obszaru $D$.
\end{definicja}

\bigskip
%%%%%%%%%%%%%%%%%%%%%%%%%%%%%%%%%%%%%%%%%%%%%%%%%
\subsubsection{Warunek konieczny i~wystarczaj�cy}

Niech funkcje $P(x,y)$ i~$Q(x,y)$ s� klasy $C^1$ w~obszarze jednosp�jnym $D$ (Def. \ref{def:obszar_jednospojny}, str. \pageref{def:obszar_jednospojny}).

\medskip

\begin{twierdzenie}
Warunkiem koniecznym i~wystarczaj�cym na to, aby wyra�enie 

\[
P(x,y) \: \textnormal{d}x + Q(x,y) \: \textnormal{d}y
\]

\medskip

\noindent by�o \textbf{\emph{r�niczk� zupe�n�}} w~tym obszarze, jest spe�nienie r�wno�ci

\[
\frac{\partial P}{\partial y} = \frac{\partial Q}{\partial x}
\]

\medskip

\noindent w~ka�dym punkcie obszaru $D$.
\end{twierdzenie}
