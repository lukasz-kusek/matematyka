%\documentclass[10pt,a4paper]{article}
\documentclass[10pt,b5paper]{book}
\usepackage[latin2]{inputenc}
\usepackage[polish]{babel}
\selectlanguage{polish}
\usepackage[OT4]{fontenc}
\usepackage{polski}
\usepackage{ucs}
\usepackage{amsthm}
\usepackage{amsmath}
\usepackage{amsfonts}
\usepackage{array}
\usepackage{tabularx}
\usepackage{makeidx}
%\usepackage{pict2e}

\newcommand{\codomain}{\mathrel{%
  \setlength{\unitlength}{2pt}
  \begin{picture}(6,4)(-1,0)
     \qbezier(3,3.75)(0,2)(3,0.25)
     \qbezier(3,3.75)(4,4)(6,4)
     \qbezier(3,0.25)(4,0)(6,0)
     \qbezier(3,3.75)(0,2)(3,0.25)
     \qbezier(3,3.75)(4,4)(6,4)
     \qbezier(3,0.25)(4,0)(6,0)
     \put(6,0){\line(0,1){4}}
  \end{picture}}}

\newcommand{\domain}{\mathrel{%
  \setlength{\unitlength}{2pt}
  \begin{picture}(6,4)(-2,0)
     \qbezier(3,3.75)(6,2)(3,0.25)
     \qbezier(3,3.75)(2,4)(0,4)
     \qbezier(3,0.25)(2,0)(0,0)
     \qbezier(3,3.75)(6,2)(3,0.25)
     \qbezier(3,3.75)(2,4)(0,4)
     \qbezier(3,0.25)(2,0)(0,0)
     \put(0,0){\line(0,1){4}}
  \end{picture}}}

\makeindex

\usepackage[pdftex]{graphicx}
\usepackage[unicode, colorlinks, pdftex, plainpages=false, hyperindex, pdffitwindow=false, pdfusetitle=true]{hyperref}
  \hypersetup{
    unicode=true
  }
\RequirePackage{color}
%\hypersetup{colorlinks=false}
%\definecolor{KolorLink}{cmyk}{0,0.89,0.94,0.28}
%\definecolor{KolorUrl}{cmyk}{1,1,0,0.5}%<--
\definecolor{KolorLink}{cmyk}{0,0.6,1,0.2}
\definecolor{KolorUrl}{cmyk}{1,0,1,0.28}
\hypersetup{linkcolor=KolorLink,citecolor=KolorLink,urlcolor=KolorUrl}

\newtheorem{twierdzenie}{Twierdzenie}[section]
\newtheorem{definicja}{Definicja}[section]

\author{�ukasz Kusek}
\title{Matematyka - notatki z wyk�adu}
%\institute{Wy�sza Szko�a Oficerska Si� Powietrznych w D�blinie}


\begin{document}

\frontmatter

\begingroup
%% Pocz�tek strony tutu�owej:
\sffamily %%<-- str. tytu�owa sans-serif
\thispagestyle{empty}

\vspace*{\stretch{1}}
\noindent

\makebox[0pt][l]{\begin{minipage}{\textwidth}
\flushright{\Huge\bfseries 
 Matematyka }
\noindent\rule[-1ex]{\textwidth}{3.3pt}\\[2.5ex]
\hfill
 \begingroup
    \emph{\Large notatki z wyk�adu}
 \endgroup
\end{minipage}}

\vspace{\stretch{2}}
\noindent\makebox[0pt][l]{\begin{minipage}{\textwidth}

\flushright
{\bfseries �ukasz Kusek}\\[25mm]
Wy�sza Szko�a Oficerska Si� Powietrznych w D�blinie\\[1.5ex]
\bfseries Wersja robocza z dnia \today
\end{minipage}}

\vspace{\stretch{2}}


\pagebreak
\endgroup

%\maketitle
%\pagebreak

\begingroup
\begin{small}
Copyright \copyright{} 2009-2010 �ukasz Kusek.

Wszelkie prawa zastrze�one.

\bigskip

Kontakt:

tel. +48 509 955 365
\end{small}
\endgroup


% Spis tre�ci
\tableofcontents


% Wst�p
\include{rozdzialy/wstep}

% Poczatek tresci
\mainmatter

\chapter{Relacje}

\begin{definicja}
Przez \textbf{\emph{par� uporz�dkowan�}}\label{def:para_uporzadkowana}\index{para!uporz�dkowana} $(a, b)$ rozumie� b�dziemy zbi�r pewnych podzbior�w zbioru $\{a,b\}$, a~mianowicie:
\[
(a,b) \quad = \quad \{\{a\}, \{a,b\}\}
\]

\smallskip

\cite[Definicja 1.1.1]{furdzik}
\end{definicja}

\bigskip

\begin{definicja}
\textbf{\emph{Iloczynem kartezja�skim}}\label{def:iloczyn_kartezjanski}\index{iloczyn!kartezja�ski} zbior�w $A$ i~$B$ nazywamy \textbf{zbi�r wszystkich par uporz�dkowanych} $(a,b)$, takich, �e

\begin{itemize}
\item $a \in A$,
\item $b \in B$
\end{itemize}

\noindent i~oznaczmy symbolem

\[
A \times B \quad = \quad \left\{ (a,b) \colon a \in A \wedge b \in B \right\}
\]

\smallskip

\cite[Definicja 1.1.3]{furdzik}
\end{definicja}

\bigskip
%%%%%%%%%%%%%%%%%
\section{Relacja}

\begin{definicja}
\textbf{\emph{Relacj�}}\label{def:relacja}\index{relacja} $\mathcal{R}$ okre�lon� w zbiorach $A$ i~$B$ (zachodz�c� mi�dzy elementami zbior�w $A$ i~$B$) nazywamy \textbf{tr�jk� uporz�dkowan�} (Def. \ref{def:para_uporzadkowana}, str. \pageref{def:para_uporzadkowana})

\[
\left( A, gr\mathcal{R}, B \right)
\]

\noindent gdzie

\begin{itemize}
\item $gr\mathcal{R}$ jest \textbf{podzbiorem iloczynu kartezja�skiego} $A \times B$.
\end{itemize}

\smallskip

\cite[Definicja 1.2.1]{furdzik}
\end{definicja}

\bigskip
\begin{definicja}Niech

\begin{itemize}
\item $a \in A$,
\item $b \in B$.
\end{itemize}

M�wimy, �e element $a$ \textbf{\emph{pozsotaje w relacji}} $\mathcal{R}$ z elementem $b$ ($a \mathcal{R} b$), wtedy i~tylko wtedy, gdy

\[
a \mathcal{R} b \quad \Leftrightarrow \quad (a, b) \; \in \; gr \mathcal{R}
\]

\smallskip

\cite[Definicja 1.2.2]{furdzik}
\end{definicja}

\bigskip
\begin{definicja}
\textbf{\emph{Dziedzin� relacji}}\label{def:dziedzina_relacji}\index{relacja!dziedzina}\index{dziedzina!relacji} ($\domain \mathcal{R}$)

\[
\mathcal{R} \; = \; \left(A, \: gr\mathcal{R}, \: B\right)
\]

\noindent nazywamy \textbf{podzbi�r zbioru} $A$, okre�lony nast�puj�co:

\[
\domain\mathcal{R} \; = \; \left\{ a \in A \colon \; \exists_{b \in B} \quad a \mathcal{R} b \right\}
\]

\smallskip

\cite[Definicja 1.2.3]{furdzik} 
\end{definicja}

\bigskip
\begin{definicja}
\textbf{\emph{Przeciwdziedzin� relacji}}\label{def:przeciwdziedzina_relacji}\index{relacja!przeciwdziedzina}\index{przeciwdziedzina!relacji} ($\codomain \mathcal{R}$)

\[
\mathcal{R} \; = \; \left(A, \: gr\mathcal{R}, \: B\right)
\]

\noindent nazywamy \textbf{podzbi�r zbioru} $B$, okre�lony nast�puj�co:

\[
\codomain \mathcal{R} \; = \; \left\{ b \in B \colon \; \exists_{a \in A} \quad a \mathcal{R} b \right\}
\]

\smallskip

\cite[Definicja 1.2.4]{furdzik}
\end{definicja}

\bigskip
\begin{definicja}
Dana jest relacja 
\[
\mathcal{R} \; = \; \left(A, \: gr\mathcal{R}, \: B\right)
\]

\textbf{\emph{Relacj�}} do niej \textbf{\emph{odwrotn�}}\label{def:relacja_odwrotna}\index{relacja!odwrotna} nazywamy relacj� 

\[
\mathcal{R}^{-1} \; = \; \left(B, \: gr\mathcal{R}^{-1}, \: A\right)
\]

\noindent gdzie

\[
gr \mathcal{R}^{-1} \; = \; \left\{(b,a) \: \in \: B \times A \colon \quad (a,b) \: \in \; gr\mathcal{R} \: \subset \: A \times B\right\}
\]

\smallskip

\cite[Definicja 1.2.5]{furdzik}
\end{definicja}

\medskip

\begin{definicja}
\textbf{\emph{Relacj�}}

\[
\mathcal{R} \; = \; \left(A, \: gr\mathcal{R}, \: B\right)
\]

\noindent nazywamy \textbf{\emph{wsz�dzie okre�lon�}}\label{def:relacja_wszedzie_okreslona}\index{relacja!wsz�dzie okre�lona}, je�eli

\[
\domain \mathcal{R} \; = \; A
\]

\noindent czyli

\[
\forall \: a \in A \quad \exists b \: \in B \colon \qquad (a,b) \: \in \: gr\mathcal{R}
\]

\smallskip

\cite[Definicja 1.2.16]{furdzik}
\end{definicja}


\bigskip

\begin{definicja}
\textbf{\emph{Relacj�}} 

\[
\mathcal{R} \; = \; \left(A, \: gr\mathcal{R}, \: B\right)
\]

\noindent nazywamy \textbf{\emph{surjektywn�}}\label{def:relacja_surjektywna}\index{relacja!surjektywna}, je�eli

\[
\codomain \mathcal{R} \; = \; B
\]

\noindent czyli

\[
\forall \: b \in B \quad \exists \: a \in A \colon \qquad (a,b) \: \in \: gr\mathcal{R}
\]

\smallskip

\cite[Definicja 1.2.17]{furdzik}
\end{definicja}

\bigskip

\begin{definicja}
\textbf{\emph{Relacj�}}

\[
\mathcal{R} \; = \; \left(A, \: gr\mathcal{R}, \: B\right)
\]

\noindent nazywamy \textbf{\emph{injektywn�}}\label{def:relacja_injektywna}\index{relacja!injektywna} (r�nowarto�ciow�)\index{relacja!r�nowarto�ciowa}, je�eli

\begin{tabular*}{\textwidth}%
{@{\extracolsep{\stretch{1}}}lc}
& \\
$\forall \: a_1, a_2 \in A \quad \forall \: b \in B \colon$
& \vspace{0.3cm} \\
\multicolumn{2}{c}{
$\left(a_1, b\right) \in gr\mathcal{R} \; 
\wedge \;
\left(a_2, b\right) \in gr\mathcal{R}
\quad
\Rightarrow
\quad
a_1 = a_2$
\vspace{0.3cm}
}
\end{tabular*}

\smallskip

\cite[Definicja 1.2.18]{furdzik}
\end{definicja}

\bigskip
\begin{definicja}
\textbf{\emph{Relacj�}} $\mathcal{R}$ nazywamy \textbf{\emph{bijektywn�}}\label{def:relacja_bijektywna}, je�eli jest

\begin{itemize}
\item \textbf{suriektywna}
\item i~\textbf{injektywna}.
\end{itemize}
\end{definicja}

\bigskip
%%%%%%%%%%%%%%%%%
\section{Funkcja}

\begin{definicja}
Relacj� 

\[
\mathcal{R} \; = \; \left(A, \: gr\mathcal{R}, \: B\right)
\]

\noindent nazywamy \textbf{\emph{funkcj�}}\label{def:funkcja}\index{funkcja}, je�eli

\begin{tabular*}{\textwidth}%
{@{\extracolsep{\stretch{1}}}lc}
& \\
$\forall \: a \in A \quad \forall \: b_1, b_2 \in B \colon$
& \vspace{0.3cm} \\
\multicolumn{2}{c}{
$\left(a,b_1\right) \in gr\mathcal{R} \; \wedge \; \left(a, b_2\right) \in gr\mathcal{R} \quad \Rightarrow \quad b_1 = b_2$
\vspace{0.3cm}
}
\end{tabular*}

\smallskip

\cite[Definicja 1.2.20]{furdzik}
\end{definicja}

\bigskip

\begin{twierdzenie}
Relacja \textbf{odwrotna} (Def. \ref{def:relacja_odwrotna}, str. \pageref{def:relacja_odwrotna}) do relacji \textbf{injektywnej} (Def. \ref{def:relacja_injektywna}, str. \pageref{def:relacja_injektywna}) jest \textbf{funkcj�}.
\end{twierdzenie}

\bigskip
%%%%%%%%%%%%%%%%%%%%%%
\section{Odwzorowanie}

\begin{definicja}
\textbf{Funkcj�}

\[
\mathcal{R} \; = \; \left(A, \: gr\mathcal{R}, \: B\right)
\]

\noindent \textbf{wsz�dzie okre�lon�} (Def. \ref{def:relacja_wszedzie_okreslona}, str. \pageref{def:relacja_wszedzie_okreslona}) nazywamy \textbf{\emph{odwzorowaniem}}\label{def:odwzorowanie}\index{odwzorowanie} i~oznaczmy

\[
\mathcal{R} \colon \: A \rightarrow B
\]

\smallskip

\cite[Definicja 1.2.21]{furdzik} \cite[Rozdzia� V]{onyszkiewicz}
\end{definicja}



\bigskip
%%%%%%%%%%%%%%%%%%%%%%%%%%%%%%%
\section{Funkcje. Odwzorowania}

\label{podstawowe_definicje_dla_funkcji}
Niech b�d� dane

\begin{itemize}
\item zbi�r $X$
\item zbi�r $Y$
\item relacja (Def. \ref{def:relacja}, str. \pageref{def:relacja})
\[
f \; = \; \left(X, \: gr\: f, \: Y\right)
\]
\end{itemize}

\bigskip

W przypadku gdy \textbf{\emph{relacja}} $f$ jest \textbf{\emph{funkcj�}} (Def. \ref{def:funkcja}, str. \pageref{def:funkcja}) lub \textbf{\emph{odwzorowaniem}} (Def. \ref{def:odwzorowanie}, str. \pageref{def:odwzorowanie}), zamiast m�wi�
\begin{center}
$x$ pozostaje w relacji $f$ z $y$
\end{center}
m�wimy:
\begin{itemize}
\item $x$-owi odpowiada $y$,
\item $x$ przechodzi w $y$,
\item $x$ odwzorowuje si� w $y$,
\item $y$ jest warto�ci� $f$ w $x$, co zapisujemy
\[
y = f(x)
\]
lub
\[
x \rightarrow y = f(x)
\]
\end{itemize}

\bigskip

Zapis $f \; = \; \left(X, \: gr\: f, \: Y\right)$, je�li jest

\medskip

\begin{itemize}
\item \textbf{odwzorowaniem} zapisujemy przez
\[
f \colon \; X \ni \: x \: \rightarrow \: y \quad = \quad  f(x) \: \in Y
\]
lub kr�tko
\[
f \colon \; X \: \rightarrow \: Y
\]
\medskip
i~czytamy \emph{$f$ odworowuje zbi�r $X$ w~zbi�r~$Y$}

\medskip

\item \textbf{funkcj�} to piszemy
\[
f \colon \; X \supset \domain f \ni \: x \: \rightarrow \: y \quad = \quad \: f(x) \: \in Y
\]
lub kr�tko
\[
f \colon \; X \supset \: \domain f \: \rightarrow \: Y
\]
\medskip
i~czytamy \emph{$f$ odwzorowuje swoj� dziedzin� zawart� w~zbiorze~$X$ w~zbi�r~ $Y$}
\end{itemize}

\bigskip

Je�eli $x \rightarrow y = f(x)$ to

\medskip

\begin{itemize}
\item element $y$ nazywamy
\begin{itemize}
\item \textbf{\emph{obrazem}}\index{funkcja!obraz elementu} elementu $x$ poprzez funkcj� (odwzorowanie) $f$
\item \textbf{\emph{warto�ci�}}\index{funkcja!warto��} funkcji (odwzorowania) $f$ w punkcie $x$
\end{itemize}
\item element $x$ nazywamy
\begin{itemize}
\item \textbf{\emph{przeciwobrazem}}\index{funkcja!przeciwobraz elementu} elementu $y$ poprzez funkcj� (odwzorowanie) $f$
\item \textbf{\emph{argumentem}}\index{funkcja!argument} funkcji (odwzorowania) $f$
\end{itemize}
\end{itemize}

\bigskip
%%%%%%%%%%%%%%%%%%%%%%%%%%%%%%%%%%%%%%%%
\subsection{Dziedzina. Przeciwdziedzina}

Zgodnie z definicj� \textbf{\emph{dziedziny relacji}} (Def. \ref{def:dziedzina_relacji}, str. \pageref{def:dziedzina_relacji}) otrzymujemy \label{def:dziedzina_funkcji}\index{dziedzina!funkcji}\index{funkcja!dziedzina}

\[
\domain f \; = \; \left\{x \in X \colon \quad \exists_{y \in Y} \quad y \: = \: f(x)\right\}
\]

\medskip

\noindent jak r�wnie� zgodnie z~definicj� \textbf{\emph{przeciwdziedziny relacji}} (Def. \ref{def:przeciwdziedzina_relacji}, str. \pageref{def:przeciwdziedzina_relacji}) otrzymujemy \label{def:przeciwdziedzina_funkcji}\index{przeciwdziedzina!funkcji}\index{funkcja!przeciwdziedzina}

\[
\codomain f \; = \; \left\{y \in Y \colon \quad \exists_{x \in X} \quad y \: = \: f(x)\right\}
\]

W~przypadku
\begin{itemize}
\item \textbf{\emph{odwzorowania}}

\[
\domain f \: = \: X
\]

\noindent w�wczas zbi�r $X$ nazywamy \textbf{\emph{zbiorem argument�w}}\index{odwzorowanie!zbi�r argument�w}\index{zbi�r!argument�w}

\item \textbf{\emph{funkcji}} na og� (gdy nie jest odwzorowaniem)

\[
\domain f \: \neq \: X, \qquad \domain f \: \subset \: X
\]

\noindent w�wczas zbi�r $X$ nazywamy \textbf{\emph{naddziedzin� funkcji}}\index{funkcja!naddziedzina}.
\end{itemize}

\medskip
Zbi�r $Y$ nazywamy \textbf{\emph{zapasem}} funkcji (odwzorowania)\index{funkcja!zapas}\index{odwzorowanie!zapas}.

\medskip

\cite[Definicja 1.3.1]{furdzik}

\bigskip
%%%%%%%%%%%%%%%%%%%
\subsection{Wykres}

\begin{definicja}
\textbf{\emph{Wykresem}}\label{def:wykres_funkcji} funkcji (odwzorowania) $f$ jest zbi�r

\[
gr \: f \; = \; \left\{(x,y) \; \in \; X \times Y \colon \quad x \: \rightarrow \: y \; = \; f(x)\right\}
\]

\smallskip

\cite[Definicja 1.3.2]{furdzik}
\end{definicja}

\bigskip
%%%%%%%%%%%%%%%%%%%%%%%%%%%%%%%%%%%%%%%%%%
\subsection{Surjekcja. Injekcja. Bijekcja}

\begin{definicja}
\textbf{Funkcj�} (\textbf{odwzorowanie}) nazywamy \textbf{\emph{surjekcj�}}\label{def:surjekcja}

\begin{center}
zbioru $\domain f \subset X$ na $Y$,
\end{center}

\noindent je�eli

\begin{center}
$f$ jest relacj� surjektywn� (Def. \ref{def:relacja_surjektywna}, str. \pageref{def:relacja_surjektywna}).
\end{center}

\smallskip

\cite[Definicja 1.3.2]{furdzik}
\end{definicja}

\medskip

\begin{definicja}
\textbf{Funkcj�} (\textbf{odwzorowanie}) nazywamy \textbf{\emph{injekcj�}}\label{def:injekcja}

\begin{center}
zbioru $\domain f \subset X$ w $Y$,
\end{center}

\noindent je�eli

\begin{center}
$f$ jest relacj� injektywn� (Def. \ref{def:relacja_injektywna}, str. \pageref{def:relacja_injektywna}).
\end{center}

\smallskip

\cite[Definicja 1.3.2]{furdzik}
\end{definicja}

\medskip

\begin{definicja}
\textbf{Odwzorowanie} jest \textbf{\emph{bijekcj�}}\label{def:bijekcja}

\begin{center}
zbioru $X$ na $Y$,
\end{center}

\noindent je�eli jest \textbf{r�wnocze�nie}
\begin{itemize}
\item surjekcj�
\item i injekcj�.
\end{itemize}

\smallskip

\cite[Definicja 1.3.2]{furdzik}
\end{definicja}


\bigskip
%%%%%%%%%%%%%%%%%%%%%%%%%%%%%%%%%%%%%%%%%%%%%%
\subsection{Obraz zbioru. Przeciwobraz zbioru}

\begin{definicja}
\textbf{\emph{Obrazem zbioru}}\index{obraz!zbioru}\index{zbi�r!obraz} $A$ poprzez funkcj� (odwzorowanie) 

\[
f \colon \; X \supset \; \domain f \: \rightarrow \: Y
\]

\noindent nazywamy zbi�r

\[
f[A] \; = \; \left\{y \in Y \colon \quad \exists_{x \in A} \quad x \: \rightarrow \: y \; = \; f(x)\right\}
\]

\smallskip

\cite[Definicja 1.3.3]{furdzik}
\end{definicja}

\bigskip

\begin{definicja}
\textbf{\emph{Przeciwobrazem zbioru}}\index{przeciwobraz!zbioru}\index{zbi�r!przeciwobraz} $B$ poprzez funkcj� (odwzorowanie)

\[
f \colon \; X \supset \; \domain f \: \rightarrow \: Y
\]

\noindent nazywamy zbi�r 

\[
f^{-1}[B] \; = \; \left\{x \in X \colon \quad \exists_{y \in B} \quad x \: \rightarrow \: y \; = \; f(x)\right\}
\]

\smallskip

\cite[Definicja 1.3.3]{furdzik}
\end{definicja}
\chapter{Struktury algebraiczne}

\medskip

Literatura do tego dzia�u: \cite{rut} - 1.1.1, 1.1.2, 1.2.1 \newline
Zadania do tego dzia�u: \cite{rut} - 1.1.1, 1.1.2, 1.2.1

\medskip

\begin{definicja}
\textbf{\emph{Dzia�aniem wewn�trznym}}\label{def:dzialanie_wewnetrzne}\index{dzia�anie}\index{dzia�anie!wewn�trzne} (lub kr�tko \textbf{\emph{dzia�aniem}}\label{def:dzialanie}) w~zbiorze $A$ nazywamy dowolne odwzorowanie (Def. \ref{def:odwzorowanie}, str. \pageref{def:odwzorowanie}) produktu kartezja�skiego (Def. \ref{def:iloczyn_kartezjanski}, str. \pageref{def:iloczyn_kartezjanski})

\begin{center}
$A \times A$ w~zbi�r~$A$.
\end{center}

\smallskip

\cite[Definicja 3, rozdzia� 1.1.1]{rut}
\end{definicja}

\bigskip

\begin{definicja}
M�wimy, �e dzia�anie $\circ$ w zbiorze $A$ jest \textbf{\emph{przemienne}}\label{def:dzialanie_przemienne}\index{dzia�anie!przemienne}, je�li 

\[
\forall \: a,b \in A \colon \qquad a\circ b \; = \; b \circ a
\]

\smallskip

\cite[Definicja 5, rozdzia� 1.1.2]{rut}
\end{definicja}

\bigskip

\begin{definicja}
M�wimy, �e dzia�anie $\circ$ w zbiorze $A$ jest \textbf{\emph{��czne}}\label{def:dzialanie_laczne}\index{dzia�anie!��czne}, je�li 

\[
\forall a, b, c \in A \colon \qquad (a \circ b) \circ c \; = \; a \circ (b \circ c)
\]

\smallskip

\cite[Definicja 6, rozdzia� 1.1.2]{rut}
\end{definicja}

\bigskip

\begin{definicja}
M�wimy, �e element $e \in A$ jest \textbf{\emph{elementem neutralnym}}\label{def:element_neutralny}\index{element!neutralny}\index{dzia�anie!element neutralny} dzia�ania $\circ$ okre�lonego w $A$, je�li

\[
\forall a \in A \colon \qquad a \circ e \; = \; e \circ a \; = \; a
\]

\smallskip

Element neutralny w~notacji

\begin{itemize}
\item multiplikatywnej nazywa si� elementem jednostkowym\index{element!jednostkowy} lub \textbf{\emph{jedynk�}}\label{def:jedynka_dzialania}\index{dzia�anie!jedynka}\index{jedynka!dzia�ania} i~oznacza si� go cz�sto symbolem $\mathbf{1}$.

\item addytywnej nazywamy \textbf{\emph{zerem}}\label{def:zero_dzialania}\index{dzia�anie!zero}\index{zero!dzia�ania} i~oznaczamy go symbolem $\mathbf{0}$.
\end{itemize}

\smallskip

\cite[Definicja 7, rozdzia� 1.1.2]{rut}
\end{definicja}

\bigskip

\begin{definicja}
Niech 

\begin{itemize}
\item dzia�anie $\circ$ w zbiorze $A$ \textbf{ma element neutralny} $e$

\item $a \in A$.
\end{itemize}

\medskip

Ka�dy element $b \in A$ spe�niaj�cy r�wno�� 

\[
a \circ b \; = \; b \circ a \; = \; e
\]

\noindent nazywamy \textbf{\emph{elementem odwrotnym do $a$}}\label{def:element_odwrotny}\index{element!odwrotny}.

\bigskip

Je�li istnieje \textbf{dok�adnie jeden} element odwrotny do $a$, to oznaczamy go symbolem $a^{-1}$.

\medskip

W~notacji addytywnej element odwrotny do $a$ nazywamy \textbf{elementem przeciwnym}\label{def:element_przeciwny}\index{element!odwrotny} do $a$ i~zamiast $a^{-1}$ piszemy~$-a$.

\smallskip

\cite[Definicja 8, rozdzia� 1.1.2]{rut}
\end{definicja} 

\bigskip

\begin{definicja}
Niech w zbiorze $A$ okre�lone b�d� dzia�ania $\odot$ oraz $\oplus$.

\medskip

M�wimy, �e dzia�anie $\odot$ jest \textbf{\emph{rozdzielne}}\label{def:dzialanie_rozdzielne}\index{dzia�anie!rozdzielne} wzgl�dem dzia�ania $\oplus$, je�li

\[
\forall \: a, b, c \in A \colon \qquad a \odot (b \oplus c) \; = \; (a \odot b) \oplus (a \odot c)
\]

\noindent oraz

\[
\forall \: a, b, c \in A \colon \qquad (a \oplus b) \odot c \; = \; (a \odot c) \oplus (b \odot c)
\]

\smallskip

\cite[Definicja 9, rozdzia� 1.1.2]{rut}

\end{definicja}

\bigskip

\begin{definicja}
Niech $A$ i $F$ b�d� dowolnymi zbiorami niepustymi.

\medskip

\textbf{\emph{Dzia�aniem zewn�trznym}}\label{def:dzialanie_zewnetrzne}\index{dzia�anie!zewn�trzne} w~zbiorze $A$ nazywamy dowolne odwzorowanie (Def. \ref{def:odwzorowanie}, str. \pageref{def:odwzorowanie}) produktu kartezja�skiego (Def. \ref{def:iloczyn_kartezjanski}, str. \pageref{def:iloczyn_kartezjanski})

\begin{center}
$F \times A$ w~zbi�r $A$.
\end{center}

\medskip

Zbi�r $F$ nazywamy \emph{zbiorem operator�w}\index{zbi�r!operator�w}.

\smallskip

\cite[Definicja 10, rozdzia� 1.1.3]{rut}.
\end{definicja}

\bigskip

\begin{definicja}
\textbf{\emph{Struktur� algebraiczn�}}\label{def:struktura_algebraiczna}\index{struktura algebraiczna} okre�lon� na zbiorze $A$ nazywamy zesp�

\[
(A; \quad F_1, \ldots, F_m; \quad \circ_1, \ldots, \circ_n; \quad \bullet_1, \ldots, \bullet_m),
\]

\noindent gdzie

\begin{tabular}{m{3cm}m{8cm}m{0,1cm}}
 
$F1, \ldots, F_m \qquad$ &
s� zbiorami, &
 \\[10pt]
 
$\circ_1, \ldots, \circ_n \qquad$ &
s� dzia�aniami wewn�trznymi (Def. \ref{def:dzialanie_wewnetrzne}, str. \pageref{def:dzialanie_wewnetrzne}) w~zbiorze $A$ &
 \\[20pt]
 
$\bullet_1, \ldots, \bullet_m \qquad$ &
s� takimi dzia�aniami zewn�trznymi (Def. \ref{def:dzialanie_zewnetrzne}, str. \pageref{def:dzialanie_zewnetrzne}) w~$A$, �e &
 \\[20pt]
 
&
$\bullet_1 \colon \qquad F_1 \times A \to A, \quad \ldots, \quad F_m \times A \to A$. &
 \\[10pt]
 
\end{tabular}

\smallskip

\cite[Definicja 11, rozdzia� 1.2.1]{rut}
\end{definicja}

\bigskip
%%%%%%%%%%%%%%%%%%%%%
\section{Grupa}

\medskip

\begin{definicja}
\textbf{\emph{Grup�}}\index{grupa} nazywamy par� (Def. \ref{def:para_uporzadkowana}, str. \pageref{def:para_uporzadkowana})

\[
(G, \: \cdot), 
\]

\noindent sk�adaj�c� si�

\begin{itemize}
\item ze zbioru $G$
\item oraz z takiego dzia�ania $\cdot$ (Def. \ref{def:dzialanie}, str. \pageref{def:dzialanie}) okre�lonego w~zbiorze $G$ 

\[
\forall \: a, b \in G \colon \qquad G \times G \in \: (a, b) \: \to \: a + b \: \in G
\]

\noindent kt�re spe�nia warunki:
\begin{enumerate}
\item Dzia�anie $\cdot$ jest \textbf{\emph{��czne}} (Def. \ref{def:dzialanie_laczne}, str. \pageref{def:dzialanie_laczne})
\item Dzia�anie $\cdot$ ma \textbf{\emph{element neutralny}} (Def. \ref{def:element_neutralny}, str. \pageref{def:element_neutralny})
\item Dla ka�dego elementu zbioru $G$ istnieje \textbf{\emph{element odwrotny}} (Def. \ref{def:element_odwrotny}, str. \pageref{def:element_odwrotny})
\end{enumerate}
\end{itemize}

\smallskip

\cite[Definicja 19, rozdzia� 2.1.1]{rut}

\end{definicja}

\bigskip

\begin{definicja}
Grup� $(G, \cdot)$ nazywamy \textbf{\emph{grup� abelow�}}\label{def:grupa_abelowa}\index{grupa!abelowa} (\textbf{\emph{przemienn�}}), je�li dzia�anie $\cdot$ jest \textbf{\emph{przemienne}} (Def. \ref{def:dzialanie_przemienne}, str. \pageref{def:dzialanie_przemienne}). 
\end{definicja}

\bigskip

\begin{definicja}
\textbf{\emph{Elementem neutralnym grupy}}\index{element!neutralny!grupy}\index{grupa!element neutralny} $G$  nazywamy \textbf{\emph{element neutralny dzia�ania}} (Def. \ref{def:element_neutralny}, str. \pageref{def:element_neutralny}), wzgl�dem kt�rego $G$ jest grup�.

\medskip

Element neutralny grupy nazywamy w notacji

\begin{itemize}
\item multiplikatywnej - jedynk� grupy $G$\index{grupa!jedynka}\index{jedynka!dzia�ania},
\item addytywnej - zerem grupy $G$\index{grupa!zero}\index{zero!grupy}.
\end{itemize}

\smallskip

\cite[Definicja 21, rozdzia� 2.1.1]{rut}
\end{definicja}

\bigskip
%%%%%%%%%%%%%%%%%%%%%%%%%
\section{Pier�cie�}

\medskip

\begin{definicja}
Tr�jk� uporz�dkowan� (Def. \ref{def:para_uporzadkowana}, str. \pageref{def:para_uporzadkowana})

\[
(A, \: +, \: \cdot)
\]

\noindent sk�adaj�cy si� z

\begin{itemize}
\item niepustego zbioru $A$,
\item dzia�ania $+$ okre�lonego w~$A$ (Def. \ref{def:dzialanie}, str. \pageref{def:dzialanie})
\item oraz dzia�ania $\cdot$ okre�lonego w~$A$ (Def. \ref{def:dzialanie}, str. \pageref{def:dzialanie})
\end{itemize}

\noindent nazywamy \textbf{\emph{pier�cieniem}}\label{def:pierscien}\index{pier�cie�}, je�li spe�nione s� warunki:

\begin{enumerate}
\item $(A, +)$ jest \textbf{\emph{grup� abelow�}} (Def. \ref{def:grupa_abelowa}, str. \pageref{def:grupa_abelowa})
\item dzia�anie $\cdot$ jest \textbf{\emph{��czne}} (Def. \ref{def:dzialanie_laczne}, str. \pageref{def:dzialanie_laczne})
\item dzia�anie $\cdot$ jest \textbf{\emph{rozdzielne}} (Def. \ref{def:dzialanie_rozdzielne}, str. \pageref{def:dzialanie_rozdzielne}) wzgl�dem $+$
\end{enumerate}
\cite[Definicja 78, rozdzia� 3.1.1]{rut}
\end{definicja}

\bigskip

\begin{definicja}
Je�li dzia�anie $\cdot$ jest \textbf{\emph{przemienne}} (Def. \ref{def:dzialanie_przemienne}, str. \pageref{def:dzialanie_przemienne}), to pier�cie� nazywamy \textbf{\emph{pier�cieniem przemiennym}}\label{def:pierscien_przemienny}\index{pier�cie�!przemienny}.

\smallskip

\cite[Definicja 79, rozdzia� 3.1.1]{rut}
\end{definicja}

\bigskip

\begin{definicja}
\textbf{\emph{Element neutralny dodowania}} (Def. \ref{def:zero_dzialania}, str. \pageref{def:zero_dzialania}) w~pier�cieniu $A$ nazywamy \textbf{\emph{zerem pier�cienia}}\label{def:zero_pierscienia}\index{pier�cie�!zero}\index{zero!pier�cienia} $A$.

\smallskip

\cite[Definicja 80, rozdzia� 3.1.1]{rut}
\end{definicja}

\bigskip

\begin{definicja}
Je�li mno�enie $\cdot$ w pier�cieniu~A ma jedynk� (Def. \ref{def:jedynka_dzialania}, str. \pageref{def:jedynka_dzialania}), to jedynk� t� nazywamy \textbf{\emph{jedynk� pier�cienia}}\label{def:jedynka_pierscienia}\index{pier�cie�!jedynka}\index{jedynka!pier�cienia} $A$ i~m�wimy wtedy, �e $A$ jest \textbf{\emph{pier�cieniem z~jedynk�}}\label{def:pierscien_z_jedynka}\index{pier�cie�!z jedynk�}.

\smallskip

\cite[Definicja 81, rozdzia� 3.1.1]{rut}
\end{definicja}

\bigskip

\begin{definicja}
O~pier�cieniu $A$ m�wimy, �e jest \textbf{\emph{pier�cieniem zerowym}}\label{def:pierscien_zerowy}\index{pier�cie�!zerowy}, je�li zbi�r $A$ jest \textbf{jednoelementowy}.

W~przeciwnym przypadku o~pier�cieniu $A$ m�wimy, �e jest \textbf{\emph{pier�cieniem niezerowym}}\label{def:pierscien_niezerowy}\index{pier�cie�!niezerowy}.

\smallskip

\cite[Definicja 82, rozdzia� 3.1.1]{rut}
\end{definicja}

\bigskip
%%%%%%%%%%%%%%%%%%%%%
\section{Cia�o}

\medskip

\begin{definicja}
\textbf{\emph{Cia�em}}\label{def:cialo}\index{cia�o} nazywamy \textbf{\emph{pier�cie� z jedynk�}} (Def. \ref{def:pierscien_z_jedynka}, str. \pageref{def:pierscien_z_jedynka}) spe�niaj�cy warunki:
\begin{enumerate}
\item zbi�r $K$ ma przynajmniej dwa elementy
\item dla ka�dego elementu zbioru $K$ r�nego od zera grupy $(K, +)$ istnieje element \textbf{\emph{odwrotny}} (Def. \ref{def:element_odwrotny}, str. \pageref{def:element_odwrotny}):
\begin{center}
$\forall \: x \in K, \: x \neq \mathbf{0} \quad \exists \: x^{-1} \colon \qquad x \cdot x^{-1} \; = \; x^{-1} \cdot x  \; = \; \mathbf{1}$
\end{center}
\end{enumerate} 

\smallskip

\cite[Definicja 86, rozdzia� 3.1.4]{rut}
\end{definicja}


\chapter{Przestrze� liniowa (wektorowa)}

\medskip

Literatura do tego dzia�u: \cite{rut}~-~2.1.1, 3.1.1, 3.1.4, \cite{przybylo}~-~3, \cite{kostrykin2}~-~1.1, 1.2, \cite{ptak}~-~9.1, 9.2 \newline
Zadania do tego dzia�u: \cite{rut}~-~2.1.1, 3.1.1, 3.1.4, \cite{przybylo}~-~3.1, 3.2, \cite{ptak}~-~9

\bigskip

\begin{definicja}
\textbf{\emph{Przestrzeni� wektorow�}}\label{def:przestrzen_wektorowa} (\textbf{\emph{przestrzeni� liniow�}}\label{def:przestrzen_liniowa}) nad cia�em (Def. \ref{def:cialo}, str. \pageref{def:cialo}) $(F, +, \cdot)$ nazywamy struktur� algebraiczn� (Def. \ref{def:struktura_algebraiczna}, str. \pageref{def:struktura_algebraiczna}) 

\[
(V, F, \oplus, \odot)
\]

\noindent z�o�on� z:

\begin{itemize}
\item zbioru $V$ - zwanego \emph{zbiorem wektor�w},
\item zbioru $F$ - zwanego \emph{zbiorem skalar�w},
\item dzia�ania $\oplus \colon V \times V \to V$ wewn�trznego (Def. \ref{def:dzialanie_wewnetrzne}, str. \pageref{def:dzialanie_wewnetrzne}) w~zbiorze $V$
\item i~dzia�ania zewn�trznego (Def. \ref{def:dzialanie_zewnetrzne}, str. \pageref{def:dzialanie_zewnetrzne}) $\odot \colon F \times V \to V$,
\end{itemize}

\bigskip

\noindent kt�ra spe�nia nast�puj�ce warunki:

\begin{enumerate}
\item $(V, \oplus)$ jest grup� abelow�
\item $\forall \alpha \in F \quad \forall x,y \in V \colon \qquad \alpha \odot (x \oplus y) = (\alpha \odot x) \oplus (\alpha \odot y)$
\item $\forall \alpha, \beta \in F \quad \forall x \in V \colon \qquad (\alpha + \beta) \odot x = (\alpha \odot x) \oplus (\beta \odot x)$
\item $\forall \alpha, \beta \in F \quad \forall x \in V \colon \qquad \alpha \odot (\beta \odot x) = (\alpha \cdot \beta) \odot x$
\item $\forall x \in V \colon \qquad \qquad \qquad \quad \mathbf{1} \odot x = x$
\end{enumerate}

\smallskip

\cite[Rozdzia� 3]{przybylo}
\end{definicja}

\bigskip
%%%%%%%%%%%%%%%%%%%%%%%%%%%%%%%%%%%%%%%%%%%
\subsection{Kombinacja liniowa wektor�w}

\medskip

Niech $T = {1, \ldots, n}$ oznacza zbi�r wska�nik�w.

\begin{definicja}
Element $x \in V$ przestrzeni $(V, F, \oplus, \odot)$ nazywamy \textbf{\emph{kombinacj� liniow� wektor�w}}\label{def:kombinacja_liniowa_wektorow} $(x_t)_{t \in T}$, je�li istnieje uk�ad skalar�w $(\alpha_t)_{t \in T}$ z~cia�a $F$, taki, �e

\[
x = \sum_{t \in T} \alpha_t x_t
\]

\medskip

Skalary $\alpha_t$ nazywamy \textbf{\emph{wsp�czynnikami}} tej kombinacji liniowej.

\smallskip

\cite[Rozdzia� 3]{przybylo}
\end{definicja}

\bigskip

\begin{definicja}
Uk�ad $(x_t)_{t \in T}$ wektor�w przestrzeni $(V, F, \oplus, \odot)$ nazywamy \textbf{\emph{uk�adem wektor�w liniowo niezale�nych}}\label{def:liniowa_niezaleznosc_wektorow}, je�li dla dowolnego uk�adu $(\alpha_t)_{t \in T}$ skalar�w jest spe�niony warunek

\[
\sum_{t \in T} \alpha_t x_t = 0 \qquad \Rightarrow \qquad \forall t \in T \colon \quad \alpha_t = 0
\]

\smallskip

\cite[Rozdzia� 3]{przybylo}
\end{definicja}

\begin{definicja}
Uk�ad, kt�ry nie jest uk�adem wektor�w liniowo niezale�nych nazywamy \textbf{\emph{uk�adem wektor�w liniowo zale�nych}}\label{def:liniowa_zaleznosc_wektorow}.

\smallskip

\cite[Rozdzia� 3]{przybylo}
\end{definicja}

\bigskip
%%%%%%%%%%%%%%%%%%%%%%%%%%%%%%%%%%%%%%%%%
\subsection{Baza przestrzeni liniowej}

\medskip

\begin{definicja}
\textbf{\emph{Baz� przestrzeni}}\label{def:baza_przestrzeni} $(V, F, \oplus, \odot)$ nazywamy uk�ad wektor�w liniowo niezale�nych $(e_1, \ldots, e_n)$, kt�re generuj� ca�� przestrze�, tzn.

\[
\forall x \in V \quad \exists \alpha_i \in F \colon \qquad x = \sum_i^n \alpha_i e_i
\]

\smallskip

\cite[Rozdzia� 3]{przybylo}
\end{definicja}

\bigskip
%%%%%%%%%%%%%%%%%%%%%%%%%%%%%%%%%%%%%%%%%%%
\subsection{Wymiar przestrzeni liniowej}

\medskip

\begin{definicja}
Moc (liczb� wektor�w) bazy nazywamy \textbf{\emph{wymiarem przestrzeni}}\label{def:wymiar_przestrzeni} $(V,F, \oplus, \odot)$ i~oznaczamy 

\[
\dim V
\]

\smallskip

\cite[Rozdzia� 3]{przybylo}
\end{definicja}



\bigskip
%%%%%%%%%%%%%%%%%%%%%%%%%%%%%%%%%%%%%%%%%%%%
\section{Odwzorowania w przestrzeni linowej}

\medskip

Niech $X$, $Y$, $V_1, \ldots V_n$ oraz $Z$ b�d� przestrzeniami wektorowymi (Def. \ref{def:przestrzen_wektorowa}, str. \pageref{def:przestrzen_wektorowa}) nad cia�em (Def. \ref{def:cialo}, str. \pageref{def:cialo}) $K$.

\medskip

\noindent Niech $W$ oraz $V$ b�d� przestrzeniami wektorowymi (Def. \ref{def:przestrzen_wektorowa}, str. \pageref{def:przestrzen_wektorowa}) nad cia�em liczb zespolonych (Def. \ref{def:liczby_zespolone}, str. \pageref{def:liczby_zespolone}) $\mathbb{C}$.

\bigskip
%%%%%%%%%%%%%%%%%%%%%%%%%%%%%%%%
\subsection{Odwzorowanie linowe}

\begin{definicja}
Odwzorowanie (Def. \ref{def:odwzorowanie}, str. \pageref{def:odwzorowanie}) 

\[
f \colon X \rightarrow Y
\]

\medskip

\noindent nazywamy \textbf{\emph{liniowym}}\label{def:odwzorowanie_liniowe}, je�eli:
\[
\begin{array}{rcl}
\forall x, y \in X & \colon & f(x+y) = f(x) + f(y) \vspace{0.3cm}\\
\forall \alpha \in K \quad \forall x \in X & \colon & f(\alpha \: x) = \alpha \: f(x)
\end{array}
\]

\smallskip

\cite[Definicja 3.4.1]{furdzik}
\end{definicja}

\bigskip
%%%%%%%%%%%%%%%%%%%%%
\subsection{Endomorfizm}

\begin{definicja}
Je�eli

\[
X = Y
\]

\medskip

\noindent to odwzorowanie liniowe (Def. \ref{def:odwzorowanie_liniowe}, str. \pageref{def:odwzorowanie_liniowe}) nazywamy \textbf{\emph{endomorfizmem}}\label{def:endomorfizm}.

\smallskip

\cite[Definicja 3.4.1]{furdzik}
\end{definicja}

\bigskip
%%%%%%%%%%%%%%%%%%%%%%%%%%%%%%%%%%%%
\subsection{Odwzorowanie antylinowe}

\begin{definicja}
Odwzorowanie (Def. \ref{def:odwzorowanie}, str. \pageref{def:odwzorowanie}) 

\[
f \colon W \rightarrow V
\]

\medskip

\noindent nazywamy \textbf{\emph{antyliniowym}}\label{def:odwzorowanie_antyliniowe} (\textbf{\emph{p�liniowym}})\label{def:odwzorowanie_polliniowe}, je�eli:
\[
\begin{array}{rcl}
\forall x, y \in W & \colon & f(x+y) = f(x) + f(y) \vspace{0.3cm}\\
\forall \alpha \in \mathbb{C} \quad \forall x \in W & \colon & f(\alpha \: x) = \overline{\alpha} \: f(x)
\end{array}
\]

\medskip

\noindent Liczba $\overline{\alpha}$ oznacza liczb� sprz�on� (Def. \ref{def:liczba_zespolona_sprzezona}, str. \pageref{def:liczba_zespolona_sprzezona}) z~$\alpha$.
\end{definicja}

\bigskip
%%%%%%%%%%%%%%%%%%%%%%%%%%%%%%%%%%%
\subsection{Odwzorowanie p�toraliniowe}

\begin{definicja}
Odwzorowanie (Def. \ref{def:odwzorowanie}, str. \pageref{def:odwzorowanie}) 

\[
f \colon X \times Y \rightarrow W
\]

\medskip

\noindent nazywamy \textbf{\emph{p�toraliniowym}}\label{def:odwzorowanie_poltoraliniowe}, je�eli jest

\begin{itemize}
\item \textbf{odwzorowaniem liniowym} (Def. \ref{def:odwzorowanie_liniowe}, str. \pageref{def:odwzorowanie_liniowe}) ze wzgl�du na pierwszy argument
\item i~\textbf{odwzorowaniem p�liniowym} (Def. \ref{def:odwzorowanie_polliniowe}, str. \pageref{def:odwzorowanie_polliniowe}) ze wzgl�du na drugi argument
\end{itemize} 
 
czyli
\[
\begin{array}{rcl}
\forall x, x' \in X \quad y \in Y & \colon & f(x+x',y) = f(x,y) + f(x',y) \vspace{0.3cm}\\
\forall \alpha \in \mathbb{C} \quad \forall x \in X \quad \forall y \in Y & \colon & f(\alpha \: x,y) = \alpha \: f(x,y) \vspace{0.3cm}\\
\forall x \in X \quad y, y' \in Y & \colon & f(x,y+y') = f(x,y) + f(x,y') \vspace{0.3cm}\\
\forall \alpha \in \mathbb{C} \quad \forall x \in X \quad \forall y \in Y & \colon & f(x,\alpha \: y) = \overline{\alpha} \: f(x,y)
\end{array}
\]

\medskip

Liczba $\overline{\alpha}$ oznacza liczb� sprz�on� (Def. \ref{def:liczba_zespolona_sprzezona}, str. \pageref{def:liczba_zespolona_sprzezona}) z~$\alpha$.

\smallskip

\cite[Rozdzia� 3.2.1]{kostrykin2}
\end{definicja}

\bigskip
%%%%%%%%%%%%%%%%%%%%%%%%%%%%%%%%%%%%
\subsection{Odwzorowanie dwuliniowe}

\begin{definicja}
Odwzorowanie (Def. \ref{def:odwzorowanie}, str. \pageref{def:odwzorowanie}) 

\[
f \colon X \times Y \rightarrow Z
\]

\medskip

\noindent nazywamy \textbf{\emph{dwuliniowym}}\label{def:odwzorowanie_dwuliniowe}, je�eli jest \textbf{odwzorowaniem liniowym} (Def. \ref{def:odwzorowanie_liniowe}, str. \pageref{def:odwzorowanie_liniowe}) ze wzgl�du na \textbf{ka�d� zmienn�}, czyli
\[
\begin{array}{rcl}
\forall x, x' \in X \quad y \in Y & \colon & f(x+x',y) = f(x,y) + f(x',y) \vspace{0.3cm}\\
\forall \alpha \in K \quad \forall x \in X \quad \forall y \in Y & \colon & f(\alpha \: x,y) = \alpha \: f(x,y) \vspace{0.3cm}\\
\forall x \in X \quad y, y' \in Y & \colon & f(x,y+y') = f(x,y) + f(x,y') \vspace{0.3cm}\\
\forall \alpha \in K \quad \forall x \in X \quad \forall y \in Y & \colon & f(x,\alpha \: y) = \alpha \: f(x,y)
\end{array}
\]

\smallskip

\cite[Definicja 3.9.1]{furdzik} \cite[Rozdzia� 0.2]{trajdos}
\end{definicja}

\bigskip
%%%%%%%%%%%%%%%%%%%%%%%%%%%%%%%%%%%%
\subsection{Odwzorowanie wieloliniowe}

\begin{definicja}
Odwzorowanie (Def. \ref{def:odwzorowanie}, str. \pageref{def:odwzorowanie}) 

\[
f \colon V_1 \times V_2 \times \ldots V_n \rightarrow Z
\]

\medskip

\noindent nazywamy \textbf{\emph{wieloliniowym}}\label{def:odwzorowanie_wieloliniowe} (\textbf{\emph{$n$-liniowym}})\label{def:odwzorowanie_n-liniowe}, je�eli jest \textbf{odwzorowaniem liniowym} (Def. \ref{def:odwzorowanie_liniowe}, str. \pageref{def:odwzorowanie_liniowe}) ze wzgl�du na \textbf{ka�d� zmienn�}.

\smallskip

\cite[Paragraf 1.4.1]{kostrykin2}
\end{definicja}

\bigskip
%%%%%%%%%%%%%%%%%%%%%%%%%%%%%%%%%%%%%%%%%%%%
\section{Formy w przestrzeni linowej}

%%%%%%%%%%%%%%%%%%%%%%%%%%%%%%%%%%%%%%%%%%%%
\subsection{Forma p�toraliniowa}

\begin{definicja}
\textbf{\emph{Form� p�toraliniow�}}\label{def:forma_poltoraliniowa} na iloczynie kartezja�skim (Def. \ref{def:iloczyn_kartezjanski}, str. \pageref{def:iloczyn_kartezjanski}) $X \times Y$ nazywamy \textbf{odwzorowanie p�toraliniowe} (Def. \ref{def:odwzorowanie_poltoraliniowe}, str. \pageref{def:odwzorowanie_poltoraliniowe})

\[
f \colon \; X \times Y \to \mathbb{C}
\]

\smallskip

\cite[Rozdzia� 3.2.1]{kostrykin2}
\end{definicja}

\bigskip
%%%%%%%%%%%%%%%%%%%%%%%%%%%%%%%%%%%%%%%%%%%%
\subsection{Forma dwuliniowa}

\begin{definicja}
\textbf{\emph{Form� dwuliniow�}}\label{def:forma_dwuliniowa} na iloczynie kartezja�skim (Def. \ref{def:iloczyn_kartezjanski}, str. \pageref{def:iloczyn_kartezjanski}) $X \times Y$ nazywamy \textbf{odwzorowanie dwuliniowe} (Def. \ref{def:odwzorowanie_dwuliniowe}, str. \pageref{def:odwzorowanie_dwuliniowe})

\[
f \colon \; X \times Y \to K
\]

\smallskip

\cite[Rozdzia� 0.2]{trajdos}

\bigskip
%%%%%%%%%%%%%%%%%%%%%%%%%%%%%%%%%%%%%%%%%%%%%%%%%%%%
\subsubsection{Posta� analityczna forma dwuliniowej}

Je�eli $dimX = m$ i~$dimY = n$ (Def. \ref{def:wymiar_przestrzeni}, str. \pageref{def:wymiar_przestrzeni}), to form� mo�emy zapisa� w~postaci analitycznej

\[
\begin{array}{ccl}
f(x,y) = a_{ij} \: x^i \: y^j & \qquad & a_{ij} \in K, \; x \in X, \; y \in Y, \vspace{0.3cm}\\
& & i \in \{1, \ldots, m\}, \; j \in \{1, \ldots, n\}
\end{array}
\]

\smallskip

\cite[Rozdzia� 0.2]{trajdos}

\bigskip
%%%%%%%%%%%%%%%%%%%%%%%%%%%%%%%%%%%%%%%%%%%%%%%%%%%
\subsubsection{Posta� macierzowa forma dwuliniowej}

Je�eli $dimX = m$ i $dimY = n$ (Def. \ref{def:wymiar_przestrzeni}, str. \pageref{def:wymiar_przestrzeni}), to form� mo�emy zapisa� w~postaci macierzowej (Def. \ref{def:macierz}, str. \pageref{def:macierz})

\medskip

\[
f(x,y) = \left[x^1, \dots, x^m\right]
\left[
\begin{array}{ccc}
a_{11} & \ldots & a_{1n} \\
\vdots & \ddots & \vdots \\
a_{m1} & \ldots & a_{mn}
\end{array}
\right]
\left[
\begin{array}{c}
y^1  \\
\vdots \\
y^n
\end{array}
\right]
\]

\medskip

\noindent lub przy oznaczeniach kolumnowych

\begin{itemize}
\item $x = \left[x^1, \dots, x^m\right]$,
\item $x^T$ (Def. \ref{def:macierz_transponowana}, str. \pageref{def:macierz_transponowana}),
\item $y = \left[y^1, \dots, y^n\right]$
\item oraz $A = \left[a_{ij} \right]_{m \times n}$
\end{itemize}

\medskip

\[
f(x,y) = x^T \; A \; y
\]

\medskip

\noindent Macierz $A$ nazywamy \textbf{macierz� formy dwulinowej $f$}.

\smallskip

\cite[Rozdzia� 0.2]{trajdos}
\end{definicja}

\bigskip
%%%%%%%%%%%%%%%%%%%%%%%%%%%%%%
\subsection{Forma hermitowska}

\begin{definicja}
Form� p�toraliniow� (Def. \ref{def:forma_poltoraliniowa}, str. \pageref{def:forma_poltoraliniowa})

\[
f \colon \; X \times X \to \mathbb{C}
\]

\medskip

\noindent nazywamy \textbf{\emph{hermitowsk�}}\label{def:forma_hermitowska}, je�li

\[
f(x,y) \; = \; \overline{f(y,x)}
\]

\smallskip

\cite[Paragraf 1.4.4]{kostrykin2}
\end{definicja}

\bigskip
%%%%%%%%%%%%%%%%%%%%%%%%%%%%%%%%%%%%%%%%%
\subsection{Forma dwuliniowa symetryczna}

\begin{definicja}
Form� dwuliniow� (Def. \ref{def:forma_dwuliniowa}, str. \pageref{def:forma_dwuliniowa})

\[
f \colon \; X \times X \to K
\]

\medskip

\noindent nazywamy \textbf{\emph{symetryczn�}}\label{def:forma_dwuliniowa_symetryczna}, je�li

\[
f(x,y) \; = \; f(y,x)
\]

\smallskip

\cite[Paragraf 1.4.4]{kostrykin2}
\end{definicja}

\bigskip
%%%%%%%%%%%%%%%%%%%%%%%%%%%%%%%%%%%%%%%%%
\subsection{Forma dwuliniowa antysymetryczna}

\begin{definicja}
Form� dwuliniow� (Def. \ref{def:forma_dwuliniowa}, str. \pageref{def:forma_dwuliniowa})

\[
f \colon \; X \times X \to K
\]

\medskip

\noindent nazywamy \textbf{\emph{antysymetryczn�}}\label{def:forma_dwuliniowa_antysymetryczna}, je�li

\[
f(x,y) \; = \; -f(y,x)
\]

\smallskip

\cite[Paragraf 1.4.4]{kostrykin2}
\end{definicja}


\bigskip
%%%%%%%%%%%%%%%%%%%%%%%%%%%%%
\subsection{Forma kwadratowa}

\begin{definicja}
Odwzorowanie $g \colon X \to K$ nazywamy \textbf{\emph{form� kwadratow�}}\label{def:forma_kwadratowa} generowan� przez form� dwuliniow� $f$ je�eli:

\[
\forall x \in X \colon \quad g(x) = f(x,x)
\]

\cite[Definicja 4.1.1]{furdzik}
\end{definicja}

\bigskip
%%%%%%%%%%%%%%%%%%%%%%%%%%%%%%%%%%%%%%%
\subsection{Forma dwuliniowa biegunowa}

\begin{definicja}
Jedyn� form� dwuliniow� (Def. \ref{def:forma_dwulinowa}, str. \pageref{def:forma_dwulinowa}), symetryczn� (Def. \ref{def:forma_dwulinowa_symetryczna}, str. \pageref{def:forma_dwulinowa_symetryczna}) generuj�c� form� kwadratow� (Def. \ref{def:forma_kwadratowa}, str. \pageref{def:forma_kwadratowa}) $g$ nazywamy \textbf{\emph{form� biegunow� dla $g$}}\label{def:forma_dwuliniowa_biegunowa}.

\smallskip

\cite[Definicja 4.1.2]{furdzik}
\end{definicja}

\bigskip
%%%%%%%%%%%%%%%%%%%%%%%%%%%%%%%%%%%%%%%%%%%%
\subsection{Forma wieloliniowa}

\begin{definicja}
\textbf{\emph{Form� wieloliniow�}}\label{def:forma_wieloliniowa} (\textbf{\emph{$n$-liniow�}})\label{def:forma_n-liniowa} na iloczynie kartezja�skim (Def. \ref{def:iloczyn_kartezjanski}, str. \pageref{def:iloczyn_kartezjanski}) $V_1 \times V_2 \times \ldots \times V_n$ nazywamy \textbf{odwzorowanie wieloliniowe} (\textbf{$n$-liniowe}) (Def. \ref{def:odwzorowanie_wieloliniowe}, str. \pageref{def:odwzorowanie_wieloliniowe})

\[
f \colon \; V_1 \times V_2 \times \ldots V_n \to K
\]

\smallskip

\cite[Paragraf 1.4.1]{kostrykin2}
\end{definicja}

\bigskip
%%%%%%%%%%%%%%%%%%%%%%%%%%%%%%%%%%%%%%%%%%%
\subsection{Forma wieloliniowa symetryczna}

\begin{definicja}
Form� wieloliniow� ($n$-liniow�) (Def. \ref{def:forma_wieloliniowa}, str. \pageref{def:forma_wieloliniowa})

\[
f \colon \; X^n \to K
\]

\medskip

\noindent nazywamy \textbf{\emph{symetryczn�}}\label{def:forma_wieloliniowa_symetryczna}, je�li

\[
f(x_{{i_1}1},x_{{i_2}2}, \ldots, x_{{i_n}n}) \; = \; f(x_1,x_2, \ldots, x_n)
\]

\smallskip

\cite[Paragraf 1.4.1]{kostrykin2}
\end{definicja}

\bigskip
%%%%%%%%%%%%%%%%%%%%%%%%%%%%%%%%%%%%%%%%%%%
\subsection{Forma wieloliniowa antysymetryczna}

\begin{definicja}
Form� wieloliniow� ($n$-liniow�) (Def. \ref{def:forma_wieloliniowa}, str. \pageref{def:forma_wieloliniowa})

\[
f \colon \; X^n \to K
\]

\medskip

\noindent nazywamy \textbf{\emph{antysymetryczn�}}\label{def:forma_wieloliniowa_antysymetryczna}, je�li

\[
f(x_{{i_1}1},x_{{i_2}2}, \ldots, x_{{i_n}n}) \; = \; \textsl{sgn}(i_1,i_2, \ldots, i_n) \; f(x_1,x_2, \ldots, x_n)
\]

\medskip

\noindent gdzie sumowanie rozci�ga si� na wszystkie permutacje (Def. \ref{def:permutacja}, str. \pageref{def:permutacja}) $(i_1, i_2, \ldots, i_n)$ zbioru $\{1, 2, \ldots , n\}$.

\smallskip

\cite[Paragraf 1.4.1]{kostrykin2}
\end{definicja}

\bigskip
%%%%%%%%%%%%%%%%%%%%%%%%%%%%%%%%%%%%%%%%%%%%%%%%%%%%%
\section{Przestrze� unitarna. Iloczyn skalarny}

%%%%%%%%%%%%%%%%%%%%%%%%%%%%%
\subsection{Iloczyn skalarny}

\medskip

\begin{definicja}
Niech
\begin{itemize}
\item $V$ b�dzie \textbf{przestrzeni� wektorow�} (Def. \ref{def:przestrzen_wektorowa}, str. \pageref{def:przestrzen_wektorowa}) nad \textbf{cia�em liczb zespolonych} (Def. \ref{def:liczby_zespolone}, str. \pageref{def:liczby_zespolone}) $\mathbb{C}$
\item $g \colon V \rightarrow \mathbb{C}$ pewn� \textbf{form� kwadratow�} (Def. \ref{def:forma_kwadratowa}, str. \pageref{def:forma_kwadratowa}) \textbf{okre�lon� dodatnio}
\end{itemize}

\medskip

\noindent \textbf{Form� hermitowsk�} (Def. \ref{def:forma_hermitowska}, str. \pageref{def:forma_hermitowska})~$f$ generuj�c� form� kwadratow�~$g$ nazywamy \textbf{\emph{iloczynem skalarnym}}\label{def:iloczyn_skalarny_przestrzen_unitarna} okre�lonym w~$V$.

\medskip
\noindent Warto�� formy $f(x,y)$ oznaczamy $(x|y)$ lub $x \circ y$.

\smallskip

\cite[Paragraf 3.2.1]{kostrykin2}
\end{definicja}

\bigskip
%%%%%%%%%%%%%%%%%%%%%%%%%%%%%%%%%%%%%%%%%%%%%
\subsubsection{W�asno�ci iloczynu skalarnego}

\label{wl:iloczyn_skalarny_przestrzen_unitarna}

\[
\begin{array}{lrcl}
1. & \forall x, y \in V & \colon & (x|y)= \overline{(y|x)} \vspace{0.3cm}\\
2. & \forall \alpha \in \mathbb{C} \quad \forall x,y \in V & \colon & (\alpha x|y) = \alpha(x|y) \vspace{0.3cm}\\
3. & \forall x_1, x_2, y \in V & \colon & (x_1 + x_2|y) = (x_1|y) + (x_2|y) \vspace{0.3cm}\\
4. & \forall x \in V & \colon & re(x|x) \geq 0 \qquad \textsl{ oraz } \qquad (x|x) = 0 \Leftrightarrow x = 0
\end{array}
\]

\smallskip

\cite[Rozdzia� 6]{przybylo} \cite[Paragraf 3.2.1]{kostrykin2}

\bigskip
%%%%%%%%%%%%%%%%%%%%%%%%%%%%%%%%%%%
\subsection{Przestrze� unitarna}

\begin{definicja}
Przestrze� $V$ nad cia�em $\mathbb{C}$, w kt�rej okre�lono iloczyn skalarny nazywamy \textbf{\emph{przestrzeni� unitarn�}}\label{def:przestrzen_unitarna}.

\smallskip

\cite[Rozdzia� 6]{przybylo} \cite[Paragraf 3.2.1, Definicja 2]{kostrykin2}
\end{definicja}

\bigskip

%%%%%%%%%%%%%%%%%%%%%%%%%%%%%
\subsection{Norma wektora}

\begin{definicja}
\textbf{\emph{D�ugo�ci�}}\label{def:dlugosc_wektora_unitarna} lub \textbf{\emph{norm�}}\label{def:norma_wektora_unitarna} wektora $v \in V$ nazywamy liczb� rzeczywist� nieujemn�

\[
\|v\| \; = \; \sqrt{(v|v)}
\]

\smallskip

\cite[Rozdzia� 3.2.1]{kostrykin2}
\end{definicja}

\bigskip
%%%%%%%%%%%%%%%%%%%%%%%%%%%%%%%%%%%%%%%%%%%%%%%%%%%%%%%%
\section{Przestrze� euklidesowa. Iloczyn skalarny}

%%%%%%%%%%%%%%%%%%%%%%%%%%%%%%%%%%%%%%%%%%%%%
\subsection{Iloczynu skalarny}

\medskip

\begin{definicja}
Niech
\begin{itemize}
\item $V$ b�dzie \textbf{przestrzeni� wektorow�} (Def. \ref{def:przestrzen_wektorowa}, str. \pageref{def:przestrzen_wektorowa}) nad \textbf{cia�em liczb rzeczywistych} $\mathbb{R}$ (TODO cia�o liczb rzeczywistych?)
\item $g \colon V \rightarrow \mathbb{R}$ pewn� \textbf{form� kwadratow�} (Def. \ref{def:forma_kwadratowa}, str. \pageref{def:forma_kwadratowa}) \textbf{okre�lon� dodatnio}
\end{itemize}

\medskip

\noindent \textbf{Form� dwuliniow� $f$ biegunow�} (Def. \ref{def:forma_dwuliniowa_biegunowa}, str. \pageref{def:forma_dwuliniowa_biegunowa}) dla~$g$ nazywamy \textbf{\emph{iloczynem skalarnym}}\label{def:iloczyn_skalarny_przestrzen_euklidesowa} okre�lonym w~$V$.

\medskip
\noindent Warto�� formy $f(x,y)$ oznaczamy $(x|y)$ lub $x \circ y$.

\smallskip

\cite[Definicja 4.1.6]{furdzik} \cite[Paragraf 3.2.1]{kostrykin2}
\end{definicja}

\bigskip
%%%%%%%%%%%%%%%%%%%%%%%%%%%%%%%%%%%%%%%%%%%%%
\subsubsection{W�asno�ci iloczynu skalarnego}

\label{wl:iloczyn_skalarny_przestrzen_euklidesowa}

\[
\begin{array}{lrcl}
1. & \forall x, y \in V & \colon & (x|y)= (y|x) \vspace{0.3cm}\\
2. & \forall \alpha \in \mathbb{R} \quad \forall x,y \in V & \colon & (\alpha x|y) = \alpha(x|y) \vspace{0.3cm}\\
3. & \forall x_1, x_2, y \in V & \colon & (x_1 + x_2|y) = (x_1|y) + (x_2|y) \vspace{0.3cm}\\
4. & \forall x \in V & \colon & (x|x) \geq 0 \qquad \textsl{ oraz } \qquad (x|x) = 0 \Leftrightarrow x = 0
\end{array}
\]

\smallskip

\cite[Rozdzia� 6]{przybylo} \cite[Paragraf 3.2.1]{kostrykin2}

\bigskip
%%%%%%%%%%%%%%%%%%%%%%%%%%%%%%%%%%%%%%
\subsection{Przestrze� euklidesowa}

TODO kt�ra definicja bardziej prawdziwa?

\begin{definicja}
Przestrze� $V$ nad cia�em $\mathbb{R}$, w kt�rej okre�lono iloczyn skalarny, nazywamy \textbf{\emph{przestrzeni� euklidesow�}}\label{def:przestrzen_euklidesowa}.

\smallskip

\cite[Rozdzia� 6]{przybylo}
\end{definicja}

\begin{definicja}
Przestrze� $\mathbb{R}^n$ z

\begin{itemize}
\item iloczynem skalarnym (Def. \ref{def:iloczyn_skalarny_przestrzen_euklidesowa}, str. \pageref{def:iloczyn_skalarny_przestrzen_euklidesowa})

\[
(x|y) \; = \; \sum_{i=1}^n x_i \: y_i
\]

\item i~norm� wektora (Def. \ref{def:norma_wektora_euklidesowa}, str. \pageref{def:norma_wektora_euklidesowa})

\[
\|v\|=\sqrt{(v|v)}
\]

\end{itemize}

\noindent nazywamy \textbf{\emph{przestrzeni� euklidesow�}} i~oznaczamy $\overrightarrow{E_n}$

\smallskip

\cite[Definicja 4.1.7]{furdzik}
\end{definicja}

\bigskip

%%%%%%%%%%%%%%%%%%%%%%%%%%%%%%%%%%%%%%%%%%%%%%%%%%%%%%%%%%%%%%%%%%%%%%%%%%%%
\subsection{Baza ortogonalna, baza ortonormalna przestrzeni euklidesowej}

\begin{definicja}
Baz� (Def. \ref{def:baza_przestrzeni}, str. \pageref{def:baza_przestrzeni}) 

\[
(e_1, \ldots, e_n)
\]

\noindent przestrzeni euklidesowej $V$ nazywamy \textbf{\emph{ortogonaln�}}\label{def:baza_ortogonalna}, je�li 

\[
(e_i|e_j) = 0
\]

\noindent dla dowolnych $i \neq j \quad (i,j = 1, \ldots, n)$.

\smallskip

\cite[Definicja 4, rozdzia� 3.1]{kostrykin2}
\end{definicja}

\bigskip

\begin{definicja}
Baz� \textbf{ortogonaln�} (Def. \ref{def:baza_ortogonalna}, str. \pageref{def:baza_ortogonalna})

\[
(e_1, \ldots, e_n)
\]

\noindent przestrzeni euklidesowej $V$ nazywamy \textbf{\emph{ortonormaln�}}\label{def:baza_ortonormalna}, je�li

\[
(e_i|e_i) = 1
\]

\noindent dla ka�dego $i$.

\smallskip

\cite[Definicja 4, rozdzia� 3.1]{kostrykin2}
\end{definicja}

\bigskip

%%%%%%%%%%%%%%%%%%%%%%%%%%%%%
\subsection{Norma wektora}

\begin{definicja}
\textbf{\emph{D�ugo�ci�}}\label{def:dlugosc_wektora_euklidesowa} lub \textbf{\emph{norm�}}\label{def:norma_wektora_euklidesowa} wektora $v \in V$ nazywamy liczb� rzeczywist� nieujemn�

\[
\|v\| \; = \; \sqrt{(v|v)}
\]

\smallskip

\cite[Definicja 2, rozdzia� 3.1]{kostrykin2}
\end{definicja}

\bigskip
%%%%%%%%%%%%%%%%%%%%%%%%%%%%%%
\section{Przestrze� afiniczna}

TODO sprawdzic czy $\chi$ czy $X$ jest przestrzeni� afiniczna (jesli $X$ to zmienic w geometrii analitycznej)

\begin{definicja}
Uporz�dkowan� tr�jk�

\[
\chi \; = \; (X, V, +)
\]

\medskip

\noindent nazywa� b�dziemy \textbf{\emph{przestrzeni� afiniczn�}}\label{def:przestrzen_afiniczna} je�eli

\begin{itemize}
\item $X$ b�dzie pewnym zbiorem
\item $V$ - przestrzeni� wektorow� (Def. \ref{def:przestrzen_wektorowa}, str. \pageref{def:przestrzen_wektorowa})
\item $+$ - dzia�aniem zewn�trznym (Def. \ref{def:dzialanie_zewnetrzne}, str. \pageref{def:dzialanie_zewnetrzne}) w~$X$

\[
X \: \times \: Y \; \ni \; (x,v) \quad \rightarrow \quad x + v \; \in \; X
\]

spe�niaj�cym warunki:

\begin{tabular}{m{0,2cm}m{3,8cm}m{0,1cm}m{4,6cm}m{0,1cm}}
 
1. &
$\forall x \in X$ &
$:$ &
$x + \mathbf{0} \; = \; x \qquad (\mathbf{0} \in V)$ &
 \\[10pt]
 
2. &
$\forall x,y \in X \quad \exists v \in V $ &
$:$ &
$x + v \; = \; y$ &
 \\[10pt]
 
3. &
$\forall x \in X \quad \forall v_1, v_2 \in V$ &
$:$ &
$x + v_1 = x + v_2 \; \Rightarrow \; v_1 = v_2$ TODO sprawdzic &
 \\[10pt]
 
4. &
$\forall x \in X \quad \forall u,v \in V$ &
$:$ &
$x + (u + v) \; = \; (x + u) + v$ &
 \\[10pt]

\end{tabular}

\end{itemize}

\smallskip

\cite[Definicja 3.5.1]{furdzik}
\end{definicja}

\bigskip
%%%%%%%%%%%%%%%%%%%%%%%%%%%%%%%%%%%%%%%%%%%
\subsection{Przestrze� wektor�w swobodnych}

\begin{definicja}
Przestrze� wektorow� $V$ przestrzeni afinicznej $\chi$ nazywamy \textbf{przestrzeni� wektor�w swobodnych przestrzeni afinicznej}\label{def:przestrzen_wektorow_swobodnych} $X$ i~oznacza� b�dziemy

\[
\vec{X}
\]

\smallskip

TODO sprawdzi� 'przestrzeni afinicznej $X$'

\cite[Definicja 3.5.1]{furdzik}
\end{definicja}

\bigskip
%%%%%%%%%%%%%%%%%%%%%%%%%%%%
\subsection{R�nica punkt�w}

\begin{definicja}
\textbf{\emph{R�nic� punkt�w}}\label{def:roznica_punktow} $x$ i $y$ ($x,y \in X$) lub \textbf{\emph{wektorem ��cz�cym punkty}} $x$ i $y$ nazywamy \textbf{jedyny wektor} spe�niaj�cy aksjomat~2 definicji przestrzeni afinicznej (Def. \ref{def:przestrzen_afiniczna}, str. \pageref{def:przestrzen_afiniczna}).

\medskip

Oznaczmy

\[
y - x \qquad \textsl{lub} \qquad \overrightarrow{xy}
\]

\smallskip

\cite[Definicja 3.5.1]{furdzik}
\end{definicja}

\bigskip
%%%%%%%%%%%%%%%%%%%%%%%%%%%%%%%%
\subsection{Uk�ad wsp�rz�dnych}

\begin{definicja}
\textbf{\emph{Uk�adem wsp�rz�dnych}}\label{def:uklad_wspolrzednych} w~$n$-wymiarowej (Def. \ref{def:wymiar_przestrzeni}, str. \pageref{def:wymiar_przestrzeni}) przestrzeni afinicznej $\chi$ (Def. \ref{def:przestrzen_afiniczna}, str. \pageref{def:przestrzen_afiniczna}) nazywamy par�

\[
(o; \; e_1, \ldots, e_n)
\]

\noindent z�o�on� z

\begin{itemize}
\item punktu $o \in X$
\item i~bazy (Def. \ref{def:baza_przestrzeni}, str. \pageref{def:baza_przestrzeni}) $(e_1, \ldots, e_n)$ w $V$
\end{itemize}

\smallskip

\cite[Paragraf 4.1.3, Definicja 3]{kostrykin2}
\end{definicja}

\bigskip

\begin{definicja}
\textbf{\emph{Wsp�rz�dnymi}} 

\[
x_1, \ldots, x_n
\]

\noindent \textbf{\emph{punktu}}\label{def:wspolrzedne_punktu} $p$ w~uk�adzie $(o; \; e_1, \ldots, e_n)$ nazywamy wsp�rz�dne wektora $\overrightarrow{op}$ w~bazie $(e_1, \ldots, e_n)$

\[
\overrightarrow{op} \; = \; x_1 e_1 + \ldots + x_n e_n
\]

\smallskip

\cite[Paragraf 4.1.3, Definicja 3]{kostrykin2}
\end{definicja}




\chapter{Algebra Boole'a}

\begin{definicja}
\textbf{\emph{Algebra Boole'a}}\index{algebra!Boole'a} to struktura algebraiczna (Def. \ref{def:struktura_algebraiczna}, str. \pageref{def:struktura_algebraiczna})

\[
\mathbb{B} = \left(\mathbf{B}, \cup, \cap, \overline{\phantom{0}}, 0, 1\right),
\]

\noindent w kt�rej

\begin{itemize}
\item $\cup$ i $\cap$ s� dzia�aniami dwuargumentowymi,
\item $\overline{\phantom{0}}$ jest dzia�aniem jednoargumentowym,
\item a $0$ i $1$ s� \emph{wyr�nionymi, r�nymi} elementami zbioru $\mathbf{B}$,
\end{itemize}

\medskip

\noindent spe�niaj�ca nast�puj�ce warunki:

\begin{enumerate}
\item dzia�anie $\cup$ jest \textbf{przemienne} (Def. \ref{def:dzialanie_przemienne}, str. \pageref{def:dzialanie_przemienne})
\item dzia�anie $\cup$ jest \textbf{��czne} (Def. \ref{def:dzialanie_laczne}, str. \pageref{def:dzialanie_laczne})
\item aksjomat Huntingtona\index{aksjomat!Huntingtona}: 

\[
\forall \: x, y \in B \qquad \Rightarrow \qquad \overline{\left( \overline{x} \cup \overline{y} \right)} \cup \overline{\left(\overline{x} \cup y \right)} \; = \; x
\]
\end{enumerate}
\end{definicja}

\medskip

\begin{definicja}
\textbf{\emph{Elementem jednostkowym}}\label{def:element_jednostkowy} nazywamy element $\mathbf{1}$ taki, �e:

\[
\forall \: x \in B \colon \qquad x \cup \overline{x} \; = \; \mathbf{1}
\]
\end{definicja}

\medskip

\begin{definicja}
\textbf{\emph{Elementem zerowym}}\label{def:element_zerowy} nazywamy element $\mathbf{0}$ taki, �e:

\[
\forall \: x \in B \colon \qquad \overline{x \cup \overline{x}} \; = \; \mathbf{0}
\]
\end{definicja}

\medskip

\begin{definicja}
\textbf{\emph{Dzia�anie $\cap$}}:
\[
\forall \; x,y \in B \colon \qquad x \cap y \; = \; \overline{\overline{x} \cup \overline{y}}
\]
\end{definicja}

\bigskip

%%%%%%%%%%%%%%%%%%%%%%%%%%%
\section{Tabela dzia�a�}

$\alpha$, $\beta$ - zdania (formy zdaniowe, kt�rym mo�na przypisa� warto��)

\medskip

\begin{tabular}{|c|c|c|c|c|c|c|}
\hline
$\alpha$ & $\beta$ & $\overline{\alpha}$ & $\alpha \vee \beta$ & $\alpha \wedge \beta$ & $\alpha \Rightarrow \beta$ & $\alpha \Leftrightarrow \beta$ \\
\hline
0 & 0 & 1 & 0 & 0 & 1 & 1 \\
0 & 1 & 1 & 1 & 0 & 1 & 0 \\
1 & 0 & 0 & 1 & 0 & 0 & 0 \\
1 & 1 & 0 & 1 & 1 & 1 & 1 \\
\hline
\end{tabular}

\bigskip
%%%%%%%%%%%%%%%%%%%%%%%%%%%%%%%%%%%%%%%%%%%%
\section{Twierdzenia dla Algebry Boole'a}

\begin{twierdzenie}[o unikalno�ci]
Jest \textbf{tylko jeden} element jednostkowy~$\mathbf{1}$ (Def. \ref{def:element_jednostkowy}, str. \pageref{def:element_jednostkowy}).

Jest \textbf{tylko jeden} element zerowy~$\mathbf{0}$ (Def. \ref{def:element_zerowy}, str. \pageref{def:element_zerowy}).
\end{twierdzenie}

\medskip

\begin{twierdzenie}[o dope�nianiu]
\[
x \cup \overline{x} \; = \; \mathbf{1}
\]
\[
x \cap \overline{x} \; = \; \mathbf{0}
\]
\end{twierdzenie}

\medskip

\begin{twierdzenie}[o podw�jnej negacji]
\[
\overline{\overline{x}} \; = \; x
\]
\end{twierdzenie}

\medskip

\begin{twierdzenie}[prawa De Morgana]
\[
\overline{x \cup y} \; = \; \overline{x} \cap \overline{y}
\]
\[
\overline{x \cap y} \; = \; \overline{x} \cup \overline{y}
\]
\end{twierdzenie}

\medskip

\begin{twierdzenie}
\[
x \cap x \; = \; x
\quad
x \cup x \; = \; x
\quad
x \cup \mathbf{0} \; = \; x
\quad
x \cup \mathbf{1} \; = \; \mathbf{1}
\quad
x \cap \mathbf{1} \; = \; x
\quad
x \cap \mathbf{0} \; = \; \mathbf{0}
\]
\end{twierdzenie}

\medskip

\begin{twierdzenie}[o rozdzielno�ci]
\[
x \cup \left(y \cap z \right) \; = \; \left(x \cup y \right) \cap \left(x \cup z \right)
\]
\[
x \cap \left(y \cup z\right) \; = \; \left(x \cap y\right) \cup \left(x \cap z\right)
\]
\end{twierdzenie}

\bigskip
%%%%%%%%%%%%%%%%%%%%%%%%%%%%%%%%%%%%%%%%%%%%%%%%%%%%%%%%
\section{Funktory w elektrotechnice. Bramki logiczne}

\smallskip
%%%%%%%%%%%%%%%%%%%%%%%%%%
\subsection{Bramka NOT}
\index{bramka!NOT}

\begin{tabular}{m{6cm}m{7cm}m{0.1cm}}
\includegraphics[width=4cm]{img/gates-not.png} &
\begin{tabular}{|m{1cm}|m{1cm}|m{0.1cm}}
\cline{1-2} $x$ & $\overline{x}$ &\\[1em]
\cline{1-2} 0 & 1 &\\[0.3em]
1 & 0 &\\[0.3em]
\cline{1-2}
\end{tabular}
& \\
\end{tabular}


\bigskip
%%%%%%%%%%%%%%%%%%%%%%%%%
\subsection{Bramka OR}
\index{bramka!OR}

\begin{tabular}{m{6cm}m{7cm}m{0.1cm}}
\includegraphics[width=4cm]{img/gates-or.png} &
\begin{tabular}{|m{1cm}|m{1cm}|m{1cm}|m{0.1cm}}
\cline{1-3} $x$ & $y$ & $x \vee y$ &\\[1em]
\cline{1-3} 0 & 0 & 0 &\\[0.3em]
0 & 1 & 1 &\\
1 & 0 & 1 &\\
1 & 1 & 1 &\\[0.3em]
\cline{1-3}
\end{tabular}
& \\
\end{tabular}

\bigskip
%%%%%%%%%%%%%%%%%%%%%%%%%%
\subsection{Bramka NOR}
\index{bramka!NOR}

\begin{tabular}{m{6cm}m{7cm}m{0.1cm}}
\includegraphics[width=4cm]{img/gates-nor.png} &
\begin{tabular}{|m{1cm}|m{1cm}|m{1cm}|m{0.1cm}}
\cline{1-3} $x$ & $y$ & $\overline{x \vee y}$ &\\[1em]
\cline{1-3} 0 & 0 & 1 &\\[0.3em]
0 & 1 & 0 &\\
1 & 0 & 0 &\\
1 & 1 & 0 &\\[0.3em]
\cline{1-3}
\end{tabular}
& \\
\end{tabular}


\bigskip
%%%%%%%%%%%%%%%%%%%%%%%%%%
\subsection{Bramka AND}
\index{bramka!AND}

\begin{tabular}{m{6cm}m{7cm}m{0.1cm}}
\includegraphics[width=4cm]{img/gates-and.png} &
\begin{tabular}{|m{1cm}|m{1cm}|m{1cm}|m{0.1cm}}
\cline{1-3} $x$ & $y$ & $x \wedge y$ &\\[1em]
\cline{1-3} 0 & 0 & 0 &\\[0.3em]
0 & 1 & 0 &\\
1 & 0 & 0 &\\
1 & 1 & 1 &\\[0.3em]
\cline{1-3}
\end{tabular}
& \\
\end{tabular}

\bigskip
%%%%%%%%%%%%%%%%%%%%%%%%%%%
\subsection{Bramka NAND}
\index{bramka!NAND}

\begin{tabular}{m{6cm}m{7cm}m{0.1cm}}
\includegraphics[width=4cm]{img/gates-nand.png} &
\begin{tabular}{|m{1cm}|m{1cm}|m{1cm}|m{0.1cm}}
\cline{1-3} $x$ & $y$ & $\overline{x \wedge y}$ &\\[1em]
\cline{1-3} 0 & 0 & 1 &\\[0.3em]
0 & 1 & 1 &\\
1 & 0 & 1 &\\
1 & 1 & 0 &\\[0.3em]
\cline{1-3}
\end{tabular}
& \\
\end{tabular}

\bigskip
%%%%%%%%%%%%%%%%%%%%%%%%%%
\subsection{Bramka XOR}
\index{bramka!XOR}

\begin{tabular}{m{6cm}m{7cm}m{0.1cm}}
\includegraphics[width=4cm]{img/gates-xor.png} &
\begin{tabular}{|m{1cm}|m{1cm}|m{1cm}|m{0.1cm}}
\cline{1-3} $x$ & $y$ & $x \underline{\vee} y$ &\\[1em]
\cline{1-3} 0 & 0 & 0 &\\[0.3em]
0 & 1 & 1 &\\
1 & 0 & 1 &\\
1 & 1 & 0 &\\[0.3em]
\cline{1-3}
\end{tabular}
& \\
\end{tabular}


\chapter{Cia�o liczb zespolonych}

\medskip

Literatura do tego dzia�u: \cite{przybylo} - 2, \cite{ptak} - 3.3, \cite{kostrykin1} - 5.1\newline
Zadania do teog dzia�u: \cite{krysicki1} - 8.1, \cite{przybylo} - 2, \cite{ptak} - 3

\begin{definicja}
Cia�em \textbf{\emph{liczb zespolonych}}\label{def:liczby_zespolone}\label{def:liczba_zespolona} nazywamy cia�o (Def. \ref{def:cialo}, str. \pageref{def:cialo}) $(C, \oplus, \odot)$, w kt�rym $C = \mathbb{R} \times \mathbb{R}$, a dzia�ania $\oplus$ oraz $\odot$ s� okre�lone nast�puj�co:
\begin{center}
\begin{enumerate}
\item $\forall a, b, c, d \in \mathbb{R} \colon \qquad (a, b) \oplus (c, d) = (a + b, c + d)$
\item $\forall a, b, c, d \in \mathbb{R} \colon \qquad (a, b) \odot (c, d) = (ac - bd, ad + bc)$
\end{enumerate}
\end{center}
\cite[Rozdzia� 2]{przybylo}
\end{definicja}

\bigskip

W�a�ciwo�ci:
\begin{itemize}
\item elementem neutralnym (Def. \ref{def:element_neutralny}, str. \pageref{def:element_neutralny}) dzia�ania $\oplus$ jest liczba $\mathbf{(0,0)}$  
\item elementem neutralnym dzia�ania $\odot$ jest liczba $\mathbf{(1,0)}$
\item elementem przeciwnym (Def. \ref{def:element_przeciwny}, str. \pageref{def:element_przeciwny}) do liczby $(a, b)$ jest liczba
\[
-(a,b) = (-a, -b)
\]
\item elementem odwrotnym (Def. \ref{def:element_odwrotny}, str. \pageref{def:element_odwrotny}) do liczby $(a, b)$ jest liczba 
\[
(a,b)^{-1} = \left(\dfrac{a}{a^2 + b^2}, \dfrac{-b}{a^2 + b^2} \right)
\]
\end{itemize}

\bigskip
\begin{definicja}
\textbf{\emph{Cz�ci� rzeczywist�}}\label{def:czesc_rzeczywista} liczby zespolonej $z = (a, b)$ nazywamy liczb� rzeczywist� $a$ i oznaczamy
\[
Re \: z \quad = \quad a
\]
\end{definicja}

\medskip

\begin{definicja}
\textbf{\emph{Cz�ci� urojon�}}\label{def:czesc_urojona} liczby zespolonej $z = (a, b)$ nazywamy liczb� rzeczywist� $b$ i oznaczamy
\[
Im \: z \quad = \quad b
\]
\end{definicja}

\bigskip

\begin{definicja}
\textbf{\emph{Odejmowaniem}} liczb zespolonych nazywamy dodawanie pierwszego argumentu dzia�ania i elementu przeciwnego do drugiego argumentu dzia�ania.
\[
(a, b) \oplus -(c, d)
\]

\smallskip
co oznaczamy
\[
(a, b) \ominus (c, d)
\]
\end{definicja}

\bigskip
Liczba $(x, y)$ jest wynikiem odejmowania $(a, b) \ominus (c, d)$, gdy:
\[
\begin{array}{rcl}
(a, b) \ominus (c, d) & = & (x, y) \\ 
(a, b) \oplus -(c, d) & = & (x, y) \\
(a, b) \oplus (-c, -d) & = & (x, y)
\end{array}
\]

\medskip

Z definicji dodawania i r�wno�ci liczb zespolonych wynika, �e $a - c = x$ i $b - d = y$, st�d
\[
(a, b) \ominus (c, d) = (a - c, b -d)
\]

\bigskip
\begin{definicja}
\textbf{\emph{Dzieleniem}} liczb zespolonych nazywamy mno�enie pierwszego argumentu i elementu odwrotnego do drugiego argumentu dzia�ania.
\[
(a,b) \odot (c, d)^{-1}
\]
\smallskip
co oznaczamy
\[
\cdot\dfrac{(a,b)}{(c,d)}
\]
\end{definicja}

\bigskip
Lizcba $(x, y)$ jest wynikiem dzielenia $\cdot\dfrac{(a,b)}{(c,d)}$, gdy:
\[
\begin{array}{rcl}
\cdot\dfrac{(a,b)}{(c,d)} & = & (x, y) \vspace{0.2cm} \\
(a,b) \odot (c, d)^{-1} & = & (x, y) \vspace{0.2cm} \\
(a,b) \odot \left(\dfrac{c}{c^2 + d^2}, \dfrac{-d}{c^2 + d^2} \right) &=& (x,y)
\end{array}
\]

\medskip
Z definicji mno�enia i r�wno�ci liczb zespolonych wynika, �e:
\[
\left\{
\begin{array}{rlc}
a \cdot \dfrac{c}{c^2 + d^2} - b \cdot \dfrac{-d}{c^2 + d^2} &=& x
\vspace{0.5cm}
\\

a \cdot \dfrac{-d}{c^2 + d^2} + b \cdot \dfrac{c}{c^2 + d^2} &=& y
\end{array}
\right.
\]

\smallskip
st�d
\[
\cdot\dfrac{(a,b)}{(c,d)} = 
\left(
\dfrac{ac +bd}{c^2 + d^2},
\dfrac{bc -ad}{c^2 + d^2}
\right)
\]

\bigskip
%%%%%%%%%%%%%%%%%%%%%%%%%%%%%%%%%%%%
\section{Interpretacja geometryczna}

\medskip
Liczby zespolone i dzia�ania na nich mo�na interpretowa� geometrycznie. Wykorzystuj�c twierdzenie z geometrii analitycznej o istnieniu wzajemnie jednoznacznej odpowiednio�ci mi�dzy punktami p�aszczyzny i uporz�dkowanymi parami ortokartezja�skich wsp�rz�dnych punktu b�dziemy \textbf{\emph{liczb� zespolon�}} $\mathbf{(a, b)}$ interpretowa� jako \textbf{\emph{punkt}} o wsp�rz�dnych $\mathbf{a}$ i $\mathbf{b}$. Ka�dej wi�c liczbie zespolonej odpowiada dok�adnie jeden punkt p�aszczyzny, zwanej wtedy \textbf{\emph{p�aszczyzn� zespolon�}}.

\smallskip
Osie uk�adu nazywa� b�dziemy
\begin{itemize}
\item \textbf{\emph{osi� rzeczywist�}}\label{def:os_rzeczywista} - o�, na kt�rej le�� punkty odpowiadaj�ce liczbom zespolonym o cz�ci urojonej (Def. \ref{def:czesc_urojona}, str. \pageref{def:czesc_urojona}) r�wnej zeru
\item \textbf{\emph{osi� urojon�}}\label{os_urojona} - o�, na kt�rej le�� punkty odpowiadaj�ce liczbom zespolonym o cz�ci rzeczywistej (Def. \ref{def:czesc_rzeczywista}, str. \pageref{def:czesc_rzeczywista}) r�wnej zeru
\end{itemize}

\medskip
Analogicznie jak to czynili�my w uk�adzie ortokartezja�skim na p�aszczy�nie euklidesowej punktowi $z = a + bi$ b�dziemy przypisywa� wektor wodz�cy punktu, oznaczany tak�e przez $z$
\[
z = a \cdot \mathbf{1} + b \cdot i
\]
\smallskip
gdzie przez $\mathbf{1}$ oznaczyli�my wersor osi rzeczywistej, a przez $i$ wersor osi urojonej.
\cite[Rozdzia� 1.2]{trajdos}

\includegraphics[scale=1.5]{img/zespolone-plaszczyzna_zespolona.png}

\bigskip
%%%%%%%%%%%%%%%%%%%%%%%%%%%%%
\section{Posta� algebraiczna}

\medskip

\begin{definicja}
Par� $(0, 1)$ oznaczamy symbolem $i$ oraz nazywamy \textbf{\emph{jednostk� urojon�}}\label{def:jednostka_urojona}.
\cite[Rozdzia� 2]{przybylo}
\end{definicja}

\bigskip

\begin{definicja}
Ka�d� liczb� zespolon� $(a, b)$ mo�na zapisa� w postaci $\mathbf{z = a + b\mathit{i}}$ (gdzie $i$ to jednostka urojona). Zapis taki nazywamy \textbf{\emph{postaci� algebraiczn�}}\label{def:postac_algebraiczna} liczby zespolonej.
\cite[Rozdzia� 2]{przybylo}
\end{definicja}

\bigskip
%%%%%%%%%%%%%%%%%%%%%%
\subsection{Dodawanie i odejmowanie}
\medskip
Dodawanie dw�ch liczb $z_1, z_2 \in \mathbb{C}$ postaci $z_1 = a_1 + b_1i$ oraz $z_2 = a_2 + b_2i$ mo�emy zapisa� w nast�puj�cy spos�b
\[
z_1 \oplus z_2 = \left( a_1 + a_2 \right) + \left( b_1 + b_2 \right)i
\]

\smallskip
a odejmowanie
\[
z_1 \ominus z_2 = \left( a_1 - a_2 \right) + \left( b_1 - b_2 \right)i
\]

\medskip
Dodawanie liczb zespolonych interpretujemy geometrycznie jako dodawanie przyporz�dkowanych im wektor�w, za� odejmowanie jako odejmowanie wektor�w
\cite[Rozdzia� 1.2]{trajdos}

\includegraphics[scale=1.5]{img/zespolone-dodawanie_odejmowanie.png}

\bigskip
%%%%%%%%%%%%%%%%%%%%%%
\subsection{Mno�enie}
\medskip
Mno�enie dw�ch liczb $z_1, z_2 \in \mathbb{C}$ postaci $z_1 = a_1 + b_1i$ oraz $z_2 = a_2 + b_2i$ mo�emy zapisa� w nast�puj�cy spos�b
\[
z_1 \odot z_2 = \left(a_1a_2 - b_1b_2\right) + \left(a_1b_2 + b_1a_2\right)i 
\]

\bigskip
%%%%%%%%%%%%%%%%%%%%%%
\subsection{Dzielenie}
\medskip
Dzielenie dw�ch liczb $z_1, z_2 \in \mathbb{C}$ postaci $z_1 = a_1 + b_1i$ oraz $z_2 = a_2 + b_2i$ mo�emy zapisa� w nast�puj�cy spos�b
\[
\cdot\dfrac{z_1}{z_2} = \dfrac{a_1a_2 + b_1b_2}{a_2^2 + b_2^2} + \dfrac{b_1a_2 - a_1b_2}{a_2^2 + b_2^2}i 
\]

\bigskip
%%%%%%%%%%%%%%%%%%%%%%
\section{Modu� liczby zespolonej}
\medskip
\begin{definicja}
\textbf{\emph{Modu�em}}\label{def:modul_liczby_zespolonej} liczby zespolonej $z = a + bi$ nazywamy rzeczywist� liczb� nieujemn�
\[
|z| \quad = \quad \sqrt{a^2 + b^2}
\]
\end{definicja}
\medskip
Modu� liczby zespolonej interpretujemy geometrycznie jako d�ugo�� wektora wodz�cego punktu odpowiadaj�cego tej liczbie i oznaczamy $r$
\[
r \quad = \quad |z|
\]

\includegraphics[scale=1.5]{img/zespolone-modul.png}

\bigskip
%%%%%%%%%%%%%%%%%%%%%%
\section{Liczba zespolona sprz�ona}
\medskip
\begin{definicja}
\textbf{\emph{Liczb� sprz�on�}}\label{def:liczba_zespolona_sprzezona} z~liczb� 

\[
z \; = \; a + bi
\]

\noindent nazywamy liczb� 

\[
a - bi
\]

\noindent i~oznaczamy $\overline{z}$

\[
\overline{z} \; = \; a - bi
\]
\cite[Rozdzia� 1.1]{trajdos}
\end{definicja}

\medskip

\begin{definicja}
Dwie liczby, z kt�rach jedna jest sprz�ona z drug�, nazywamy \textbf{\emph{liczbami sprz�onymi}}.
\end{definicja}

\medskip
W�asno�ci:
\begin{itemize}
\item Liczby sprz�one maj� r�wne modu�y
\[
|z| \quad = \quad \left|\overline{z}\right|
\]
\item Ilocznyn liczb sprz�onych jest r�wny kwadratowi ich wsp�lnego modu�u
\[
z \cdot \overline{z} \quad = \quad |z|^2
\] 
\end{itemize}

\medskip
Geometrycznie liczba sprz�ona z liczb� $z$ jest symetrycznym odbiciem wzgl�dem osi rzeczywistej (Def. \ref{def:os_rzeczywista}, str. \pageref{def:os_rzeczywista}).

\includegraphics[scale=1.5]{img/zespolone-liczba_sprzezona.png}

\bigskip
%%%%%%%%%%%%%%%%%%%%%%
\section{Posta� trygonometryczna}
\medskip


\bigskip
%%%%%%%%%%%%%%%%%%%%%%
\subsection{Argument liczby zespolonej}
\medskip
\begin{definicja}
\textbf{\emph{Argumentem}}\label{def:argument_liczby_zespolonej} liczby zespolonej $z = a + bi \: \neq \mathbf{0}$ nazywamy ka�d� liczb� rzeczywist� $\varphi$ spe�niaj�c� warunki
\[
\left\{
\begin{array}{rcl}
cos\: \varphi &=& \dfrac{a}{|z|} \\
sin\: \varphi &=& \dfrac{b}{|z|}
\end{array}
\right.
\]
gdzie $|z|$ jest modu�em (Def. \ref{def:modul_liczby_zespolonej}, str. \pageref{def:modul_liczby_zespolonej}) liczby zespolonej $z$.
\medskip
Argument liczby zespolonej $z$ oznaczamy
\[
Arg \: z
\]
\cite[Rozdzia� 1.2]{trajdos}
\end{definicja}

\medskip
\begin{definicja}
\textbf{\emph{Argumentem g��wnym}} liczby zespolonej $z$ nazywamy ten sposr�d argument�w liczby zespolonej $z$, kt�ry nale�y do przedzia�u $\langle 0, 2\pi)$ i oznaczamy
\[
arg \: z
\]
\cite[Rozdzia� 1.2]{trajdos}
\end{definicja}

\medskip
Pomi�dzy argumentem g��wny a argumentem liczby zespolonej $z$ zachodzi zale�no��
\[
Arg \: z \quad = \quad arg \: z + 2k\pi, \qquad k = 0, \pm1, \ldots
\]
\medskip

Dla liczby $(0, 0)$ nie okre�la si� argumentu.

\medskip
Geometrycznie, argument liczby zespolonej jest miar� wzgl�dn� k�ta, jaki tworzy wektor wodz�cy punktu $z$ z osi� rzeczywist� (Def. \ref{def:os_rzeczywista}, str. \pageref{def:os_rzeczywista}).

\includegraphics[scale=1.5]{img/zespolone-argument.png}

\bigskip
%%%%%%%%%%%%%%%%%%%%%%
\subsection{Posta� trygonometryczna liczby zespolonej}
\medskip
\begin{definicja}
Ka�d� liczb� zespolon� mo�na zapisa� w postaci
\[
r \: \left( cos \: \varphi + i \: sin \: \varphi \right)
\]
zwanej \textbf{\emph{postaci� trygonometryczn�}} liczby zespolonej. Czynnik $r$ jest modu�em (Def. \ref{def:modul_liczby_zespolonej}, str. \pageref{def:modul_liczby_zespolonej}) liczby za� $\varphi$ jest dowolnym jej argumentem (Def. \ref{def:argument_liczby_zespolonej}, str. \pageref{def:argument_liczby_zespolonej}).
\end{definicja}

\bigskip
%%%%%%%%%%%%%%%%%%%%%%
\subsection{Mno�enie}
\medskip
Niech b�d� dane dwie liczby zespolone w postaci trygonometrycznej
\[
z_1 \quad = \quad r_1 \: \left(cos \: \varphi_1 + i \: sin \: \varphi_1 \right)
\]
\[
z_2 \quad = \quad r_2 \: \left(cos \: \varphi_2 + i \: sin \: \varphi_2 \right)
\]

\medskip
Ich iloczynem b�dzie liczba
\[
\begin{array}{rcl}
z_1 z_2 & = &
\left[
r_1 \: \left(cos \: \varphi_1 + i \: sin \: \varphi_1 \right)
\right]
\left[
r_2 \: \left(cos \: \varphi_2 + i \: sin \: \varphi_2 \right)
\right] = \\
& = & r_1 r_2
\left[
\left(
cos \: \varphi_1
cos \: \varphi_2
-
sin \: \varphi_1
sin \: \varphi_2
\right)
+
i
\left(
cos \: \varphi_1
sin \: \varphi_2
+
sin \: \varphi_1
cos \: \varphi_2
\right)
\right]

\end{array}
\]

\medskip
lub po zastosowaniu wzor�w trygonometrycznych
\[
z_1 z_2 = r_1 r_2
\left[
cos \left(\varphi_1 + \varphi_2 \right)
+ i \:
sin \left(\varphi_1 + \varphi_2 \right)
\right]
\]

\medskip
Wnioski:
\begin{itemize}
\item modu� iloczynu liczb zespolonych jest r�wny iloczynowi ich modu��w
\[
\left| z_1 z_2 \right| = \left|z_1\right| \left|z_2\right|
\]
\item argument iloczynu liczb zespolonych jest r�wny sumie ich argument�w
\[
Arg \left(z_1 z_2 \right) = Arg \: z_1 + Arg \: z_2
\]
\end{itemize}

\bigskip
%%%%%%%%%%%%%%%%%%%%%%
\subsection{Dzielenie}
\medskip
Niech b�d� dane dwie liczby zespolone w postaci trygonometrycznej
\[
z_1 \quad = \quad r_1 \: \left(cos \: \varphi_1 + i \: sin \: \varphi_1 \right)
\]
\[
z_2 \quad = \quad r_2 \: \left(cos \: \varphi_2 + i \: sin \: \varphi_2 \right)
\]

\medskip
Ich ilorazem b�dzie liczba
\[
\begin{array}{rcl}
\cdot \dfrac{z_1}{z_2} & = &
\dfrac{
r_1 \: \left(cos \: \varphi_1 + i \: sin \: \varphi_1 \right)
}
{
r_2 \: \left(cos \: \varphi_2 + i \: sin \: \varphi_2 \right)
} = \vspace{0.5cm}\\
& = &
\dfrac{r_1}{r_2}
\dfrac{\left(cos \: \varphi_1 + i sin \: \varphi_1 \right) \left(cos \: \varphi_2 - i sin \: \varphi_2 \right)}{cos^2 \: \varphi_2 + cos^2 \: \varphi_2}

\end{array}
\]

\medskip
lub po zastosowaniu wzor�w trygonometrycznych
\[
\cdot \dfrac{z_1}{z_2} = \dfrac{r_1}{r_2}
\left[
cos \left(\varphi_1 - \varphi_2 \right)
+ i \:
sin \left(\varphi_1 - \varphi_2 \right)
\right]
\]

\medskip
Wnioski:
\begin{itemize}
\item modu� ilorazu liczb zespolonych jest r�wny ilorazowi ich modu��w
\[
\left| \cdot \dfrac{z_1}{z_2} \right| = \dfrac{\left|z_1\right|}{\left|z_2\right|}
\]
\item argument ilorazu liczb zespolonych jest r�wny r�nicy ich argument�w
\[
Arg \: \cdot \dfrac{z_1}{z_2} = Arg \: z_1 - Arg \: z_2
\]
\end{itemize}


\bigskip
%%%%%%%%%%%%%%%%%%%%%%
\subsection{Pot�gowanie}
\medskip
Niech b�dzie dana liczba zespolona w postaci trygonometrycznej
\[
z \quad = \quad r \: \left(cos \: \varphi + i \: sin \: \varphi\right)
\]

\medskip
Jej pot�g� o wyk�adniku naturalnym b�dzie
\[
z^n \quad = \quad r^n \: \left(cos \: n\varphi + i \: sin \: n\varphi\right)
\]
gdzie $n \in \mathbb{N}$

\bigskip
\subsubsection{Wyprowadzanie wzor�w trygonometrycznych}
Niech b�dzie dana liczba zespolona $z$ o module r�wnym 1 ($|z| = 1$)
\[
z \quad = \quad cos \: \varphi + i \: sin \: \varphi
\]
\medskip
Chcemy policzy� $z^2$. \linebreak
\medskip
Ze wzoru na pot�g� liczby zespolonej w postaci trygonometrycznej mamy
\[
z^2 \quad = \quad cos \: 2\varphi + i \: sin \: 2\varphi
\]
\medskip
Mo�emy tak�e skorzysta� z takiego r�wnania
\[
\begin{array}{rcl}
z^2 & = & \left(cos \: \varphi + i \: sin \: \varphi\right)^2 = \\
& = & cos^2 \: \varphi - sin^2 \: \varphi + i \: 2 \: sin \: \varphi cos \: \varphi
\end{array}
\]
\medskip
Z r�wno�ci liczb zespolonych wynika, �e
\[
\left\{
\begin{array}{rcl}
cos \: 2\varphi & = & cos^2 \: \varphi - sin^2 \: \varphi \\
sin \: 2\varphi & = & 2 \: sin \: \varphi cos \: \varphi
\end{array}
\right.
\]
\medskip
Wyprowadzili�my w ten spos�b wzory na $cos \: 2\varphi$ i $sin \: 2\varphi$.
W og�lno�ci, dla dowolnego $n$, otrzymyjemy wz�r Moivre'a.
\medskip
\begin{definicja}
\textbf{\emph{Wzorem Moivre'a}} nazywamy r�wnanie
\[
\left(
cos \: \varphi + i \: sin \: \varphi \right)^n
\quad 
=
\quad
cos\left(n\varphi\right) + i \: sin\left(n\varphi\right)
\]
\end{definicja}


\bigskip
%%%%%%%%%%%%%%%%%%%%%%
\subsection{Pierwiastkowanie}
\medskip
\begin{definicja}
Je�eli $z_0 = r_0\left(cos \: \varphi_0 + i \: sin \: \varphi_0\right) \: \neq \: 0$, przy czym $\varphi_0 = arg \: z_0$, to liczba $\omega = R \left(cos \: \psi + i \: sin \: \psi \right)$ jest \textbf{\emph{pierwiastkiem stopnia $n$}} z $z_0$ wtedy i tylko wtedy, gdy
\[
R^n \quad = \quad r_0
\]

oraz
\[
n \: \psi \quad = \quad \varphi_0 + 2k\pi
\]

gdzie $k$ jest liczb� ca�kowit�.
\cite[Rozdzia� 1.3]{trajdos}
\end{definicja}


\bigskip
St�d
\[
R \quad = \quad \sqrt[n]{r_0}
\]

oraz
\[
\psi \quad = \quad \dfrac{\varphi_0}{n} + \dfrac{2\pi}{n}k, \qquad k = 0, \pm1, \pm2, \ldots
\]

\bigskip
Wszystkie r�ne pierwiastki stopnia $n$ z liczby $z_0$ mo�na zapisa� wzorem
\[
\omega_k \quad = \quad \sqrt[n]{r_0}
\left[
  cos \left(
    \dfrac{\varphi_0}{n} + \dfrac{2\pi}{n}k
      \right)
  +
  i \:
  sin \left(
    \dfrac{\varphi_0}{n} + \dfrac{2\pi}{n}k
      \right)
\right]
\]

\bigskip
Na rysunku zosta�o przedstawionych $n$ pierwiastk�w stopnia $n$ liczby $z_0 \neq 0$. Odcinki przerywane ��cz�ce kolejne pierwiastki tworz� wielobok foremny wpisany w okr�g o promieniu $\sqrt[n]{r_0}$.

\includegraphics[scale=1.5]{img/zespolone-pierwiastkowanie.png}

\bigskip
%%%%%%%%%%%%%%%%%%%%%%
\section{Wz�r Eulera}
\medskip
\begin{definicja}
\label{def:wzor_eulera} Pot�g� $e^z$ o podstawie $e$ i wyk�adniku $z = a + bi$, nale��cym do cia�a liczb zespolonych, okre�lamy w spos�b nast�puj�cy
\[
\begin{array}{rcl}
e^{ib} & = & cos \: b + i \: sin \: b \\
e^z & = & e^a \: e^{ib}
\end{array}
\]
\end{definicja}

\medskip
Definicja \ref{def:wzor_eulera} wprowadza nowy symbol $e^{i\varphi}$ na oznaczenie liczby o module r�wnym $1$ i argumencie $\varphi$, mianowicie
\[
e^{i\varphi} = cos \: \varphi + i \: sin \: \varphi
\]

\smallskip

R�wno�� ta umo�liwia zapisanie dowolnej liczby zespolonej $a + ib = r \left( cos\: \varphi + i \: sin \: \varphi \right)$ w tzw. \textbf{\emph{postaci wyk�adniczej}}.
\[
r \: e^{i\varphi}
\]

\chapter{Wielomiany. Funkcje wymierne}

\section{Pier�cie� ca�kowity wielomian�w}

Niech
\begin{itemize}
\item $x$ oznacza \emph{zmienn� zespolon�}, tzn. zmienn� nale��c� do cia�a $\mathbb{C}$ liczb zespolonych,
\item $\mathbb{K}$ za� jedno z cia� liczbowych:
\begin{itemize}
\item $\mathbb{C}$ - liczb zespolonych lub
\item $\mathbb{R}$ - liczb rzeczywistych
\end{itemize}
\end{itemize}

\medskip
\begin{definicja}
\textbf{\emph{Wielomianem}}\label{def:wielomian} nad cia�em liczbowym $\mathbb{K}$ nazywamy funkcj� zmiennej zespolonej $x$, okre�lon� wzorem
\[
f(x) \; = \; a_n x^n + a_{n-1} x^{n-1} + \ldots + a_2 x^2 + a_1 x + a_0
\]
gdzie
\begin{itemize}
\item liczby $a_k$ nale�� do cia�a $\mathbb{K}$ i nazywane s� \textbf{\emph{wsp�czynnikami wielomianu}}\label{def:wspolczynniki_wielomianu} ($k \: = \: 1, 2, \ldots, n$)
\item $n$ jest liczb� ca�kowit� nieujemn�
\item $a_0$ nazywamy \textbf{\emph{wyrazem wolnym}}\label{def:wyraz_wolny_wielomianu}
\item je�eli $a_n \neq 0$, to liczb� $n$ nazywamy \textbf{\emph{stopniem wielomianu}}\label{def:stopien_wielomianu}.
\end{itemize}
\end{definicja}

\medskip
\begin{twierdzenie}\label{tw:rownosc_wielomianow}
Dwa wielomiany

\medskip

\begin{tabular}{rclc}
$f(x)$ & $=$ & $a_n x^n + \ldots + a_0$ & $a_n \neq 0$ \\
$g(x)$ & $=$ & $b_m x^m + \ldots + b_0$ & $b_m \neq 0$
\end{tabular}

\begin{itemize}
\item dla kt�rych $m \leq n$
\item kt�re w $n + 1$ r�nych punktach $x_0, x_1, \ldots, x_n$ przybieraj� r�wne warto�ci
\item s� tego samego stopnia
\item maj� odpowiednie wsp�czynniki r�wne
\end{itemize}

s� \textbf{identyczne}.

\end{twierdzenie}

\medskip

\begin{definicja}
\textbf{\emph{Pierwiastkiem}}\label{pierwiastek_wielomianu} niezerowego wielomianu $f(x)$ nazywamy liczb� $a$, gdy
\[
f(a) \; = \; 0
\]

Pierwiastek niezerowego wielomianu nazywamy tak�e
\begin{itemize}
\item \textbf{\emph{miejscem zerowym}}\label{def:miejsce_zerowe_wielomianu}
\item \textbf{\emph{zerem wielomianu}}\label{def:zero_wielomianu}
\end{itemize}
\end{definicja}

\medskip

\begin{twierdzenie}[B\'{e}zout]
Wielomian $f(x)$ naturalnego stopnia jest wtedy i tylko wtedy podzielny przez dwumian $x - a$, gdy liczba $a$ jest zerem tego wielomianu.
\end{twierdzenie}

\medskip

\begin{definicja}
\textbf{\emph{$k$-krotnym pierwiastkiem}}\label{def:k-krotny_pierwiastek_wielomianu} niezerowego wielomianu $f(x)$ nazywamy liczb� $a$ gdy $(x - a)^k$ jest dzielnikiem tego wielomianu, za� $(x - a)^{k+1}$ nie jest jego dzielnikiem.

Liczb� $k$ nazywamy \textbf{\emph{krotno�ci�}} tego pierwiastka.
\end{definicja}

\medskip

\cite[Rozdzia� 9.1]{trajdos}

\bigskip
%%%%%%%%%%%%%%%%%%%%%%%%%%%%%%%%%%%%%%%%%%%%%%%%%
\section{Wielomiany nad cia�em liczb zespolonych}
Niech $W_n(z)$ b�dzie wielomianem naturalnego stopnia $n$ o wsp�czynnikach rzeczywistych
\[
W_n \quad = \quad
a_n \: z^n
+ a_{n-1} \: z^{n-1}
+ \ldots
+ a_1 \: z
+ a_0
\]

gdzie
\[
a_i \in R \qquad i = 0, \: 1, \: \ldots, \: n
\]

\medskip
\begin{twierdzenie}
Je�eli $W_n(z)$ jest wielomianem o wsp�czynnikach rzeczywistych, to
\[
\overline{ W_n(z) } \quad = \quad W_n( \overline{z} )
\]
\cite[Rozdzia� 9.2]{trajdos}
\end{twierdzenie}

\medskip
\begin{twierdzenie}
Je�eli $a$ jest $k$-krotnym pierwiastkiem wielomianu $W_n(z)$, to liczba $\overline{a}$, sprz�ona z pierwiastkiem $a$, jest tak�e $k$-krotnym pierwiastkiem tego wielomianu.
\[
W_n(a) = 0 \qquad \Rightarrow \qquad W_n\left(\overline{a}\right) = 0
\]
\cite[Rozdzia� 9.2]{trajdos}
\end{twierdzenie}


\bigskip
%%%%%%%%%%%%%%%%%%%%%%%%%%%%%%%%%%
\section{Cia�o funkcji wymiernych}

\begin{definicja}
\textbf{\emph{Funkcj� wymiern�}}\label{def:funkcja_wymierna} nad cia�em liczbowym $\mathbb{K}$ nazywamy funkcj� zmiennej zespolonej $x$, okre�lon� wzorem postaci
\[
f(x) \; = \; \dfrac{P_n(x)}{Q_m(x)}, \qquad Q \neq 0
\]

gdzie $P_n(x)$ i $Q_m(x)$ s� wielomianami zmiennej zespolonej $x$ nad cia�em liczbowym $K$.

\medskip
Funkcj� wymiern� nazywamy
\begin{itemize}
\item \textbf{\emph{w�a�ciw�}}\label{def:funkcja_wymierna_wlasciwa} gdy $n < m$
\item \textbf{\emph{niew�a�ciw�}}\label{def:funkcja_wymierna_niewlasciwa}, gdy $m \geq n$
\end{itemize}

\end{definicja}

\medskip

\begin{twierdzenie}\label{tw:funkcja_wymierna_suma_wielomianu_i_funkcji_wymiernej_wlasciwej}
Ka�d� funkcj� wymiern� mo�na przedstawi� w postaci sumy wielomianu i funkcji wymiernej w�a�ciwej.
\[
\dfrac{P_p(x)}{Q_q(x)} \; = \; W_w(x) \: + \: \dfrac{R_r(x)}{Q_q(x)}
\]

przy czym
\begin{itemize}
\item $w = 0$ lub $w = p - q$ gdy $P_p(x)/Q_q(x)$ jest funkcj� wymiern� niew�a�ciw�
\item $r < q$
\end{itemize}
\end{twierdzenie}

\medskip

\cite[Rozdzia� 9.3]{trajdos}


\bigskip
%%%%%%%%%%%%%%%%%%%%%%%%%%
\subsection{U�amki proste}

\begin{definicja}
\textbf{\emph{U�amkiem prostym}}\label{def:ulamek_prosty} nad cia�em $\mathbb{K}$ nazywamy funkcj� wymiern� nad tym cia�em:
\[
\dfrac{P_n(x)}{\left[Q_m(x)\right]^n}, \qquad Q \neq 0
\]

przy czym
\begin{itemize}
\item $Q$ jest wielomianem \textbf{nierozk�adalnym} w tym ciele
\item $n < m$
\item $n$ jest liczb� naturaln� 
\end{itemize}

\end{definicja}

\bigskip

U�amki proste nad cia�em
\begin{itemize}
\item $\mathbb{C}$ liczb zespolonych maj� posta�
\[
\dfrac{A}{(x - a)^n}, \qquad \textnormal{ gdzie } A, a \textnormal{ - liczby zespolone}
\]
\item $\mathbb{R}$ liczb rzeczywistych maj� posta�
\begin{itemize}
\item u�amki proste pierwszego rodzaju\label{def:ulamek_prosty_pierwszego_rodzaju}
\[
\dfrac{A}{(x - a)^n}, \qquad \textnormal{ gdzie } A, a \textnormal{ - liczby rzeczywiste}
\]
\item u�amki proste drugiego rodzaju\label{def:ulamek_prosty_drugiego_rodzaju}
\[
\dfrac{Ax + B}{(x^2 + px + q)^n},
\begin{array}{c}
\textnormal{ gdzie } A, B, p, q \textnormal{ - liczby rzeczywiste}, \\ 
p^2 - 4q < 0
\end{array}
\]
\end{itemize}
\end{itemize}

\medskip

\begin{twierdzenie}\label{tw:o_rozkladzie_na_ulamki_proste}
\textbf{Ka�d� funkcj� wymiern� w�a�ciw�} (Def. \ref{def:funkcja_wymierna_wlasciwa}, str. \pageref{def:funkcja_wymierna_wlasciwa}) w ciele $\mathbb{C}$ liczb zespolonych ($\mathbb{R}$ liczb rzeczywistych) mo�na przedstawi� w postaci \textbf{sumy u�amk�w prostych}.

\medskip

Ka�demu czynnikowi w rozk�adzie jej \textbf{mianownika} typu
\begin{itemize}
\item $(x - a)^n$ (w ciele $\mathbb{C}$ i $\mathbb{R}$) odpowiada sk�adnik postaci
\[
\dfrac{A_n}{(x - a)^n} + \dfrac{A_{n-1}}{(x - a)^{n-1}} + \ldots + \dfrac{A_1}{x - a}
\]
\item $(x^2 + px + q)^n, \; p^2 - 4q < 0$ (w ciele $\mathbb{R}$) odpowiada sk�adnik postaci
\[
\dfrac{A_nx + B_n}{(x^2 + px + q)^n} + \dfrac{A_{n-1}x + B_{n-1}}{(x^2 + px + q)^{n-1}} + \ldots + \dfrac{A_1x + B_1}{x^2 + px + q}
\]
\end{itemize}
\end{twierdzenie}

\cite[Rozdzia� 9.3]{trajdos}
\chapter{Macierze}

\begin{definicja}
Niech $K$ b�dzie pewnym zbiorem. \textbf{\emph{Macierz�}}\label{def:macierz} nazywamy odwzorowanie postaci:
\begin{displaymath}
\{1, 2, \ldots, m\} \times \{1, 2, \ldots, n\} \ni (i, j) \rightarrow a_{ij} \in K
\end{displaymath}
\end{definicja}

Element $a_{ij}$ nazywa� b�dziemy wyrazem macierzy.
\smallskip
Macierz b�dziemy zapisywa� w postaci $A = [a_{ij}]_{m \times n}$ lub:
\begin{displaymath}
\left[
\begin{array}{ccccc}
a_{11} & a_{12} & a_{13} & \cdots & a_{1n} \\
a_{21} & a_{22} & a_{23} & \cdots & a_{2n} \\
\vdots & \vdots &  & \ddots & \\
a_{m1} & a_{m2} & a_{m3} & \cdots & a_{mn}
\end{array}
\right]
\end{displaymath}
\cite[Definicja 3.8.1]{furdzik}

\smallskip
Ci�g $a_{i1}, a_{i2}, \ldots, a_{in}$ nazywamy $i$-tym \textbf{\emph{wierszem}}, ci�g $a_{1j}, a_{2j}, \ldots, a_{jm}$ nazywamy $j$-t� \textbf{\emph{kolumn�}} macierzy.

\medskip
Gdy $m \neq n$ macierz $A$ nazywamy macierz� \textbf{\emph{prostok�tn�}}.

Gdy $m = n$ macierz $A$ nazywamy macierz� \textbf{\emph{kwadratow�}}\label{def:macierz_kwadratowa}.

\medskip
\begin{definicja}
Wiersz (kolumn�) macierzy, kt�rej wszystkie elementy s� zerami nazywamy \textbf{\emph{wierszem (kolumn�) zerowym}}\label{def:wiersz_zerowy}\label{def:kolumna_zerowa}.
\cite[Rozdzia� 2.3]{trajdos}
\end{definicja}

\bigskip
%%%%%%%%%%%%%%%%%%%%%%%%%%%%%%%%%%%%
\section{Dzia�ania na macierzach}

\smallskip
Niech $K$ b�dzie pewnym cia�em (Def. \ref{def:cialo}, str. \pageref{def:cialo}). Niech $A$ i $B$ b�d� macierzami o $m$ wierszach i $n$ kolumnach i $\alpha \in K$.
\cite[Definicja 3.8.2]{furdzik}

\medskip

%%%%%%%%%%%%%%%%%%%%%%%%%%%%%
\subsection{Suma macierzy}

\begin{definicja}
\textbf{\emph{Sum� macierzy}} $A$ i $B$ nazywamy macierz
\begin{displaymath}
A + B \quad := \quad \left[ a_{ij} + b_{ij} \right]_{m \times n}
\end{displaymath}
\end{definicja}

\bigskip
%%%%%%%%%%%%%%%%%%%%%%%%%%%%%%%%%%%%%%%%%%
\subsection{Iloczyn skalara i macierzy}

\begin{definicja}
\textbf{\emph{Iloczynem skalara $\alpha$ i macierzy $A$}} nazwiemy macierz
\begin{displaymath}
\alpha A \quad := \quad \left[\alpha a_{ij} \right]_{m \times n}
\end{displaymath}
\end{definicja}

\bigskip
%%%%%%%%%%%%%%%%%%%%%%%%%%%%%%%%
\subsection{Iloczyn macierzy}

\smallskip
Niech teraz macierz $A$ b�dzie macierz� o $m$ wierszach i \textbf{\emph{$n$ kolumnach}}, a macierz $B$ macierz� o \textbf{\emph{$n$ wierszach}} i $p$ kolumnach.

\medskip

\begin{definicja}
\textbf{\emph{Iloczynem macierzy}} $A$ i $B$ nazywamy macierz
\begin{displaymath}
AB \quad := \quad \left[ a_{i1}b_{1j} + a_{i2}b_{2j} + \ldots + a_{in}b_{nj}\right]_{m \times p}
\end{displaymath}
\end{definicja}

\bigskip
%%%%%%%%%%%%%%%%%%%%%%%%%%%%%%%%%%%%%%%%
\section{Szczeg�lne rodzaje macierzy}

\medskip
%%%%%%%%%%%%%%%%%%%%%%%%%%%%%%%%
\subsection{Macierz jednostkowa}

\begin{definicja}
Macierz kwadratow� (Def. \ref{def:macierz_kwadratowa}, str. \pageref{def:macierz_kwadratowa}) nazywamy macierz� \textbf{\emph{jednostkow�}} gdy
\begin{displaymath}
\delta_{ij} = \left\{ \begin{array}{ll} 1 & i = j \\ 0 & i \neq j \end{array} \right.
\end{displaymath}
\end{definicja}

\smallskip
Macierz jednostkow� b�dziemy oznacza� $I$.

\bigskip
%%%%%%%%%%%%%%%%%%%%%%%%%%%%%%%
\subsection{Macierz nieosobliwa}

\begin{definicja}
Macierz kwadratow� $A$ nazywamy \textbf{\emph{nieosobliw�}} (odwracaln�), je�eli istnieje macierz kwadratowa $B$ taka, �e
\begin{displaymath}
AB \quad = \quad BA \quad = \quad I
\end{displaymath}
\cite[Definicja 3.8.4]{furdzik}
\end{definicja}

\bigskip
%%%%%%%%%%%%%%%%%%%%%%%%%%%%%%%
\subsection{Macierz diagonalna}

\begin{definicja}
Macierz kwadratow� $\left[a_{ij}\right]$ (Def. \ref{def:macierz_kwadratowa}, str. \pageref{def:macierz_kwadratowa}) nazywamy macierz� \textbf{\emph{diagonaln�}} gdy
\begin{displaymath}
a_{ij} = \left\{ \begin{array}{ll} a_{ij} \neq 0 & i = j \\ 0 & i \neq j \end{array} \right.
\end{displaymath}

\end{definicja}

\bigskip
%%%%%%%%%%%%%%%%%%%%%%%%%%%%%%%%%%
\subsection{Macierz transponowana}

\begin{definicja}
\textbf{\emph{Macierz� transponowan�}}\label{def:macierz_transponowana} do macierzy $A = \left[a_{ij}\right]_{m \times n}$ nazywamy tak� macierz $B = \left[b_{ij}\right]_{n \times m}$, �e dla ka�dego $(i,j)$ zachodzi $a_{ij}=b_{ji}$. 
\cite[Definicja 3.11.3]{furdzik}
\end{definicja}

\smallskip
Macierz transponowan� do macierzy $A$ oznaczamy przez $A^T$.

\bigskip
%%%%%%%%%%%%%%%%%%%%%%%%%%%%%%%%
\subsection{Macierz symetryczna}

\begin{definicja}
Je�eli $A = A^T$, to macierz $A$ nazywamy \textbf{\emph{macierz� symetryczn�}}.
\cite[Rozdzia� 4]{przybylo}
\end{definicja}

\bigskip
%%%%%%%%%%%%%%%%%%%%%%%%%%%%%%%%%%%%
\subsection{Macierz antysymetryczna}

\begin{definicja}
Macierz kwadratow� $\left[a_{ij}\right]$ nazywamy \textbf{\emph{macierz� antysymetryczn�}} je�eli
\begin{displaymath}
a_{ij} \quad = \quad -a_{ij} \qquad i \neq j
\end{displaymath}
\end{definicja}

\bigskip
%%%%%%%%%%%%%%%%%%%%%%%%%%
\section{Rz�d macierzy}

\begin{definicja}
\textbf{\emph{Rz�dem $r(W)$ macierzy $W$}} nazywamy najwi�kszy stopie� wyj�tego z niej r�nego od zera minora (Def. \ref{def:minor}, str. \pageref{def:minor}), przy czym je�eli wszystkie elementy macierzy s� r�wne zero, to przyjmujemy, �e rz�d jej jest r�wny zero.
\cite[Rozdzia� 9.6]{krysicki1}
\end{definicja}

\smallskip
\emph{Wniosek}. Macierz $A = [a_{ij}]_{m \times n} \qquad rzA \leq min(m,n)$ %\leqslant

\medskip
\emph{Wniosek}. $rzA$ nazywamy maksymaln� ilo�� liniowo niezale�nych wierszy (kolumn).

\bigskip
%%%%%%%%%%%%%%%%%%%%%%%%%
\subsection{W�asno�ci}

\begin{enumerate}
\item Wiersz lub kolumna zerowa (Def. \ref{def:wiersz_zerowy}, str. \pageref{def:wiersz_zerowy}) nie zwi�ksza rz�du macierzy.
\item Je�li 2 wiersze lub 2 kolumny s� proporcjonalne to skre�lenie jednego/jednej z nich nie zmienia rz�du macierzy.
\item Je�eli wiersz (kolumna) jest kombinacj� liniow� innych wierszy (kolumn), to taki wiersz (kolumna) nie zwi�ksza rz�du macierzy.
\end{enumerate}

\bigskip

%%%%%%%%%%%%%%%%%
\section{Warto�ci i wektory w�asne}

Niech $V$ b�dzie przestrzeni� wektorow� (Def. \ref{def:przestrzen_wektorowa}, str. \pageref{def:przestrzen_wektorowa}) nad cia�em (Def. \ref{def:cialo}, str. \pageref{def:cialo}) $K$ i $f$ endomorfizmem (Def. \ref{def:endomorfizm}, str. \pageref{def:endomorfizm}) w $V$.

\begin{definicja}
Skalar $\lambda \in K$ nazywamy \textbf{\emph{warto�ci� w�asn�}}\label{def:wartosc_wlasna} endomorfizmu $f$, je�eli
\[
\exists \: v \in V, \; v \neq 0 \colon \qquad f(v) = \lambda v
\]
\end{definicja}

\medskip

\begin{definicja}
Je�eli $\lambda$ jest w�asno�ci� w�asn� endomorfizmu $f$, to ka�dy niezerowy wektor $v$ spe�niaj�cy 
\[
f(v) = \lambda v
\]
nazywamy \textbf{\emph{wektorem w�asnym}}\label{def:wektor_wlasny} endomorfizmu $f$ odpowiadaj�cym $\lambda$.
\end{definicja}

\medskip
\begin{definicja}
\textbf{\emph{Warto�ci� w�asn�}}\label{def:watosc_wlasna_macierzy} macierzy $A$ o elementach $K$ nazywamy taki skalar $\lambda$, �e
\[
det\left(A - \lambda I\right) = 0
\]
\cite[Definicja 3.14.7]{furdzik}
\end{definicja}

\medskip
\begin{definicja}
\textbf{\emph{Wektorem w�asnym}}\label{def:wektor_wlasny_macierzy} macierzy $A$ odpowiadaj�cym warto�ci w�asnej $\lambda$ nazywamy niezerowy wektor $v \in K^n$ spe�niaj�cy r�wnanie
\[
\left(A - \lambda I \right) v = \bar{0}
\]
\cite[Definicja 3.14.7]{furdzik}
\end{definicja}

\medskip
\begin{definicja}
\[
\Delta \left(\lambda\right) = det \left(\lambda I - A \right)
\]
nazywamy \textbf{\emph{wielomianem charakterystycznym}}\label{def:wielomian_charakterystyczny} macierzy $A$.
\cite[Definicja 3.14.7]{furdzik}
\end{definicja}

\medskip
\begin{definicja}
R�wnanie
\[
\Delta \left(\lambda\right) = 0
\]
nazywamy \textbf{\emph{r�wnaniem charakterystycznym}}\label{def:rownanie_charakterystyczne} macierzy $A$.
\cite[Definicja 3.14.7]{furdzik}
\end{definicja}

\medskip
\begin{twierdzenie}
Je�li macierz $A$ jest symetryczna, to wszystkie warto�ci w�asne s� rzeczywiste.
\end{twierdzenie}

\medskip
\begin{twierdzenie}
Je�li macierz $A$ jest symetryczna, to wektory w�asne odpowiadaj�ce r�nym warto�ciom w�asnym s� ortogonalne (Def. \ref{def:baza_ortogonalna}, str. \pageref{def:baza_ortogonalna}).
\end{twierdzenie}

\medskip
\begin{twierdzenie}
Je�li dla warto�ci w�asnej $\lambda$ istniej� dwa wektory w�asne $v, u \in K^n$ to ich kombinacja liniowa (Def. \ref{def:kombinacja_liniowa_wektorow}, str. \pageref{def:kombinacja_liniowa_wektorow}) jest te� wektorem w�asnym.
\begin{proof}
Niech $f\left(v\right) = \lambda v$ oraz $f\left(u\right) = \lambda u$. Zbadamy czy
\[
f\left(\alpha u + \beta v\right) \stackrel{?}{=} \lambda \left( \alpha u + \beta v \right)
\]
Z definicji odwzorowania liniowego (Def. \ref{def:odwzorowanie_liniowe}, str. \pageref{def:odwzorowanie_liniowe})
\[
f\left(\alpha u + \beta v\right) = f\left(\alpha u\right) + f\left(\beta v\right) = \alpha f\left(u\right) + \beta f\left(v\right)
\]
Z za�o�enia
\[
\alpha f\left(u\right) + \beta f\left(v\right) = \alpha \lambda u + \beta \lambda v = \lambda \left(\alpha u + \beta v \right)
\]
\end{proof}
\end{twierdzenie}

\medskip
\begin{twierdzenie}
Je�eli $\lambda$ jest warto�ci� w�asn� endomorfizmu $f$, to zbi�r $V_{\lambda}$ (wektor�w w�asnych odpowiadaj�cych $\lambda$) jest podprzestrzeni� wektorow� przestrzeni $V$.
\cite[Twierdzenie 3.14.1]{furdzik}
\end{twierdzenie}


\chapter{Wyznacznik macierzy kwadratowej}

\begin{definicja}
Niech

\begin{itemize}
\item $X$ b�dzie \textbf{przestrzeni� wektorow�} (Def. \ref{def:przestrzen_wektorowa}, str. \pageref{def:przestrzen_wektorowa}) nad cia�em~$K$ (Def. \ref{def:cialo}, str. \pageref{def:cialo})

$\dim X \; = \; n$ (Def. \ref{def:wymiar_przestrzeni}, str. \pageref{def:wymiar_przestrzeni})
\item $f_0 \; = \; X^n \to K$ b�dzie \textbf{form� $n$-liniow� antysymetryczn�} (Def. \ref{def:forma_wieloliniowa_antysymetryczna}, str. \pageref{def:forma_wieloliniowa_antysymetryczna}) tak�, �e $f_0 \neq 0$
\item $u$ jest \textbf{endomorfizmem} (Def. \ref{def:endomorfizm}, str. \pageref{def:endomorfizm}) na $X$
\item odwzorowanie 

\[
g\Big(x_1, \ldots, x_n\Big) \; = \; f_0\Big(u(x_1), \ldots, u(x_n) \Big)
\]

\medskip

jest \textbf{form� $n$-liniow� antysymetryczn�}
\end{itemize}

\medskip

\textbf{\emph{Wyznacznikiem}} endomorfizmu $u$ nazywamy skalar

\[
\alpha \; \in \; K
\]

\medskip

\noindent �e dla dowolnego $x_1, \ldots, x_n$ zachodzi

\[
g \; = \; \alpha \: f_0
\]

\medskip

\noindent czyli

\[
f_0\Big(u(x_1), \ldots, u(x_n) \Big)  \; = \; \alpha \: f_0\Big(x_1, \ldots, x_n \Big)
\]

\smallskip

\cite[Definicja 3.11.1]{furdzik}
\end{definicja}

\bigskip

\begin{definicja}
\textbf{\emph{Wyznacznikiem macierzy kwadratowej}}\label{def:wyznacznik}\label{def:wyznacznik_macierzy_kwadratowej}~$A$ (Def. \ref{def:macierz_kwadratowa}, str. \pageref{def:macierz_kwadratowa}) nazywamy wyznacznik endomorfizmu (Def. \ref{def:endomorfizm}, str. \pageref{def:endomorfizm}) przestrzeni $X$ odpowiadaj�cego danej macierzy przy wybranej bazie rozpatrywanej przestrzeni.

\smallskip

\cite[Rozdzia� 5]{przybylo} \cite[Definicja 3.11.2]{furdzik}
\end{definicja}

\medskip

Wyznacznik oznaczamy 

\[
\det A \quad \textsl{,} \quad \det \left[
\begin{array}{cccc}
a_{11} & a_{12} & \cdots & a_{1n} \\
a_{21} & a_{22} & \cdots & a_{2n} \\
\vdots & \vdots & \ddots & \\
a_{n1} & a_{n2} & \cdots & a_{nn}
\end{array}
\right] \quad \textsl{,} \quad \left|
\begin{array}{cccc}
a_{11} & a_{12} & \cdots & a_{1n} \\
a_{21} & a_{22} & \cdots & a_{2n} \\
\vdots & \vdots & \ddots & \\
a_{n1} & a_{n2} & \cdots & a_{nn}
\end{array}
\right|
\]

Warto�� wyznacznika macierzy jest niezale�na od wyboru przestrzeni $X$ i~wyboru bazy tej przestrzeni.

\bigskip

\begin{twierdzenie}[warto�� wyznacznika]
Je�li jest dana macierz

\[
A = [a_{ij}]_{n \times n} = \left[
\begin{array}{cccc}
a_{11} & a_{12} & \cdots & a_{1n} \\
a_{21} & a_{22} & \cdots & a_{2n} \\
\vdots & \vdots & \ddots & \\
a_{n1} & a_{n2} & \cdots & a_{nn}
\end{array}
\right]
\]

\medskip

\noindent to warto�� wyznacznika $\det A$ oblicza si� ze wzoru

\[
\det A \quad = \quad \sum \: \textsl{sgn}\left(i_1, i_2, \ldots, i_n\right) \; a_{i_11} \: a_{i_22} \: \ldots \: a_{i_nn}
\]

\medskip

\noindent gdzie sumowanie rozci�ga si� na wszystkie permutacje (Def. \ref{def:permutacja}, str. \pageref{def:permutacja}) $\left(i_1, i_2, \ldots, i_n\right)$ zbioru $\{1, 2, \ldots, n \}$.

\smallskip

\cite[Rozdzia� 5]{przybylo}
\end{twierdzenie}

\medskip
\noindent W szczeg�lnych przypadkach mamy

\medskip
dla $n = 1$
\[
det\left[a_{11}\right] \quad = \quad a_{11}
\]

\medskip
dla $n = 2$
\[
det\left[\begin{array}{cc} a_{11} & a_{12} \\ a_{21} & a_{22} \end{array} \right] \quad = \quad a_{11} \: a_{22} \; - \; a_{21} \: a_{12}
\]

\medskip
dla $n = 3$

\begin{center}
\begin{tabular}{m{4cm}m{0,5cm}m{5cm}m{0,1cm}}
$\det
\left[
\begin{array}{ccc}
a_{11} & a_{12} & a_{13} \\
a_{21} & a_{22} & a_{23} \\
a_{31} & a_{32} & a_{33}
\end{array}
\right]$
&
$=$
&
$a_{11} \: a_{22} \: a_{33} \; + \; a_{21} \: a_{32} \: a_{13}$
$\; + \; a_{31} \: a_{12} \: a_{23} \; - \; a_{21} \: a_{12} \: a_{33}$
$\; - \; a_{11} \: a_{32} \: a_{23} \; - \; a_{31} \: a_{22} \: a_{13}$
&
\end{tabular}
\end{center}

\bigskip
%%%%%%%%%%%%%%%%%%%%%%%%%%%
\section{Podwyznaczniki}

\begin{definicja}
\textbf{\emph{Minorem (podwyznacznikiem)}}\label{def:minor} elementu $a_{ij}$ macierzy~$A$ nazywamy \textbf{wyznacznik} macierzy powsta�ej z~$A$ przez \textbf{skre�lenie} $\mathbf{i}$-tego \textbf{wiersza} oraz $\mathbf{j}$-tej \textbf{kolumny}.
\end{definicja}

\medskip

Minor oznaczmy $M_{ij}$.

\bigskip
%%%%%%%%%%%%%%%%%%%%%%%%%%%%%%%%%%
\section{Twierdzenie Laplace'a}

\begin{definicja}
\textbf{\emph{Dope�nieniem algebraicznym}} elementu $a_{ij}$ nazywamy warto��
\[
A_{ij} \quad = \quad \left(-1\right)^{i+j} M_{ij}
\]
\end{definicja}

\bigskip

\begin{twierdzenie}[Laplace'a]
Dla ka�dej macierzy $A$ o wymiarach $n \times n$ wyznacznik $\det A$ spe�nia regu��

\[
\det A \quad = \quad \sum_{j = 1}^n a_{ij} A_{ij}
\]

\medskip

\noindent gdzie $\mathbf{i}$ oznacza \textbf{numer} dowolnie wybranego \textbf{wiersza} lub

\[
detA \quad = \quad \sum_{i = 1}^n a_{ij} A_{ij}
\]

\medskip

\noindent gdzie $\mathbf{j}$ oznacza \textbf{numer} dowolnie wybranej \textbf{kolumny}.
\end{twierdzenie}

\bigskip
%%%%%%%%%%%%%%%%%%%%%%%%%%%%%%%%%%
\section{W�asno�ci wyznacznika}
\smallskip
\cite[Rozdzia� 2.3]{trajdos}

\medskip

\begin{itemize}
\item Wyznacznik macierzy kwadratowej jest r�wny wyznacznikowi macierzy transponowanej. 

\[
\det A \quad = \quad \det A^T
\]
\item Przestawienie dw�ch wierszy (kolumn) w macierzy wyznacznika jest r�wnowa�ne pomno�eniu wyznacznika przez $-1$
\item Wyznacznik macierzy o dw�ch jednakowych wierszach (kolumnach) jest r�wny zero.
\item Mno��c wiersz (kolumn�) macierzy przez liczb� mno�ymy przez t� liczb� ca�y wyznacznik tej macierzy.
\item Wyznacznik o dw�ch proporcjonalnych wierszach (kolumnach) jest r�wny zeru.
\item Wyznacznik macierzy maj�cej wiersz (kolumn�) zerowy (Def. \ref{def:wiersz_zerowy}, str. \pageref{def:wiersz_zerowy}) jest r�wny zeru.
\item Je�eli w macierzy jeden z wierszy (lub jedna z kolumn) jest kombinacj� liniow� pozosta�ych wierszy (lub kolumn), to wyznacznik tej macierzy jest r�wny zeru.
\item Wyznacznik nie zmieni warto�ci, je�eli do wiersza (kolumny) jego macierzy dodamy kombinacj� liniow� pozosta�ych wierszy (lub kolumn).
\item W macierzy o wyznaczniku r�wnym zeru wiersze (kolumny) s� liniowo zale�ne.
\end{itemize}

\bigskip
%%%%%%%%%%%%%%%%%%%%%%%%%%%%%%%%%%%%
\section{Wyznacznik uk�adu wektor�w}

\begin{definicja}
Je�eli

\begin{itemize}
\item $B \; = \; (b_1, \ldots, b_n)$ jest baz� (Def. \ref{def:baza_przestrzeni}, str. \pageref{def:baza_przestrzeni}) w~przestrzeni $X$ (Def. \ref{def:przestrzen_wektorowa}, str. \pageref{def:przestrzen_wektorowa})
\item $(x_1, x_2, \ldots, x_n) \in X$
\item oraz

\[
x_j \; = \; \sum_{i=1}^n x_j^i \: b_i
\]
\end{itemize}

\bigskip

\noindent to \textbf{\emph{wyznacznikiem uk�adu wektor�w}}\label{def:wyznacznik_ukladu_wektorow} $(x_1, x_2, \ldots, x_n)$ nazywamy \textbf{wyznacznik macierzy}, kt�rej \textbf{kolumny} stanowi� \textbf{wsp�rz�dne} wektor�w wzgl�dem bazy $B$ i~zapisujemy

\[
\det\phantom{}_B(x_1, x_2, \ldots, x_n) = \left[
\begin{array}{cccc}
x^1_1 & x^1_2 & \cdots & x^1_n \\
x^2_1 & x^2_2 & \cdots & x^2_n \\
\vdots & \vdots & \ddots & \\
x^n_1 & x^n_2 & \cdots & x^n_n
\end{array}
\right]
\]

\cite[Definicja 3.11.4]{furdzik}
\end{definicja}

\chapter{Uk�ady r�wna�}

\section{R�wnania liniowe}

Niech

\begin{itemize}
\item $f$ b�dzie przekszta�ceniem liniowym (Def. \ref{def:odwzorowanie_liniowe}, str. \pageref{def:odwzorowanie_liniowe}) przestrzeni (Def. \ref{def:przestrzen_wektorowa}, str. \pageref{def:przestrzen_wektorowa}) $X$ w~$Y$
\item $b \in Y$
\end{itemize}

\bigskip

\begin{definicja}
\textbf{\emph{R�wnaniem liniowym}}\label{def:rownanie_liniowe} nazywamy r�wnanie postaci

\[
f(x) \quad = \quad b
\]

\medskip

\noindent gdzie $x$ jest \textbf{szukanym wektorem}.

\smallskip

\cite[Definicja 3.13.1]{furdzik}
\end{definicja}

\bigskip

\begin{definicja}
R�wnanie liniowe

\[
f(x) = b
\]

\medskip

\noindent gdzie 

\[
b \neq 0
\]

\medskip

\noindent nazywamy \textbf{\emph{r�wnaniem liniowym niejednorodnym}}\label{def:rownanie_niejednorodne}.
\end{definicja}

\bigskip

\begin{definicja}
R�wnanie liniowe

\[
f(x) \quad = \quad 0
\]

\medskip

\noindent nazywamy \textbf{\emph{r�wnaniem jednorodnym stowarzyszonym}}\label{def:rownanie_jednorodne} z~r�wnaniem

\[
f(x) \quad = \quad b
\]

\end{definicja}

\bigskip

\subsection{Uk�ady r�wna�}

\begin{definicja}
Je�eli

\begin{itemize}
\item $\dim \: X \; = \; m$
\item $\dim \: Y \; = \; n$
\end{itemize}

\medskip

\noindent to otrzymujemy \textbf{\emph{uk�ad r�wna�}}\label{def:uklad_rownan}

\[
\begin{array}{ccccccccl}
a_{11}x_1 & + & a_{12}x_2 & + & \cdots & + & a_{1m}x_m & = & b_1 \\
a_{21}x_1 & + & a_{22}x_2 & + & \cdots & + & a_{2m}x_m & = & b_2 \\
\vdots &  & \vdots &  &  &  & \ddots &  &  \\
a_{n1}x_1 & + & a_{n2}x_2 & + & \cdots & + & a_{nm}x_m & = & b_n \\
\end{array}
\]

\medskip

\noindent lub r�wnowa�n� posta� macierzow� (Def. \ref{def:macierz}, str. \pageref{def:macierz})

\begin{displaymath}
\left[
\begin{array}{cccc}
a_{11} & a_{12} & \cdots & a_{1m} \\
a_{21} & a_{22} & \cdots & a_{2m} \\
\vdots & \vdots & \ddots &  \\
a_{n1} & a_{n2} & \cdots & a_{nm}
\end{array}
\right]
\left[
\begin{array}{c}
x_1 \\
x_2 \\
\vdots \\
x_m
\end{array}
\right]
=
\left[
\begin{array}{c}
b_1 \\
b_2 \\
\vdots \\
b_n
\end{array}
\right]
\end{displaymath}

\end{definicja}

\bigskip

\begin{definicja}
\textbf{\emph{Kolumn� wyraz�w wolnych}} uk�adu r�wna� nazywamy ci�g, kt�rego kolejnymi wyrazami s� 

\[
b_1, b_2, \ldots, b_n
\]

\smallskip

\cite[Rozdia� 2.5]{trajdos}
\end{definicja}

\bigskip

\begin{definicja}
\textbf{\emph{Uk�adem r�wna� niejednorodnym}} nazywamy uk�ad z�o�ony z~r�wna� niejednorodnych (Def. \ref{def:rownanie_niejednorodne}, str. \pageref{def:rownanie_niejednorodne}).
\end{definicja}

\bigskip

\begin{definicja}
\textbf{\emph{Uk�adem r�wna� jednorodnym}} nazywamy uk�ad z�o�ony z~r�wna� jednorodnych (Def. \ref{def:rownanie_jednorodne}, str. \pageref{def:rownanie_jednorodne}).
\end{definicja}

\bigskip

\begin{twierdzenie}
Uk�ad r�wna� jednorodnych przy

\[
m \; = \; n
\]

\medskip

\noindent posiada \textbf{\emph{tylko jedno rozwi�zanie zerowe}} wtedy i~tylko wtedy, gdy

\[
\det\left[a_{ij}\right] \neq 0
\]

\smallskip

\cite[Wniosek 3.13.5]{furdzik}
\end{twierdzenie}

\bigskip

\begin{twierdzenie}
Uk�ad r�wna� jednorodnych przy

\[
m \; = \; n
\]

\medskip

\noindent ma \textbf{\emph{rozwi�zania niezerowe}} wtedy i tylko wtedy, gdy

\[
\det\left[a_{ij}\right] = 0
\]

\smallskip

\cite[Wniosek 3.13.5]{furdzik}
\end{twierdzenie}

\bigskip

\section{Uk�ad Cramera}

\begin{definicja}
Je�eli

\begin{itemize}
\item $f \: \colon \: X \rightarrow X$ jest endomorfizmem (Def. \ref{def:endomorfizm}, str. \pageref{def:endomorfizm})
\item macierz $\left[a_{ij}\right]_{n \times n}$ jest macierz� $f$
\item
\[
\det
\left[ \begin{array}{ccc}
a_{11} & \cdots & a_{1n} \\
\vdots & \ddots & \\
a_{n1} & \cdots & a_{nn}
\end{array}
\right]
\; \neq \; 0
\]
\end{itemize}

\medskip

\noindent to uk�ad nazywamy \textbf{\emph{uk�adem Cramera}}\label{def:uklad_cramera}.

\smallskip

\cite[Twierdzenie 3.13.6]{furdzik} \cite[Rozdzia� 2.5]{trajdos}
\end{definicja}

\bigskip

\subsection{Rozwi�zywanie uk�ad�w r�wna� metod� Cramera}

Niech
\begin{itemize}
\item $A \; = \; \left[a_{ij}\right]_{n \times n}$
\item $D_k$ b�dzie macierz� powsta�� przez \textbf{zast�pienie} $k$-tej \textbf{kolumny} macierzy $A$ \textbf{kolumn� wyraz�w wolnych} uk�adu.
\[
D_k = 
\left[
\begin{array}{ccccccc}
a_{11} & \cdots & a_{1k-1} & b_1 & a_{1k+1} & \cdots & a_{1n} \\
\vdots & \ddots & & \vdots &\vdots & \ddots & \\
a_{n1} & \cdots & a_{nk-1} & b_n & a_{nk+1} & \cdots & a_{nn} \\
\end{array}
\right]_{n \times n}
\]
\end{itemize}

\bigskip

\begin{twierdzenie}[Cramera]
Uk�ad Cramera ma dok�adnie jedno rozwi�zanie, dane wzorami Cramera

\[
x_1 = \dfrac{\det \: D_1}{\det \: A}, \quad
x_2 = \dfrac{\det \: D_2}{\det \: A}, \quad
\ldots, \quad
x_n = \dfrac{\det \: D_n}{\det \: A}
\]

\smallskip

\cite[Rozdzia� 2.5]{trajdos}
\end{twierdzenie}

\bigskip

\section{Twierdzenie Kroneckera-Capellego}

\begin{twierdzenie}[Kroneckera-Capellego]
Uk�ad r�wna� \textbf{ma rozwi�zanie} wtedy i~tylko wtedy, gdy \textbf{rz�d macierzy wsp�czynnik�w}

\[
A \; = \;
\left[
\begin{array}{cccc}
a_{11} & a_{12} & \cdots & a_{1m} \\
a_{21} & a_{22} & \cdots & a_{2m} \\
\vdots & \vdots & \ddots &  \\
a_{n1} & a_{n2} & \cdots & a_{nm}
\end{array}
\right]
\]

\medskip

\noindent \textbf{r�wna si� rz�dowi} tzw. \emph{macierzy uzupe�nionej}

\[
U \; = \;
\left[
\begin{array}{ccccc}
a_{11} & a_{12} & \cdots & a_{1m} & b_1 \\
a_{21} & a_{22} & \cdots & a_{2m} & b_2 \\
\vdots & \vdots & \ddots &  & \vdots \\
a_{n1} & a_{n2} & \cdots & a_{nm} & b_n
\end{array}
\right]
\]

\medskip

\noindent Co oznaczamy

\[
rz \: A \quad = \quad rz \: U
\]

\smallskip

\cite[Twierdzenie 3.13.3]{furdzik}
\end{twierdzenie}

\bigskip
\begin{twierdzenie}[o liczbie rozwi�za�]
Je�li

\begin{itemize}
\item $rz \: A = rz \: U \quad = \quad m$ (ilo�ci kolumn macierzy $A$), to uk�ad ma \textbf{\emph{dok�adnie jedno}} rozwi�zanie
\item $rz \: A = rz \: U \quad = \quad r < m$, to uk�ad ma \textbf{\emph{niesko�czenie wiele}} rozwi�za� zale�nych od $n - r$ parametr�w
\end{itemize}

\end{twierdzenie}


\chapter{Geometria analityczna}

\begin{definicja}
Niech

\begin{itemize}
\item $\chi$ b�dzie przestrzeni� afiniczn� (Def. \ref{def:przestrzen_afiniczna}, str. \pageref{def:przestrzen_afiniczna})
\item przestrze� euklidesowa $\vec{E_n}$ (Def. \ref{def:przestrzen_euklidesowa}, str. \pageref{def:przestrzen_euklidesowa}) b�dzie przestrzeni� wektor�w swobodnych przestrzeni afinicznej $\chi$ (Def. \ref{def:przestrzen_wektorow_swobodnych}, str. \pageref{def:przestrzen_wektorow_swobodnych})
\end{itemize}

Przestrze� $\chi$ nazywamy, w�wczas, \textbf{\emph{afiniczn� przestrzeni� euklidesow�}}\label{def:afiniczna_przestrzen_euklidesowa} i~oznaczamy 

\[
E_n
\]

\smallskip

\cite[Definicja 5.1.1]{furdzik}
\end{definicja}

\bigskip
%%%%%%%%%%%%%%%%%%%%%%%%%%%%%%
\section{Uk�ady wsp�rz�dnych}

\subsection{Kartezja�ski uk�ad wsp�rz�dnych}

\begin{definicja}
Niech

\begin{itemize}
\item $E_3$ b�dzie tr�jwymiarow� (Def. \ref{def:wymiar_przestrzeni}, str. \pageref{def:wymiar_przestrzeni}) afiniczn� przestrzeni� euklidesow�
\item $(0; \; e_1, e_2, e_3)$ b�dzie uk�adem wsp�rz�dnych (Def. \ref{def:uklad_wspolrzednych}, str. \pageref{def:uklad_wspolrzednych}) w~przestrzeni $E_3$
\item wektory $e_1, e_2, e_3$ tworz� baz� \textbf{ortonormaln�} (Def. \ref{def:baza_ortonormalna}, str. \pageref{def:baza_ortonormalna}) i~wyznaczaj� osie liczbowe $X,Y,Z$ przez punkt~$0$
\end{itemize}

W�wczas prostok�tny uk�ad wsp�rz�dnych o~pocz�tku w~punkcie~$0$ i~osiach $X,Y,Z$

\[
(0; \; e_1, e_2, e_3)
\]

\noindent nazywamy \textbf{\emph{uk�adem kartezja�skim}}\label{def:uklad_kartezjanski}.

\smallskip

\cite[Definicja 5.1.1]{furdzik}
\end{definicja}

Wektory stanowi�ce baz� uk�adu kartezja�skiego

\[
e_1, \; e_2, \; e_3
\]

\noindent oznacza si� przez

\[
\vec{i}, \; \vec{j}, \; \vec{k}
\]

\bigskip

Wektor

\[
v \; = \; x \: \vec{i} + y \: \vec{j} + z \: \vec{k} \quad \in \quad \vec{E_3}
\]

\noindent dany w~$E_3$ przy zadanym kartezja�skim uk�adzie wsp�rz�dnych b�dziemy zapisywa�

\[
v \; = \; [x,y,z]
\]

\cite[Definicja 5.1.1]{furdzik}


\bigskip
%%%%%%%%%%%%%%%%%%%%%%%%%%%%%%%%%%%%%%%%%%
\subsection{Biegunowy uk�ad wsp�rz�dnych}

TODO

\smallskip

\cite[Rozdzia� 5.C.1]{trajdos}

\bigskip
%%%%%%%%%%%%%%%%%%%%%%%%%%%%%%%%%%%%%%%%
\subsection{Walcowy uk�ad wsp�rz�dnych}

TODO

\smallskip

\cite[Rozdzia� 5.C.1]{trajdos}

\bigskip
%%%%%%%%%%%%%%%%%%%%%%%%%%%%%%%%%%%%%%%%%%
\subsection{Sferyczny uk�ad wsp�rz�dnych}

TODO

\smallskip

\cite[Rozdzia� 5.C.1]{trajdos}


\bigskip
%%%%%%%%%%%%%%%%%%%%%%%%%%%%%%%%%%%%%%%%%%%%%%%%%%%%%%%%
\section{Orientacja przestrzeni wektorowej rzeczywistej}

\begin{definicja}
Niech

\begin{itemize}
\item $V$ b�dzie \textbf{przestrzeni� wektorow�} (Def. \ref{def:przestrzen_wektorowa}, str. \pageref{def:przestrzen_wektorowa}) rzeczywist� $n$-wymiarow� (Def. \ref{def:wymiar_przestrzeni}, str. \pageref{def:wymiar_przestrzeni})
\item $B$ i~$B'$ b�d� dwiema \textbf{bazami} (Def. \ref{def:baza_przestrzeni}, str. \pageref{def:baza_przestrzeni}) w~przestrzeni $V$
\end{itemize}

M�wimy, �e bazy $B$ i $B'$ s� \textbf{\emph{zgodnie zorientowane}}\label{def:bazy_zgodnie_zorientowane}, je�eli

\[
\det\phantom{}_B (B') \; > \; 0
\]

\medskip

M�wimy, �e bazy $B$ i $B'$ s� \textbf{\emph{przeciwnie zorientowane}}\label{def:bazy_przeciwnie_zorientowane}, je�eli

\[
\det\phantom{}_B (B') \; < \; 0
\]

(Def. \ref{def:wyznacznik_ukladu_wektorow}, str. \pageref{def:wyznacznik_ukladu_wektorow}) 

\smallskip

\cite[Definicja 5.1.3]{furdzik}
\end{definicja}

\medskip

\begin{definicja}
TODO Przestrzen wektorowa zorientowana\label{def:przestrzen_wektorowa_zorientowana}

\smallskip

\cite[Definicja 5.1.4]{furdzik}
\end{definicja}

\medskip

\begin{definicja}
TODO Przestrzen afiniczna zorientowana

\smallskip

\cite[Definicja 5.1.4]{furdzik}
\end{definicja}


\bigskip
%%%%%%%%%%%%%%%%%%%%%%%%%%
\section{Iloczyn skalarny}

\begin{twierdzenie}
Niech

\begin{itemize}
\item $E_3$ b�dzie tr�jwymiarow� (Def. \ref{def:wymiar_przestrzeni}, str. \pageref{def:wymiar_przestrzeni}) afiniczn� przestrzeni� euklidesow� (Def. \ref{def:afiniczna_przestrzen_euklidesowa}, str. \pageref{def:afiniczna_przestrzen_euklidesowa}) 
\item $(0; \; \vec{i}, \vec{j}, \vec{k})$ b�dzie \textbf{kartezja�skim} uk�adem wsp�rz�dnych (Def. \ref{def:uklad_kartezjanski}, str. \pageref{def:uklad_kartezjanski}) w~przestrzeni $E_3$
\item $u, v \; \in \; \vec{E}_3$
\item $u = [x_1, y_1, z_1]$
\item $v = [x_2, y_2, z_2]$
\end{itemize}

\medskip
W�wczas \textbf{iloczyn skalarny} (Def. \ref{def:iloczyn_skalarny_przestrzen_euklidesowa}, str. \pageref{def:iloczyn_skalarny_przestrzen_euklidesowa})\label{tw:iloczyn_skalarny_w_E3} wektor�w $u$ i~$v$ wyra�a si� wzorem

\[
(u|v) \; = \; x_1 \: x_2 + y_1 \: y_2 + z_1 \: z_2
\]

\medskip

\noindent lub korzystaj�c z~innego oznaczenia iloczynu skalarnego

\[
u \circ v \; = \; x_1 \: x_2 + y_1 \: y_2 + z_1 \: z_2
\]

\bigskip

\begin{proof}
Wektory $u$ i~$v$ zapisujemy przy u�yciu wektor�w $\vec{i}, \; \vec{j}, \; \vec{k}$ bazy (Def. \ref{def:baza_przestrzeni}, str. \pageref{def:baza_przestrzeni}) przestrzeni $\vec{E}_3$

\begin{eqnarray*}
u & = & x_1 \: \vec{i} + y_1 \: \vec{j} + z_1 \: \vec{k}\\
v & = & x_2 \: \vec{i} + y_2 \: \vec{j} + z_2 \: \vec{k}
\end{eqnarray*}

\medskip

Obliczamy iloczyn skalarny wektor�w $u$ i~$v$

\[
(u|v) \; = \; (x_1 \: \vec{i} + y_1 \: \vec{j} + z_1 \: \vec{k} \; | \; x_2 \: \vec{i} + y_2 \: \vec{j} + z_2 \: \vec{k})
\]

\medskip

Korzystaj�c z w�asno�ci iloczynu skalarnego (W�. \ref{wl:iloczyn_skalarny_przestrzen_euklidesowa}, str. \pageref{wl:iloczyn_skalarny_przestrzen_euklidesowa})

\begin{align*}
(u|v) \; = \; (x_1 \: \vec{i} \; | \; x_2 \: \vec{i} + y_2 \: \vec{j} + z_2 \: \vec{k})\\
+ \: (y_1 \: \vec{j} \;| \; x_2 \: \vec{i} + y_2 \: \vec{j} + z_2 \: \vec{k})\\
+ \: (z_1 \: \vec{k} \; | \; x_2 \: \vec{i} + y_2 \: \vec{j} + z_2 \: \vec{k})
\end{align*}

\begin{align*}
(u|v) \; = \; (x_1 \: \vec{i} \; | \; x_2 \: \vec{i}) + (x_1 \: \vec{i} \; | \; y_2 \: \vec{j}) + (x_1 \: \vec{i} \; | \; z_2 \: \vec{k})\\
+ \: (y_1 \: \vec{j} \;| \; x_2 \: \vec{i}) + (y_1 \: \vec{j} \;| \; y_2 \: \vec{j}) + (y_1 \: \vec{j} \;| \; z_2 \: \vec{k})\\
+ \: (z_1 \: \vec{k} \; | \; x_2 \: \vec{i}) + (z_1 \: \vec{k} \; | \; y_2 \: \vec{j}) + (z_1 \: \vec{k} \; | \; z_2 \: \vec{k})
\end{align*}

\begin{align*}
(u|v) \; = \; x_1 \: x_2 \: (\vec{i} \; | \; \vec{i}) + x_1 \: y_2 \: (\vec{i} \; | \; \vec{j}) + x_1 \: z_2 \: (\vec{i} \; | \; \vec{k})\\
+ \: y_1 \: x_2 \: (\vec{j} \; | \; \vec{i}) + y_1 \: y_2 \: (\vec{j} \; | \; \vec{j}) + y_1 \: z_2 \: (\vec{j} \; | \; \vec{k})\\
+ \: z_1 \: x_2 \: (\vec{k} \; | \; \vec{i}) + z_1 \: y_2 \: (\vec{k} \; | \; \vec{j}) + z_1 \: z_2 \: (\vec{k} \; | \; \vec{k})
\end{align*}

\medskip

Poniewa� wektory $\vec{i}, \; \vec{j}, \; \vec{k}$ tworz� baz� ortonormaln� (Def. \ref{def:baza_ortonormalna}, str. \pageref{def:baza_ortonormalna}), to

\begin{eqnarray*}
(\vec{i} \; | \; \vec{i}) & = & 1\\
(\vec{j} \; | \; \vec{j}) & = & 1\\
(\vec{k} \; | \; \vec{k}) & = & 1\\
(\vec{i} \; | \; \vec{j}) & = & 0\\
(\vec{i} \; | \; \vec{k}) & = & 0\\
(\vec{j} \; | \; \vec{i}) & = & 0\\
(\vec{j} \; | \; \vec{k}) & = & 0\\
(\vec{k} \; | \; \vec{i}) & = & 0\\
(\vec{k} \; | \; \vec{j}) & = & 0
\end{eqnarray*}

\medskip

Ostatecznie otrzymujemy

\[
(u|v) \; = \; x_1 \: x_2 + y_1 \: y_2 + z_1 \: z_2
\]

\end{proof}

\smallskip

\cite[Twierdzenie 5.1.1]{furdzik}
\end{twierdzenie}

\bigskip
%%%%%%%%%%%%%%%%%%%%%%%%%
\subsection{D�ugo�� wektora}

\begin{twierdzenie}
Niech

\begin{itemize}
\item $E_3$ b�dzie tr�jwymiarow� (Def. \ref{def:wymiar_przestrzeni}, str. \pageref{def:wymiar_przestrzeni}) afiniczn� przestrzeni� euklidesow� (Def. \ref{def:afiniczna_przestrzen_euklidesowa}, str. \pageref{def:afiniczna_przestrzen_euklidesowa}) 
\item $(0; \; \vec{i}, \vec{j}, \vec{k})$ b�dzie \textbf{kartezja�skim} uk�adem wsp�rz�dnych (Def. \ref{def:uklad_kartezjanski}, str. \pageref{def:uklad_kartezjanski}) w~przestrzeni $E_3$
\item $u \; \in \; \vec{E}_3$
\item $u = [x, y, z]$
\end{itemize}

\medskip

\textbf{D�ugo��} (Def. \ref{def:dlugosc_wektora_euklidesowa}, str. \pageref{def:dlugosc_wektora_euklidesowa})\label{tw:dlugosc_wektora_w_E3} tego wektora oznaczamy 

\[
|u|
\]

\medskip

\noindent a wyra�a si� wzorem

\[
|u| \; = \; \sqrt{x^2 + y^2 + z^2}
\]

\bigskip

\begin{proof}
Zgodnie z definicj� d�ugo�ci wektora (Def. \ref{def:dlugosc_wektora_euklidesowa}, str. \pageref{def:dlugosc_wektora_euklidesowa}) mamy

\[
\|v\| \; = \; \sqrt{(v|v)}
\]

\medskip

Zgodnie z twierdzeniem (Tw. \ref{tw:iloczyn_skalarny_w_E3}, str. \pageref{tw:iloczyn_skalarny_w_E3}) mamy

\[
(v|v) \; = \; x \: x + y \: y + z \: z
\]

\medskip

\noindent czyli

\[
(v|v) \; = \; x^2 + y^2 + z^2
\]

\medskip

Wstawiaj�c do d�ugo�ci wektora mamy

\[
\|v\| \;= \; \sqrt{x^2 + y^2 + z^2}
\]

\medskip

\noindent i korzystaj�c z nowego oznaczenia

\[
|v| \; = \; \sqrt{x^2 + y^2 + z^2}
\]

\end{proof}

\smallskip

\cite[Wniosek 5.1.1]{furdzik}
\end{twierdzenie}

\bigskip

\begin{twierdzenie}
Niech

\begin{itemize}
\item $E_3$ b�dzie tr�jwymiarow� (Def. \ref{def:wymiar_przestrzeni}, str. \pageref{def:wymiar_przestrzeni}) afiniczn� przestrzeni� euklidesow� (Def. \ref{def:afiniczna_przestrzen_euklidesowa}, str. \pageref{def:afiniczna_przestrzen_euklidesowa}) 
\item $(0; \; \vec{i}, \vec{j}, \vec{k})$ b�dzie \textbf{kartezja�skim} uk�adem wsp�rz�dnych (Def. \ref{def:uklad_kartezjanski}, str. \pageref{def:uklad_kartezjanski}) w~przestrzeni $E_3$
\item $A(x_1, y_1, z_1)$ b�dzie dowolnym punktem przestrzeni $E_3$
\item $B(x_2, y_2, z_2)$ b�dzie dowolnym punktem przestrzeni $E_3$
\end{itemize}

\medskip

\textbf{Odleg�o�� punkt�w} $A$ i~$B$ wyra�a si� wzorem

\[
|AB| \; = \; \sqrt{(x_2 - x_1)^2 + (y_2 - y_1)^2 + (z_2 - z_1)^2}
\]

\bigskip

\begin{proof}
Korzystaj�c z~r�nicy punkt�w (Def. \ref{def:roznica_punktow}, str. \pageref{def:roznica_punktow}) otrzymujemy \textbf{wektor ��cz�cy punkty} $A$ i~$B$

\[
\overrightarrow{AB}
\]

\medskip

\label{:roznica_punktow_E3}Przy zadanym kartezja�skim uk�adzie wsp�rz�dnych mo�emy zapisa� (TODO poszuka� definicji dla wyprowadzenia ponizej)

\[
\overrightarrow{AB} \; = \; [x_2 - x_1, \; y_2 - y_1, \; z_2 - z_1]
\]

\medskip

D�ugo�� tego wektora wynosi

\[
|\overrightarrow{AB}| \; = \; \sqrt{(x_2 - x_1)^2 + (y_2 - y_1)^2 + (z_2 - z_1)^2}
\]

\end{proof}

\smallskip

\cite[Definicja 5.1.2]{furdzik}
\end{twierdzenie}

\bigskip
%%%%%%%%%%%%%%%%%%%%%%%%%%%%%%%%%%%%%%%%%%%%%%%%%%%%%%%%%%%%%%%%%%%%%%%%%%%%%%%%%%%%%%%%%%%
\subsection{Zwi�zek iloczynu skalarnego, d�ugo�ci wektor�w oraz k�ta zawartego mi�dzy nimi}

\begin{twierdzenie}
Niech

\begin{itemize}
\item $E_3$ b�dzie tr�jwymiarow� (Def. \ref{def:wymiar_przestrzeni}, str. \pageref{def:wymiar_przestrzeni}) afiniczn� przestrzeni� euklidesow� (Def. \ref{def:afiniczna_przestrzen_euklidesowa}, str. \pageref{def:afiniczna_przestrzen_euklidesowa}) 
\item $(0; \; \vec{i}, \vec{j}, \vec{k})$ b�dzie kartezja�skim uk�adem wsp�rz�dnych (Def. \ref{def:uklad_kartezjanski}, str. \pageref{def:uklad_kartezjanski}) w~przestrzeni $E_3$
\item $x, y, z \; \in \; \vec{E}_3$
\item $z = y - x$
\item $\varphi \; = \; \angle(x,y)$
\end{itemize}

\vspace{5cm}
TODO rysunek 

\medskip

\textbf{Zwi�zek} mi�dzy \textbf{iloczynem skalarnym} wektor�w $x$, $y$, ich \textbf{d�ugo�ciami} $|x|$, $|y|$, oraz \textbf{k�tem} zawartym mi�dzy tymi wektorami $\angle(x,y)$ wyra�a si� wzorem

\[
(x \: | \: y) \; = \; |x| \: |y| \: \cos \varphi
\]

\bigskip

\begin{proof}
\textbf{Wz�r kosinus�w} z~geometrii elementarnej zastosowany do utworzonego z~wektor�w $x$, $y$, $z$ tr�jk�ta, gdzie d�ugo�ci bok�w, to d�ugo�ci wektor�w (Tw. \ref{tw:dlugosc_wektora_w_E3}, str. \pageref{tw:dlugosc_wektora_w_E3}), przyjmuje posta�

\[
|z|^2 \; = \; |x|^2 + |y|^2 - 2 |x||y| \cos \varphi
\]

\medskip

Z~drugiej strony, korzystaj�c z \textbf{definicji d�ugo�ci wektora} (Def. \ref{def:dlugosc_wektora_euklidesowa}, str. \pageref{def:dlugosc_wektora_euklidesowa}) mamy

\[
|z|^2 \; = \; |y-x|^2 \; = \; \left(\sqrt{(y-x \; | \; y-x)}\right)^2
\]

\medskip

Poniewa� \textbf{iloczyn skalarny} jest \textbf{okre�lony dodatnio} (Def. \ref{def:iloczyn_skalarny_przestrzen_euklidesowa}, str. \pageref{def:iloczyn_skalarny_przestrzen_euklidesowa}), to

\[
|z|^2 \; = \; (y-x \; | \; y-x)
\]

\medskip

Wykorzystuj�c \textbf{dwuliniowo��} (Def. \ref{def:forma_dwuliniowa}, str. \pageref{def:forma_dwuliniowa}) i~\textbf{symetryczno��} (Def. \ref{def:forma_dwuliniowa_symetryczna}, str. \pageref{def:forma_dwuliniowa_symetryczna}) iloczynu skalarnego, otrzymujemy

\[
|z|^2 \; = \; (y \: | \: y) + (x \: | \: x) - 2 \: (x \: | \: y)
\]

\medskip

Z definicji d�ugo�ci wektora (Def. \ref{def:dlugosc_wektora_euklidesowa}, str. \pageref{def:dlugosc_wektora_euklidesowa}) wstawiamy do r�wnania i~otrzymujemy

\[
|z|^2 \; = \; |y|^2 + |x|^2 - 2 \: (x \: | \: y)
\]

\medskip

Por�wnuj�c wyra�enia na $|z|^2$, otrzymujemy ostatecznie

\[
(x \: | \: y) \; = \; |x| \: |y| \: \cos \varphi
\]
\end{proof}

\smallskip

\cite[Paragraf 3.1.1]{kostrykin2}
\end{twierdzenie}


\bigskip
%%%%%%%%%%%%%%%%%%%%%%%%%%%
\section{Iloczyn wektorowy}

Niech $\vec{E}_3$ b�dzie \textbf{zorientowan�} (Def. \ref{def:przestrzen_wektorowa_zorientowana}, str. \pageref{def:przestrzen_wektorowa_zorientowana}) przestrzeni� euklidesow� (Def. \ref{def:przestrzen_euklidesowa}, str. \pageref{def:przestrzen_euklidesowa}).

\bigskip

\begin{definicja}
\textbf{\emph{Iloczynem wektorowym}}\label{def:iloczyn_wektorowy} w~$\vec{E}_3$ nazywamy odwzorowanie (Def. \ref{def:odwzorowanie}, str. \pageref{def:odwzorowanie})

\[
\times \; \colon \; \vec{E}_3 \times \vec{E}_3 \rightarrow \vec{E}_3
\]

\medskip

\noindent okre�lone

\begin{enumerate}
\item je�eli $v_1, v_2 \in \vec{E}_3$ s� \textbf{liniowo zale�ne} (Def. \ref{def:liniowa_zaleznosc_wektorow}, str. \pageref{def:liniowa_zaleznosc_wektorow}), to 

\[
v_1 \times v_2 \; = \; 0
\]

\item je�eli $v_1, v_2 \in \vec{E}_3$ s� \textbf{liniowo niezale�ne} (Def. \ref{def:liniowa_niezaleznosc_wektorow}, str. \pageref{def:liniowa_niezaleznosc_wektorow}), to

\[
v_1 \times v_2 \; = \; v
\]

\medskip

gdzie $v$ spe�nia warunki:

\begin{itemize}
\item $v \: \bot \: v_1$
\item $v \: \bot \: v_2$
\item $|v| \; = \; |v_1|\:|v_2| \: \sin \angle (v_1, v_2), \qquad \angle (v_1, v_2) \in (0, \pi)$
\item tr�jka $(v_1, v_2, v)$ jest \textbf{zgodnie zorientowana} (Def. \ref{def:bazy_zgodnie_zorientowane}, str. \pageref{def:bazy_zgodnie_zorientowane}) z~przyj�t� w~$\vec{E}_3$ baz� (Def. \ref{def:baza_przestrzeni}, str. \pageref{def:baza_przestrzeni})
\end{itemize}
\end{enumerate}

\smallskip

\cite[Definicja 5.1.8]{furdzik}
\end{definicja}

\bigskip
%%%%%%%%%%%%%%%%%%%%%%%%%%%%%%%%%%%%%%%%%%%
\subsection{W�asno�ci iloczynu wektorowego}

Je�eli\label{wl:iloczyn_wektorowy}

\begin{itemize}
\item $v_1, v_2, v_3 \in \vec{E}_3$
\item $\alpha \in \mathbb{R}$
\end{itemize}

\medskip

\noindent to

\begin{enumerate}
\item $v_1 \: \times \: v_2 \; = \; - \: v_2 \: \times \: v_1$
\item $v_1 \: \times \: (v_2 \: + \: v_3) \; = \; (v_1 \: \times \: v_2) \: + \: (v_1 \times v_3)$
\item $\alpha \: (v_1 \: \times \: v_2) \; = \; (\alpha \: v_1) \: \times \: v_2$
\end{enumerate}

\smallskip

\cite[Twierdzenie 5.1.3]{furdzik}

\bigskip
%%%%%%%%%%%%%%%%%%%%%%%%%%%%%%%%%%%%%%%
\subsection{Interpretacja geometryczna}

\vspace{5cm}

\medskip

TODO rysunek

\medskip

\begin{twierdzenie}
Je�eli 

\begin{itemize}
\item $v_1$, $v_2$ s� niezerowymi wektorami
\end{itemize}

\medskip

\noindent to \textbf{d�ugo��} wektora 

\[
v_1 \times v_2
\]

\medskip

\noindent liczbowo \textbf{r�wna} si� \textbf{polu r�wnoleg�oboku} zbudowanego na tych wektorach.

\[
P \; = \; \left|v_1 \times v_2\right|
\]

\cite[Wniosek 5.1.2]{furdzik}
\end{twierdzenie}

\bigskip
%%%%%%%%%%%%%%%%%%%%%%%%%%%%%%%%%%%%%%%%%%%%%%%%%%%%%%
\subsection{Iloczyn wektorowy w uk�adzie kartezja�skim}

\begin{twierdzenie}
Je�eli

\begin{itemize}
\item w~$E_3$ dany jest kartezja�ski uk�ad wsp�rz�dnych $(0; \; \vec{i}, \vec{j}, \vec{k})$ (Def. \ref{def:uklad_kartezjanski}, str. \pageref{def:uklad_kartezjanski})
\item $v_1 \; = \; [x_1, y_1, z_1]$
\item $v_2 \; = \; [x_2, y_2, z_2]$
\end{itemize}

\medskip

\noindent to

\[
v_1 \: \times \: v_2 \; = \; [	y_1 \: z_2 \; - \; z_1 \: y_2, \;
								z_1 \: x_2 \; - \; x_1 \: z_2, \;
								x_1 \: y_2 \; - \; y_1 \: x_2	]
\]

\medskip

\noindent lub

\[
v_1 \: \times \: v_2 \; = \;
\left|\begin{array}{cc}
y_1 & z_1\\
y_2 & z_2
\end{array}\right| \: \vec{i}
\; - \; \left|\begin{array}{cc}
x_1 & z_1\\
x_2 & z_2
\end{array}\right| \: \vec{j}
\; + \; \left|\begin{array}{cc}
x_1 & y_1\\
x_2 & y_2
\end{array}\right| \: \vec{k}
\]

\medskip

\noindent lub

\[
v_1 \: \times \: v_2 \; = \;
\left|\begin{array}{ccc}
\vec{i} & \vec{j} & \vec{k}\\
x_1 & y_1 & z_1\\
x_2 & y_2 & z_2
\end{array}\right|
\]

\bigskip

\begin{proof}
Obliczamy iloczyn wektorowy wektor�w $v_1$ i~$v_2$

\[
v_1 \: \times \: v_2 \; = \; (x_1 \: \vec{i} + y_1 \: \vec{j} + z_1 \: \vec{k}) \; \times \; (x_2 \: \vec{i} + y_2 \: \vec{j} + z_2 \: \vec{k})
\]

\medskip

Korzystaj�c z~w�asno�ci iloczynu wektorowego (W�. \ref{wl:iloczyn_wektorowy}, str. \pageref{wl:iloczyn_wektorowy})

\begin{align*}
v_1 \: \times \: v_2 \; = \; (x_1 \: \vec{i}) \; \times \; (x_2 \: \vec{i} + y_2 \: \vec{j} + z_2 \: \vec{k})\\
+ \: (y_1 \: \vec{j}) \; \times \; (x_2 \: \vec{i} + y_2 \: \vec{j} + z_2 \: \vec{k})\\
+ \: (z_1 \: \vec{k}) \; \times \; (x_2 \: \vec{i} + y_2 \: \vec{j} + z_2 \: \vec{k})
\end{align*}

\begin{align*}
v_1 \: \times \: v_2 \; = \; (x_1 \: \vec{i}) \; \times \; (x_2 \: \vec{i}) + (x_1 \: \vec{i}) \; \times \; (y_2 \: \vec{j}) + (x_1 \: \vec{i}) \; \times \; (z_2 \: \vec{k})\\
+ \: (y_1 \: \vec{j}) \; \times \; (x_2 \: \vec{i}) + (y_1 \: \vec{j}) \; \times \; (y_2 \: \vec{j}) + (y_1 \: \vec{j}) \; \times \; (z_2 \: \vec{k})\\
+ \: (z_1 \: \vec{k}) \; \times \; (x_2 \: \vec{i}) + (z_1 \: \vec{k}) \; \times \; (y_2 \: \vec{j}) + (z_1 \: \vec{k}) \; \times \; (z_2 \: \vec{k})
\end{align*}

\begin{align*}
v_1 \: \times \: v_2 \; = \; x_1 \: x_2 \: (\vec{i} \; \times \; \vec{i}) + x_1 \: y_2 \: (\vec{i} \; \times \; \vec{j}) + x_1 \: z_2 \: (\vec{i} \; \times \; \vec{k})\\
+ \: y_1 \: x_2 \: (\vec{j} \; \times \; \vec{i}) + y_1 \: y_2 \: (\vec{j} \; \times \; \vec{j}) + y_1 \: z_2 \: (\vec{j} \; \times \; \vec{k})\\
+ \: z_1 \: x_2 \: (\vec{k} \; \times \; \vec{i}) + z_1 \: y_2 \: (\vec{k} \; \times \; \vec{j}) + z_1 \: z_2 \: (\vec{k} \; \times \; \vec{k})
\end{align*}

\medskip

Poniewa� wektory $\vec{i}, \; \vec{j}, \; \vec{k}$ tworz� baz� ortonormaln� (Def. \ref{def:baza_ortonormalna}, str. \pageref{def:baza_ortonormalna}), to

\begin{eqnarray*}
\vec{i} \; \times \; \vec{i} & = & 0\\
\vec{j} \; \times \; \vec{j} & = & 0\\
\vec{k} \; \times \; \vec{k} & = & 0\\
\vec{i} \; \times \; \vec{j} & = & \vec{k}\\
\vec{i} \; \times \; \vec{k} & = & -\vec{j}\\
\vec{j} \; \times \; \vec{i} & = & -\vec{k}\\
\vec{j} \; \times \; \vec{k} & = & \vec{i}\\
\vec{k} \; \times \; \vec{i} & = & \vec{j}\\
\vec{k} \; \times \; \vec{j} & = & -\vec{i}
\end{eqnarray*}

\medskip

Ostatecznie otrzymujemy

\begin{align*}
v_1 \: \times \: v_2 \; = \; 			(y_1 \: z_2 \; - \; z_1 \: y_2) \; \vec{i}\\
								 + \;	(z_1 \: x_2 \; - \; x_1 \: z_2) \; \vec{j}\\
								 + \;	(x_1 \: y_2 \; - \; y_1 \: x_2) \; \vec{k}
\end{align*}

\end{proof}

\smallskip

\cite[Twierdzenie 5.1.4]{furdzik}
\end{twierdzenie}

\bigskip
%%%%%%%%%%%%%%%%%%%%%%%%%%
\section{Iloczyn mieszany}

Niech $\vec{E}_3$ b�dzie \textbf{zorientowan�} (Def. \ref{def:przestrzen_wektorowa_zorientowana}, str. \pageref{def:przestrzen_wektorowa_zorientowana}) przestrzeni� euklidesow� (Def. \ref{def:przestrzen_euklidesowa}, str. \pageref{def:przestrzen_euklidesowa}).

\bigskip

\begin{definicja}
\textbf{\emph{Iloczynem mieszanym}}\label{def:iloczyn_mieszany} w~$\vec{E}_3$ nazywamy odwzorowanie (Def. \ref{def:odwzorowanie}, str. \pageref{def:odwzorowanie})

\[
\vec{E}_3 \times \vec{E}_3 \times \vec{E}_3 \rightarrow \mathbb{R}
\]

\medskip

\noindent takie, �e

\[
(v_1, v_2, v_3) \; \rightarrow \; (v_1 \: \times \: v_2) \: \circ \: v_3
\]

\smallskip

\cite[Definicja 5.1.9]{furdzik}
\end{definicja}

\bigskip
%%%%%%%%%%%%%%%%%%%%%%%%%%%%%%%%%%%%%%%%%%
\subsection{W�asno�ci iloczynu mieszanego}

Je�li $v_1, v_2, v_3 \in \vec{E}_3$, to

\begin{itemize}
\item $(v_1 \: \times \: v_2) \: \circ \: v_3 \: = \: 0 \; \Leftrightarrow \; v_1, v_2, v_3$ s� \textbf{liniowo zale�ne} (Def. \ref{def:liniowa_zaleznosc_wektorow}, str. \pageref{def:liniowa_zaleznosc_wektorow})

Dla wektor�w niezerowych $v_1, v_2, v_3$ oznacza to, �e s� r�wnoleg�e do pewnej p�aszczyzny.
\item $(v_1 \: \times \: v_2) \: \circ \: v_3 \; = \; v_1 \: \circ \: (v_2 \: \times \: v_3)$
\end{itemize}

\smallskip

\cite[Twierdzenie 5.1.5]{furdzik}

\bigskip
%%%%%%%%%%%%%%%%%%%%%%%%%%%%%%%%%%%%%%%
\subsection{Interpretacja geometryczna}

\begin{twierdzenie}
Je�eli

\begin{itemize}
\item w~$E_3$ dany jest kartezja�ski uk�ad wsp�rz�dnych $(0; \; \vec{i}, \vec{j}, \vec{k})$ (Def. \ref{def:uklad_kartezjanski}, str. \pageref{def:uklad_kartezjanski})
\item $v_1 \; = \; [x_1, y_1, z_1]$
\item $v_2 \; = \; [x_2, y_2, z_2]$
\item $v_3 \; = \; [x_3, y_3, z_3]$
\end{itemize}

\medskip

\noindent to

\[
(v_1 \: \times \: v_2) \: \circ \: v_3 \; = \; 
\left|\begin{array}{ccc}
x_1 & y_1 & z_1\\
x_2 & y_2 & z_2\\
x_3 & y_3 & z_3
\end{array}\right|
\]

\smallskip

\cite[Twierdzenie 5.1.6]{furdzik}
\end{twierdzenie}

\bigskip
%%%%%%%%%%%%%%%%%%%%%%%%%%%%%%%%%%%%%%%%%%%%%%%%%%%%%%
\subsection{Iloczyn mieszany w uk�adzie kartezja�skim}

\vspace{5cm}

\medskip

TODO rysunek

\medskip

\begin{twierdzenie}
Je�eli 

\begin{itemize}
\item $v_1$, $v_2$, $v_3$ s� niezerowymi wektorami
\end{itemize}

\medskip

\noindent to \textbf{warto�� bezwzgl�dna} iloczynu mieszanego

\[
(v_1 \: \times \: v_2) \: \circ \: v_3
\]

\medskip

\noindent \textbf{r�wna} si� \textbf{obj�to�ci r�wnoleg�o�cianu} zbudowanego na tych wektorach.

\[
V \; = \; \Big|(v_1 \: \times \: v_2) \: \circ \: v_3\Big|
\]

\cite[Wniosek 5.1.3]{furdzik}
\end{twierdzenie}



\bigskip
%%%%%%%%%%%%%%%%%%%%%%%%%%%%%%%%%%%
\section{P�aszczyzna w przestrzeni}

Niech

\begin{itemize}
\item w~$E_3$ dany b�dzie kartezja�ski uk�ad wsp�rz�dnych $(0; \; \vec{i}, \vec{j}, \vec{k})$ (Def. \ref{def:uklad_kartezjanski}, str. \pageref{def:uklad_kartezjanski})
\item $P_0 \: (x_0, y_0, z_0)$
\item $P \: (x, y, z)$
\item $v \; = \; [v_1, v_2, v_3]$
\item $u \; = \; [u_1, u_2, u_3]$
\item $v$ i~$u$ s� \textbf{liniowo niezale�ne} (Def. \ref{def:liniowa_niezaleznosc_wektorow}, str. \pageref{def:liniowa_niezaleznosc_wektorow})
\item $\vec{n} \; = \; v \: \times u$
\item dana b�dzie \textbf{p�aszczyzna} $\pi$
\item $P_0 \: \in \: \pi$
\item $P \: \in \: \pi$
\item $v \; || \; \pi$
\item $u \; || \; \pi$
\item $\vec{n} \; \bot \; \pi$
\end{itemize}

\bigskip
%%%%%%%%%%%%%%%%%%%%%%%%%%
\subsection{Posta� og�lna}

Tworzymy wektor ��cz�cy punkty $P$ i~$P_0$ (Def. \ref{def:roznica_punktow}, str. \pageref{def:roznica_punktow}) (Tw. \ref{:roznica_punktow_E3}, str. \pageref{:roznica_punktow_E3})

\[
\overrightarrow{P_0 \: P} \; = \; [x - x_0, \: y - y_0, \: z - z_0]
\]

\medskip

Przyjmujemy wsp�rz�dne wektora $\vec{n}$

\[
\vec{n} \; = \; [A, \: B, \: C]
\]

\medskip

R�wnanie p�aszczyzny $\pi$ \textbf{przechodz�cej} przez punkt $P_0$ i~\textbf{prostopad�ej} do $\vec{n}$ ($\vec{n}$ nazywamy \textbf{wektorem normalnym}\label{def:wektor_normalny_plaszczyzny} p�aszczyzny $\pi$) wyra�a si� wzorem

\[
\overrightarrow{P_0 \: P} \; \circ \; \vec{n} \; = \; 0
\]

\medskip

Z~twierdzenia o~iloczynie skalarnym w~$\vec{E}_3$ (Def. \ref{tw:iloczyn_skalarny_w_E3}, str. \pageref{tw:iloczyn_skalarny_w_E3}) z~kartezja�skim uk�adem wsp�rz�dnych (Def. \ref{def:uklad_kartezjanski}, str. \pageref{def:uklad_kartezjanski}) otrzymujemy r�wnanie p�aszczyzny w~nast�puj�cej postaci

\[
A \: (x - x_0) \; + \; B \: (y - y_0) \; + \; C \: (z - z_0) \; = \; 0
\]

\medskip

Przyjmuj�c, �e

\[
D \; = \; - A \: x_0 \; - \; B \: y_0 \; - \; C \: z_0
\]

\medskip

\noindent otrzymujemy \textbf{\emph{posta� og�ln�}} r�wnania p�aszczyzny $\pi$

\[
A \: x \; + \; B \: y \; + \; C \: z \; + \; D \; = \; 0
\]

\smallskip

\cite[R�wnanie 5.2.4]{furdzik}

\bigskip
%%%%%%%%%%%%%%%%%%%%%%%%%%%%%
\subsection{Posta� odcinkowa}

Niech

\begin{itemize}
\item $\pi$ b�dzie p�aszczyzn� o~r�wnaniu og�lnym $Ax \: + \: By \: + \: Cz \: + \: D = 0$
\item $\pi$ \textbf{nie przechodzi} przez \textbf{pocz�tek} uk�adu wsp�rz�dnych ($D \neq 0$)
\item $\pi$ \textbf{nie jest r�wnoleg�a} do �adnej \textbf{osi} uk�adu wsp�rz�dnych ($A \neq 0$, $B \neq 0$, $C \neq 0$)
\end{itemize}

\medskip

R�wnanie p�aszczyzny $\pi$ mo�emy, wtedy zapisa� w~\textbf{\emph{postaci odcinkowej}}

\[
\frac{x}{-\frac{D}{A}} + \frac{y}{-\frac{D}{B}} + \frac{z}{-\frac{D}{C}} \; = \; 1
\]

\medskip

\noindent lub, przyjmuj�c oznaczenia $-\frac{D}{A} = a$, $-\frac{D}{B} = b$, $-\frac{D}{C} = c$

\[
\frac{x}{a} + \frac{y}{b} + \frac{z}{b} \; = \; 1
\]

\medskip

Punkty

\[
(a, 0, 0)\quad
(0, b, 0)\quad
(0, 0, c)
\]

\medskip

\noindent s� punktami przeci�cia p�aszczyzny $\pi$ z~osiami uk�adu wsp�rz�dnych.


\smallskip

\cite[R�wnanie 5.2.6]{furdzik}

\bigskip
%%%%%%%%%%%%%%%%%%%%%%%%%%%%%%%%%
\subsection{Posta� parametryczna}

\[
\pi \; \colon \; P \; = \; P_0 + t \: v + s \: u, \qquad t,s \in \mathbb{R}
\]

\smallskip

\cite[R�wnanie 5.2.1]{furdzik}

\bigskip
%%%%%%%%%%%%%%%%%%%%%%%%%%%%%%
\section{Prosta w przestrzeni}

Niech

\begin{itemize}
\item w~$E_3$ dany b�dzie kartezja�ski uk�ad wsp�rz�dnych $(0; \; \vec{i}, \vec{j}, \vec{k})$ (Def. \ref{def:uklad_kartezjanski}, str. \pageref{def:uklad_kartezjanski})
\item $P_0 \: (x_0, y_0, z_0)$
\item $v \; = \; [v_1, v_2, v_3]$
\item dana b�dzie \textbf{prosta} $l$
\item $P_0 \: \in \: l$
\item $v \; || \; l$
\end{itemize}

\bigskip
%%%%%%%%%%%%%%%%%%%%%%%%%%%%%%
\subsection{Posta� kanoniczna}

\[
l \; \colon \; \frac{x - x_0}{v_1} \; = \; \frac{y - y_0}{v_2} \; = \; \frac{z - z_0}{v_3}, \qquad \qquad v_1 \: v_2 \: v_3 \; \neq \; 0
\]

\smallskip

\cite[Rozdzia� 5.2]{trajdos}

\bigskip
%%%%%%%%%%%%%%%%%%%%%%%%%%%%%%%%%
\subsection{Posta� parametryczna}

\[
l \; \colon \; P \; = \; P_0 + t \: v, \qquad t \in \mathbb{R}
\]

\smallskip

\cite[R�wnanie 5.2.1]{furdzik}

\bigskip
%%%%%%%%%%%%%%%%%%%%%%%%%%%%%%%
\subsection{Posta� kraw�dziowa}

Je�eli

\begin{itemize}
\item $\pi_1$ b�dzie p�aszczyzn� o~r�wnaniu

\[
\pi_1 \; \colon \; A_1 \: x + B_1 \: y + C_1 \: z + D_1 \; = \; 0
\]
\item $\pi_2$ b�dzie p�aszczyzn� o~r�wnaniu

\[
\pi_2 \; \colon \; A_2 \: x + B_2 \: y + C_2 \: z + D_2 \; = \; 0
\]
\item p�aszczyzny $\pi_1$ i~$\pi_2$ \textbf{nie s� r�wnoleg�e}, czyli wektory

\[
\vec{n}_1 \; = \; [A_1, \: B_1, \: C_1]
\]

\medskip

oraz 

\[
\vec{n}_2 \; = \; [A_2, \: B_2, \: C_2]
\]

\medskip

s� \textbf{liniowo niezale�ne} (Def. \ref{def:liniowa_niezaleznosc_wektorow}, str. \pageref{def:liniowa_niezaleznosc_wektorow})
\end{itemize}

\medskip

\noindent to uk�ad r�wna� (Def. \ref{def:uklad_rownan}, str. \pageref{def:uklad_rownan})

\[
\left\{\begin{array}{c}
A_1 \: x + B_1 \: y + C_1 \: z + D_1 \; = \; 0\\
A_2 \: x + B_2 \: y + C_2 \: z + D_2 \; = \; 0
\end{array}\right.
\]

\medskip

\noindent nazywamy \textbf{\emph{postaci� kraw�dziow�}}\label{def:postac_krawedziowa_prostej} prostej $l$.

\smallskip

\cite[Rozdzia� 5.2]{furdzik}

\bigskip
%%%%%%%%%%%%%%%%%%%%%%%%%%%%%%%%%%%%%%%%%%
\section{Wzajemne po�o�enie w przestrzeni}

\bigskip
%%%%%%%%%%%%%%%%%%%%%%%%%%%%%%%%%%%%%%%%%%%%%%
\subsection{Wzajemne po�o�enie dw�ch prostych}

Niech

\begin{itemize}
\item $l_1$ b�dzie prost� o~r�wnaniu

\[
l_1 \; \colon \; P \; = \; P_1 + t \: v_1
\]
\item $l_2$ b�dzie prost� o~r�wnaniu

\[
l_2 \; \colon \; P \; = \; P_2 + t \: v_2
\]
\end{itemize}

\medskip

Proste mog� by� w~stosunku do siebie

\begin{itemize}
\item \textbf{\emph{pokrywaj�ce}} si�

Proste $l_1$ i~$l_2$ pokrywaj� si�, je�eli wektory $v_1$, $v_2$ oraz $\overrightarrow{P_0 \: P}$ s� parami \textbf{liniowo zale�ne} (Def. \ref{def:liniowa_zaleznosc_wektorow}, str. \pageref{def:liniowa_zaleznosc_wektorow}), czyli s� \textbf{r�wnoleg�e}.

\item \textbf{\emph{r�wnoleg�e}} (nie maj� punkty wsp�lnego i~le�� w jednej p�aszczy�nie)

Proste $l_1$ i~$l_2$ s� r�wnoleg�e, je�eli \textbf{nie s� pokrywaj�ce} si�, a wektory $v_1$ i~$v_2$ s� \textbf{liniowo zale�ne} (Def. \ref{def:liniowa_zaleznosc_wektorow}, str. \pageref{def:liniowa_zaleznosc_wektorow}), czyli s� \textbf{r�wnoleg�e}.

\item \textbf{\emph{przecinaj�ce}} si�

Proste $l_1$ i~$l_2$ przecinaj� si�, je�eli uk�ad r�wna� (Def. \ref{def:uklad_rownan}, str. \pageref{def:uklad_rownan}) tych prostych posiada \textbf{dok�adnie jedno rozwi�zanie}.

\item \textbf{\emph{sko�ne}} (nie le�� w jednej p�aszczy�nie)

Proste $l_1$ i~$l_2$ s� sko�ne, je�eli wektory $v_1$, $v_2$ oraz $\overrightarrow{P_0 \: P}$ s� parami \textbf{liniowo niezale�ne} (Def. \ref{def:liniowa_niezaleznosc_wektorow}, str. \pageref{def:liniowa_niezaleznosc_wektorow}).
\end{itemize}

\smallskip

\cite[Rozdzia� 5.2]{furdzik}

\bigskip
%%%%%%%%%%%%%%%%%%%%%%%%%%%%%%%%%%%%%%%%%%%%%%%%
\subsection{Wzajemne po�o�enie dw�ch p�aszczyzn}

Niech

\begin{itemize}
\item $\pi_1$ b�dzie p�aszczyzn� o~r�wnaniu

\[
\pi_1 \; \colon \; \overrightarrow{P_1 \: P} \; \circ \; \vec{n}_1 \; = \; 0
\]
\item $\pi_2$ b�dzie p�aszczyzn� o~r�wnaniu

\[
\pi_2 \; \colon \; \overrightarrow{P_2 \: P} \; \circ \; \vec{n}_2 \; = \; 0
\]
\end{itemize}

\medskip

Dwie p�aszczyzny $\pi_1$ i~$\pi_2$ mog� znajdowa� si� w~nast�puj�cych po�o�enia wzgl�dem siebie

\begin{itemize}
\item P�aszczyzny $\pi_1$ i~$\pi_2$ \textbf{\emph{pokrywaj�}} si�

Warunki

\begin{itemize}
\item wektory $\vec{n}_1$ i~$\vec{n}_2$ s� \textbf{r�wnoleg�e}
\item wektory $\vec{n}_1$ i~$\overrightarrow{P_1 \: P_2}$ ($\vec{n}_2$ i~$\overrightarrow{P_1 \: P_2}$) s� \textbf{prostopad�e}
\end{itemize}

\item P�aszczyzny $\pi_1$ i~$\pi_2$ s� \textbf{\emph{r�wnoleg�e}}

Warunki

\begin{itemize}
\item p�aszczyzny $\pi_1$ i~$\pi_2$ \textbf{nie pokrywaj�} si�
\item wektory $\vec{n}_1$ i~$\vec{n}_2$ s� \textbf{r�wnoleg�e}
\end{itemize}

\item P�aszczyzny $\pi_1$ i~$\pi_2$ s� \textbf{\emph{przecinaj�}} si�

Warunek

\begin{itemize}
\item wektory $\vec{n}_1$ i~$\vec{n}_2$ s� \textbf{nie s� r�wnoleg�e}
\end{itemize}
\end{itemize}

\smallskip

\cite[Rozdzia� 5.2]{furdzik}

\bigskip
%%%%%%%%%%%%%%%%%%%%%%%%%%%%%%%%%%%%%%%%%%%%%%%%%%%%%
\subsection{Wzajemne po�o�enie prostej i p�aszczyzny}

Niech

\begin{itemize}
\item $l$ b�dzie prost� o~r�wnaniu

\[
l \; \colon \; P \; = \; P_1 + t \: v
\]
\item $\pi$ b�dzie p�aszczyzn� o~r�wnaniu

\[
\pi \; \colon \; \overrightarrow{P_0 \: P} \; \circ \; \vec{n} \; = \; 0
\]
\end{itemize}

\medskip

Prosta $l$ oraz p�aszczyzna $\pi$ mog� znajdowa� si� w~nast�puj�cych po�o�enia wzgl�dem siebie

\medskip

\begin{itemize}
\item Prosta $l$ \textbf{\emph{le�y}} na p�aszczy�nie $\pi$

Warunki

\[
v \: \circ \: \vec{n} \; = \; 0
\]

\noindent i

\[
\overrightarrow{P_0 \: P_1} \: \circ \: \vec{n} \; = \; 0
\]

\item Prosta $l$ jest \textbf{\emph{r�wnoleg�a}} do p�aszczyzny $\pi$

Warunki

\[
v \: \circ \: \vec{n} \; = \; 0
\]

\noindent i

\[
\overrightarrow{P_0 \: P_1} \: \circ \: \vec{n} \; \neq \; 0
\]

\item Prosta $l$ \textbf{\emph{przebija}} p�aszczyzn� $\pi$

Warunek

\[
v \: \circ \: \vec{n} \; \neq \; 0
\]

\end{itemize}

\smallskip

\cite[Rozdzia� 5.2]{furdzik}

\bigskip
%%%%%%%%%%%%%%%%%%%%%%%%%%%
\subsection{P�k p�aszczyzn}

Niech

\begin{itemize}
\item dana b�dzie prosta $l$ opisana r�wnaniem w~postaci kraw�dziowej (Def. \ref{def:postac_krawedziowa_prostej}, str. \pageref{def:postac_krawedziowa_prostej})

\[
\left\{\begin{array}{c}
\pi_1 \; \colon \; A_1 \: x + B_1 \: y + C_1 \: z + D_1 \; = \; 0\\
\pi_2 \; \colon \; A_2 \: x + B_2 \: y + C_2 \: z + D_2 \; = \; 0
\end{array}\right.
\]

\item p�aszczyzny $\pi_1$ i~$\pi_2$ \textbf{nie s� r�wnoleg�e}, czyli wektory

\[
\vec{n}_1 \; = \; [A_1, \: B_1, \: C_1]
\]

\medskip

oraz 

\[
\vec{n}_2 \; = \; [A_2, \: B_2, \: C_2]
\]

\medskip

s� \textbf{liniowo niezale�ne} (Def. \ref{def:liniowa_niezaleznosc_wektorow}, str. \pageref{def:liniowa_niezaleznosc_wektorow})
\end{itemize}

\medskip

\begin{definicja}
\textbf{\emph{P�kiem p�aszczyzn}} o~kraw�dzi $l$ nazywamy \textbf{zbi�r wszystkich} p�aszczyzn, na kt�rych le�y prosta $l$.

\smallskip

\cite[Rozdzia� 5.3]{trajdos}
\end{definicja}

\medskip

P�k p�aszczyzn tworzy rodzin� p�aszczyzn

\[
\lambda_1 \: \pi_1 \; + \; \lambda_2 \: \pi_2 \; = \; 0 \qquad \lambda_1^2 \: + \: \lambda_2^2 \: > \: 0
\]

\medskip

\noindent czyli

\[
\lambda_1 \: (A_1 \: x + B_1 \: y + C_1 \: z + D_1) \; + \; \lambda_2 \: (A_2 \: x + B_2 \: y + C_2 \: z + D_2) \; = \; 0
\]

\smallskip

\cite[Rozdzia� 5.3]{trajdos}

\bigskip
%%%%%%%%%%%%%%%%%%%%%%%%%%%%%%%%%
\section{Obszary na p�aszczy�nie}

\medskip

TODO

\subsection{Obszar normalny}

\subsection{Obszar regularny}

\subsection{Twierdzenie Fubiniego}

\bigskip
%%%%%%%%%%%%%%%%%%%%%%%%%%%%%%%%%%%%%%%
\section{Powierzchnie stopnia drugiego}

\smallskip

\cite[Rozdzia� 5.3]{furdzik}

\medskip

Niech

\begin{itemize}
\item $E_3$ b�dzie tr�jwymiarow� (Def. \ref{def:wymiar_przestrzeni}, str. \pageref{def:wymiar_przestrzeni}) afiniczn� przestrzeni� euklidesow� (Def. \ref{def:afiniczna_przestrzen_euklidesowa}, str. \pageref{def:afiniczna_przestrzen_euklidesowa}) 
\item $(0; \; \vec{i}, \vec{j}, \vec{k})$ b�dzie \textbf{kartezja�skim} uk�adem wsp�rz�dnych (Def. \ref{def:uklad_kartezjanski}, str. \pageref{def:uklad_kartezjanski}) w~przestrzeni $E_3$
\end{itemize}

\medskip

\begin{definicja}
\textbf{\emph{Powierzchni� stopnia drugiego}} w~$E_3$ nazywamy zbi�r wszystkich punkt�w spe�niaj�cych r�wnanie

\[
\left[
\begin{array}{ccc}
x & y & z\\
\end{array}
\right]
\left[
\begin{array}{ccc}
a_{11} & a_{12} & a_{13}\\
a_{21} & a_{22} & a_{23}\\
a_{31} & a_{32} & a_{33}
\end{array}
\right]
\left[
\begin{array}{c}
x\\
y\\
z
\end{array}
\right]
+
\left[
\begin{array}{ccc}
b_{1} & b_{2} & b_{3}\\
\end{array}
\right]
\left[
\begin{array}{c}
x\\
y\\
z
\end{array}
\right]
+
c
=
0
\]
\end{definicja}

\medskip

Powierzchnie drugiego stopnia mo�na opisa� r�wnaniami w~postaci kanonicznej

\[
\frac{x^2}{a^2} \: + \: \frac{y^2}{b^2} \: + \: \frac{z^2}{c^2} \; = \;
\left\{
\begin{array}{cl}
1  & \textnormal{elipsoida} \\
0  & \textnormal{punkt} \\
-1 & \textnormal{zbi�r pusty}
\end{array}
\right.
\]

\vspace{10cm}

TODO rysunek

\medskip

\[
\frac{x^2}{a^2} \: + \: \frac{y^2}{b^2} \: - \: \frac{z^2}{c^2} \; = \;
\left\{
\begin{array}{cl}
1  & \textnormal{hiperboloida jednopow�okowa} \\
0  & \textnormal{sto�ek eliptyczny} \\
-1 & \textnormal{hiperboloida dwupow�okowa}
\end{array}
\right.
\]

\vspace{10cm}

TODO rysunek

\medskip

\[
\frac{x^2}{a^2} \: + \: \frac{y^2}{b^2} \; = \;
\left\{
\begin{array}{cl}
2z & \textnormal{paraboloida eliptyczna} \\
1  & \textnormal{walec eliptyczny} \\
0  & \textnormal{prosta (o� } OZ \textnormal{)} \\
-1 & \textnormal{zbi�r pusty}
\end{array}
\right.
\]

\vspace{10cm}

TODO rysunek

\medskip

\[
\frac{x^2}{a^2} \: - \: \frac{y^2}{b^2} \; = \;
\left\{
\begin{array}{cl}
2z & \textnormal{paraboloida hiperboliczna} \\
1  & \textnormal{walec hiperboliczny} \\
0  & \textnormal{dwie p�aszczyzny przecinaj�ce si�}
\end{array}
\right.
\]

\vspace{10cm}

TODO rysunek

\medskip

\[
y^2 \; = \;
\left\{
\begin{array}{cl}
2px  & \textnormal{walec paraboliczny} \\
a^2  & \textnormal{dwie p�aszczyzny r�wnoleg�e} \\
0    & \textnormal{p�aszczyzna} \\
-a^2 & \textnormal{zbi�r pusty}
\end{array}
\right.
\]

\vspace{10cm}

TODO rysunek

\medskip
\chapter{Ci�g liczbowy}

\begin{definicja}
\textbf{\emph{Ci�giem}}\label{def:ciag}\index{ci�g} liczb rzeczywistych nazywamy dowoln� funkcj� (Def. \ref{def:funkcja}, str. \pageref{def:funkcja})

\[
a \colon \; \mathbb{N} \: \rightarrow \: \mathbb{R}
\]

\bigskip

\noindent B�dziemy pisa� $a_n$ zamiast $a(n)$

\medskip

\noindent oraz
\begin{itemize}
  \item $(a_n)_{n=1}^\infty \subset \mathbb{R}$
  \item $\left\{a_n\right\}_{n=1}^\infty \subset \mathbb{R}$
  \item $\left(a_n\right) \subset \mathbb{R}$
\end{itemize}
zamiast $a \colon \mathbb{N} \rightarrow \mathbb{R}$

\smallskip

\cite[Rozdzia� 4.1]{ptak}
\end{definicja}

\bigskip

\begin{definicja}
Liczby $a_1, a_2, \ldots$ nazywamy \textbf{\emph{wyrazami ci�gu}}\index{ci�g!wyrazy} $\left(a_n\right)$.

\smallskip

\cite[Paragraf 2.1]{krysicki1}
\end{definicja}

\bigskip

\begin{definicja}
Symbol $a_n$ nazywamy \textbf{\emph{wyrazem og�lnym}}\index{ci�g!wyraz og�lny} ci�gu $\left(a_n\right)$.

\smallskip

\cite[Paragraf 2.1]{krysicki1}
\end{definicja}

\bigskip
%%%%%%%%%%%%%%%%%%%%%%%%%%
\section{Ci�g ograniczony}

\begin{definicja}
Ci�g $\left(a_n\right)$ nazywamy \textbf{\emph{ograniczonym od g�ry}}\label{def:ciag_ograniczony_od_gory}\index{ci�g!ograniczony od g�ry}, je�li

\begin{center}
istnieje $M \in \mathbb{R}$ takie, �e $a_n \leq M$ dla ka�dego $n \in \mathbb{N}$
\end{center}

\noindent czyli

\[
\exists \: M \in \mathbb{R} \quad \forall \: n \in \mathbb{N} \colon \qquad a_n \leq M
\]

\smallskip

\cite[Definicja 4.1]{ptak}
\end{definicja}

\bigskip

\begin{definicja}
Ci�g $\left(a_n\right)$ nazywamy \textbf{\emph{ograniczonym od do�u}}\label{def:ciag_ograniczony_od_dolu}\index{ci�g!ograniczony od do�u}, je�li

\begin{center}
istnieje $M \in \mathbb{R}$ takie, �e $a_n \geq M$ dla ka�dego $n \in \mathbb{N}$
\end{center}

\noindent czyli

\[
\exists \: M \in \mathbb{R} \quad \forall \: n \in \mathbb{N} \colon \qquad a_n \geq M
\]

\smallskip

\cite[Definicja 4.1]{ptak}
\end{definicja}

\bigskip

\begin{definicja}
Ci�g $\left(a_n\right)$ nazywamy \textbf{\emph{ograniczonym}}\label{def:ciag_ograniczony}\index{ci�g!ograniczony}, je�li

\begin{center}
istnieje $M \in \mathbb{R}$ takie, �e $\left|a_n\right| \leq M$ dla ka�dego $n \in \mathbb{N}$
\end{center}

\noindent czyli

\[
\exists \: M \in \mathbb{R} \quad \forall \: n \in \mathbb{N} \colon \qquad \left|a_n\right| \leq M
\]

\smallskip

\cite[Definicja 4.1]{ptak}
\end{definicja}

\bigskip
%%%%%%%%%%%%%%%%%%%%%%%%%%%
\section{Ci�g monotoniczny}

\begin{definicja}
Ci�g $\left(a_n\right)$ nazywamy \textbf{\emph{rosn�cym}}\label{def:ciag_rosnacy}\index{ci�g!rosn�cy}, je�li

\[
a_n \: \leq \: a_{n+1}
\]

\noindent dla $n \in \mathbb{N}$.

\smallskip

\cite[Definicja 4.3]{ptak}
\end{definicja}

\bigskip

\begin{definicja}
Ci�g $\left(a_n\right)$ nazywamy \textbf{\emph{silnie rosn�cym}}\label{def:ciag_silnie_rosnacy}\index{ci�g!silnie rosn�cy}, je�li

\[
a_n \: < \: a_{n+1}
\]

\noindent dla $n \in \mathbb{N}$.

\smallskip

\cite[Definicja 4.3]{ptak}
\end{definicja}

\bigskip

\begin{definicja}
Ci�g $\left(a_n\right)$ nazywamy \textbf{\emph{malej�cym}}\label{def:ciag_malejacy}\index{ci�g!malej�cy}, je�li

\[
a_n \: \geq \: a_{n+1}
\]

\noindent dla $n \in \mathbb{N}$.

\smallskip

\cite[Definicja 4.3]{ptak}
\end{definicja}

\bigskip

\begin{definicja}
Ci�g $\left(a_n\right)$ nazywamy \textbf{\emph{silnie malej�cym}}\index{ci�g!silnie malej�cy}, je�li

\[
a_n \: > \: a_{n+1}
\]

\noindent dla $n \in \mathbb{N}$.

\smallskip

\cite[Definicja 4.3]{ptak}
\end{definicja}

\bigskip

\begin{definicja}
Ci�g $\left(a_n\right)$ nazywamy \textbf{\emph{monotonicznym}}\index{ci�g!monotoniczny}, je�li jest ci�giem

\begin{itemize}
\item \textbf{\emph{rosn�cym}}
\item lub \textbf{\emph{malej�cym}}.
\end{itemize}

\smallskip

\cite[Definicja 4.3]{ptak}
\end{definicja}

\bigskip

\begin{definicja}
Ci�g $\left(a_n\right)$ nazywamy \textbf{\emph{silnie monotonicznym}}\index{ci�g!silnie monotoniczny}, je�li jest ci�giem

\begin{itemize}
\item \textbf{\emph{silnie rosn�cym}}
\item lub \textbf{\emph{silnie malej�cym}}.
\end{itemize}

\smallskip

\cite[Definicja 4.3]{ptak}
\end{definicja}

\bigskip
%%%%%%%%%%%%%%%%%%%%%%
\section{Ci�g zbie�ny. Granica ci�gu}

\begin{definicja}[Cauchy'ego]
M�wimy, �e ci�g $\left(a_n\right)$ jest \textbf{\emph{zbie�ny}}\label{def:ciag_zbiezny}\index{ci�g!zbie�ny} do $g \in \mathbb{R}$ ($g$ jest \textbf{\emph{granic� ci�gu}}\index{granica!ci�gu}) je�li

\begin{center}
dla dowolnego $\varepsilon > 0$ istnieje $\delta \in \mathbb{R}$ takie, �e dla dowolnego $n > \delta$ zachodzi $\left| a_n - g \right| < \varepsilon$,
\end{center}

\noindent czyli

\[
\lim_{n \to \infty} a_n = g \; \textsl{ lub } \; a_n \to g \quad \Leftrightarrow \quad \forall \: \varepsilon\!>\!0 \;\;\; \exists \: \delta\!\in\!\mathbb{R} \;\;\; \forall \: n\!>\!\delta \quad \left|a_n - g \right| < \varepsilon
\]

\begin{center}
\includegraphics[scale=1.5]{img/ciagi-ciag_zbiezny.png}
\end{center}

\smallskip

\cite[Definicja 4.4]{ptak}
\end{definicja}

\bigskip

\begin{definicja}
M�wimy, �e ci�g $\left(a_n\right)$ jest \textbf{\emph{rozbie�ny do plus niesko�czono�ci}}\index{ci�g!rozbie�ny do plus niesko�czono�ci} wtedy i~tylko wtedy, gdy

\[
\forall \: M \quad \exists \: \delta \quad \forall \: n > \delta \qquad a_n > M
\]

\smallskip

\cite[Rozdzia� 2.1]{zakowski1}
\end{definicja}

\bigskip

\begin{definicja}
M�wimy, �e ci�g $\left(a_n\right)$ jest \textbf{\emph{rozbie�ny do minus niesko�czono�ci}}\index{ci�g!rozbie�ny do minus niesko�czono�ci} wtedy i~tylko wtedy, gdy

\[
\forall \: M \quad \exists \: \delta \quad \forall \: n > \delta \qquad a_n < M
\]

\smallskip

\cite[Rozdzia� 2.1]{zakowski1}
\end{definicja}

\bigskip
%%%%%%%%%%%%%%%%%%%%%%%%%%%%%%%
\section{Twierdzenia o ci�gach}

\begin{twierdzenie}[Zbie�no�� ci�g�w a dzia�ania]
Niech
\begin{itemize}
\item $\lim_{n \to \infty} a_n \; = \; a \: \in \: \mathbb{R}$,
\item $\lim_{n \to \infty} b_n \; = \; b \: \in \: \mathbb{R}$
\item oraz $\lambda \: \in \: \mathbb{R}$.
\end{itemize}

\medskip

\noindent Wtedy

\begin{itemize}
\item $\lim_{n \to \infty} \left(a_n \pm b_n \right) \; = \; a \pm b$
\item $\lim_{n \to \infty} \left(\lambda \: a_n\right) \; = \; \lambda \: a_n$
\item $\lim_{n \to \infty} \left(a_n \cdot b_n\right) \; = \; a_n \cdot b_n$
\item $\lim_{n \to \infty} \left(\dfrac{a_n}{b_n}\right) \; = \; \dfrac{a}{b}, \qquad b_n \neq 0, \quad b \neq 0$
\end{itemize}

\smallskip

\cite[Twierdzenie 4.14]{ptak}
\end{twierdzenie}

\bigskip

\begin{twierdzenie}
Je�li ci�g $\left(a_n\right)$ jest \textbf{zbie�ny} (Def. \ref{def:ciag_zbiezny}, str. \pageref{def:ciag_zbiezny}), to jest \textbf{ograniczony} (Def. \ref{def:ciag_ograniczony}, str. \pageref{def:ciag_ograniczony}).

\smallskip

\cite[Twierdzenie 4.18]{ptak}
\end{twierdzenie}

\bigskip

\begin{twierdzenie}[o trzech ci�gach]
Niech b�d� dane trzy ci�gi 

\[
\left(a_n\right), \left(b_n\right), \left(c_n\right)
\]

\noindent takie, �e

\[
a_n \leq b_n \leq c_n
\]

\noindent dla ka�dego $n > \delta \in \mathbb{N}$ oraz

\[
\lim_{n \to \infty} a_n \quad = \lim_{n \to \infty} c_n \quad = \quad g
\]

\noindent to $\left(b_n\right)$ jest \textbf{\emph{zbie�ny}}, a

\[
\lim_{n \to \infty} b_n \quad = \quad g
\]

\smallskip

\cite[Twierdzenie 4.21]{ptak}
\end{twierdzenie}

\bigskip

\begin{twierdzenie}
Ka�dy ci�g

\begin{itemize}
\item \textbf{rosn�cy} (Def. \ref{def:ciag_rosnacy}, str. \pageref{def:ciag_rosnacy})
\item i \textbf{ograniczony od g�ry} (Def. \ref{def:ciag_ograniczony_od_gory}, str. \pageref{def:ciag_ograniczony_od_gory})
\end{itemize}

\noindent jest \textbf{zbie�ny} (Def. \ref{def:ciag_zbiezny}, str. \pageref{def:ciag_zbiezny}).

\smallskip

\cite[Twierdzenie 4.23]{ptak}
\end{twierdzenie}

\bigskip

\begin{twierdzenie}
Ka�dy ci�g

\begin{itemize}
\item \textbf{malej�cy} (Def. \ref{def:ciag_malejacy}, str. \pageref{def:ciag_malejacy})
\item i \textbf{ograniczony od do�u} (Def. \ref{def:ciag_ograniczony_od_dolu}, str. \pageref{def:ciag_ograniczony_od_dolu})
\end{itemize}

\noindent jest \textbf{zbie�ny} (Def. \ref{def:ciag_zbiezny}, str. \pageref{def:ciag_zbiezny}).

\smallskip

\cite[Twierdzenie 4.23]{ptak}
\end{twierdzenie}

\chapter{Funkcje}

Definicj� funkcji Czytelnik znajdzie na stronie \pageref{def:funkcja}, a podstawowe definicje zwi�zane z ni� na stronie \pageref{podstawowe_definicje_dla_funkcji}.

\bigskip
%%%%%%%
\section{Wykres funkcji}

Funkcj� liczbow� mo�na interpretowa� geometrycznie sporz�dzaj�c tzw. \textbf{\emph{wykres funkcji}} (Def. \ref{def:wykres_funkcji}, str. \pageref{def:wykres_funkcji}).

Na p�aszczy�nie kartezja�skiego uk�adu wsp�rz�dnych zaznaczamy punkty o wsp�rz�dnych $\left(x, f(x)\right)$, kt�rych zbi�r, gdy $x$ przyjmuje wszystkie warto�ci dziedziny funkcji (Def. \ref{def:dziedzina_funkcji}, str. \pageref{def:dziedzina_funkcji}) $f$, stanowi wykres funkcji.

\includegraphics[scale=1.5]{img/funkcje-wykres_funkcji.png}

\cite[Rozdzia� 1.7]{zakowski1}

\bigskip
%%%%%%%
\section{Wykres funkcji odwrotnej}

Niech
\[
y = f(x)
\]
oraz
\[
y = f^{-1}(x)
\]
(Def. \ref{def:relacja_odwrotna}, str. \pageref{def:relacja_odwrotna})

\medskip

Wykres funkcji $f$ i wykres funkcji odwrotnej $f^{-1}$ s� symetrycznie po�o�one wzgl�dem dwusiecznej k�ta zawartego mi�dzy dodatnimi p�osiami wsp�rz�dnych

\includegraphics[scale=1.5]{img/funkcje-funkcja_odwrotna.png}

\cite[Paragraf 4.7]{krysicki1}

\bigskip
%%%%%%%
\section{Parzysto��. Nieparzysto��}

TODO Df

\begin{definicja}
Funkcj� $f \colon \mathbb{R} \supset Df \rightarrow \mathbb{R}$ nazywamy \textbf{\emph{parzyst�}}\label{def:funkcja_parzysta}, je�li
\[
\forall x \in Df \colon \qquad -x \in Df \quad \wedge \quad f(-x) = f(x)
\]
\end{definicja}

\medskip

\begin{definicja}
Funkcj� $f \colon \mathbb{R} \supset Df \rightarrow \mathbb{R}$ nazywamy \textbf{\emph{nieparzyst�}}\label{def:funkcja_nieparzysta}, je�li
\[
\forall x \in Df \colon \qquad -x \in Df \quad \wedge \quad f(-x) = -f(x)
\]
\end{definicja}

\cite[Rozdzia� 5.2]{ptak}


\bigskip
%%%%%%%%%%%%%%%%%%%%%%%%%
\section{Granice funkcji}

%%%%%%%%%%%%%%%%%%%%%%%%%%%%%%%%%%%%%
\subsection{Granica lewostronna funkcji}

\begin{definicja}[Cauchy'ego]
M�wimy, �e liczba $g$ jest \textbf{\emph{granic� lewostronn� funkcji}}\label{def:granica_lewostronna_funkcji} $f(x)$ w punkcie $x = x_0$, co zapisujemy

\[
\lim_{x \to x_0^-} f(x) \; = \; g
\]

je�eli dla ka�dego $\varepsilon > 0$ istnieje taka liczba $\delta > 0$, �e

\[
\left| f(x) - g \right| < \varepsilon \qquad \textsl{ dla } \qquad x_0 - \delta < x < x_0
\]

\medskip

Lub symbolicznie

\[
\lim_{x \to x_0^-} f(x) \: = \: g 
\]
\[
\Updownarrow
\]
\[
\forall \varepsilon\!>\!0 \quad \exists \delta\!>\!0 \quad \forall x \colon \qquad x_0 - \delta < x < x_0 \quad \Rightarrow \quad \left| f(x) - g \right| < \varepsilon
\]


\cite[Paragraf 5.1]{krysicki1}
\end{definicja}

\bigskip

\begin{definicja}[Heinego]
M�wimy, �e fukcja $f(x)$ ma w punkcie $x_0$ \textbf{\emph{granic� lewostronn�}} i piszemy

\[
\lim_{x \to x_0^-} f(x) \: = \: g 
\]
\[
\Updownarrow
\]
\[
\forall (x_n) \subset \{x_n \subset Df, x_n < x_0\} \colon \qquad \lim_{n \to \infty} x_n = x_0 \quad \Rightarrow \quad \lim_{n \to \infty} f\left(x_n\right) = g
\]

\cite[Rozdzia� 2.3]{zakowski1} \cite[Twierdzenie 5.33]{ptak} \cite[Paragraf 5.1]{krysicki1}
\end{definicja}

\bigskip
%%%%%%%%%%%%%%%%%%%%%%%%%%%%%%%%%%%%%%
\subsection{Granica prawostronna funkcji}

\begin{definicja}[Cauchy'ego]
M�wimy, �e liczba $g$ jest \textbf{\emph{granic� prawostronn� funkcji}}\label{def:granica_prawostronna_funkcji} $f(x)$ w punkcie $x = x_0$, co zapisujemy

\[
\lim_{x \to x_0^+} f(x) \; = \; g
\]

je�eli dla ka�dego $\varepsilon > 0$ istnieje taka liczba $\delta > 0$, �e

\[
\left| f(x) - g \right| < \varepsilon \qquad \textsl{ dla } \qquad x_0 < x < x_0 + \delta
\]

\medskip

Lub symbolicznie

\[
\lim_{x \to x_0^+} f(x) \: = \: g 
\]
\[
\Updownarrow
\]
\[
\forall \varepsilon\!>\!0 \quad \exists \delta\!>\!0 \quad \forall x \colon \qquad x_0 < x < x_0 + \delta \quad \Rightarrow \quad \left| f(x) - g \right| < \varepsilon
\]


\cite[Paragraf 5.1]{krysicki1}
\end{definicja}

\bigskip

\begin{definicja}[Heinego] TODO Df
M�wimy, �e fukcja $f(x)$ ma w punkcie $x_0$ \textbf{\emph{granic� prawostronn�}} i piszemy

\[
\lim_{x \to x_0^+} f(x) \: = \: g 
\]
\[
\Updownarrow
\]
\[
\forall (x_n) \subset \{x_n \subset Df, x_n > x_0\} \colon \qquad \lim_{n \to \infty} x_n = x_0 \quad \Rightarrow \quad \lim_{n \to \infty} f\left(x_n\right) = g
\]

\cite[Rozdzia� 2.3]{zakowski1} \cite[Twierdzenie 5.33]{ptak} \cite[Paragraf 5.1]{krysicki1}
\end{definicja}

\bigskip
%%%%%%%%%%%%%%%%%%%%%%%%%
\subsection{Granica funkcji}

\begin{definicja}[Cauchy'ego]
M�wimy, �e liczba $g$ jest \textbf{\emph{granic� funkcji}}\label{def:granica_funkcji} $f(x)$ w punkcie $x = x_0$, co zapisujemy

\[
\lim_{x \to x_0} f(x) \; = \; g
\]

je�eli dla ka�dego $\varepsilon > 0$ istnieje taka liczba $\delta > 0$, �e

\[
\left| f(x) - g \right| < \varepsilon \qquad \textsl{ dla } \qquad \left|x - x_0\right| < \delta
\]

\medskip

Lub symbolicznie

\[
\lim_{x \to x_0} f(x) \: = \: g 
\]
\[
\Updownarrow
\]
\[
\forall \varepsilon\!>\!0 \quad \exists \delta\!>\!0 \quad \forall x \colon \qquad \left|x - x_0\right| < \delta \quad \Rightarrow \quad \left| f(x) - g \right| < \varepsilon
\]


\cite[Paragraf 5.1]{krysicki1}
\end{definicja}

\bigskip

\begin{definicja}[Heinego] TODO (Df)
M�wimy, �e fukcja $f(x)$ ma w punkcie $x_0$ \textbf{\emph{granic�}} i piszemy

\[
\lim_{x \to x_0} f(x) \: = \: g 
\]
\[
\Updownarrow
\]
\[
\forall (x_n) \subset Df\setminus \{x_0\} \colon \qquad \lim_{n \to \infty} x_n = x_0 \quad \Rightarrow \quad \lim_{n \to \infty} f\left(x_n\right) = g
\]

\cite[Rozdzia� 2.3]{zakowski1} \cite[Twierdzenie 5.33]{ptak} \cite[Paragraf 5.1]{krysicki1}
\end{definicja}


\bigskip
%%%%%%%%%%%%%%%%%%%%%%%%%%%%%%%%%%%%%%%%%%%%%%
\subsection{Interpretacja geometryczna granic}

Zapis
\[
\lim_{x \to x_0^-}
\]

\noindent geometrycznie oznacza, �e jakikolwiek we�miemy w�ski pasek

\begin{equation}
\label{igg:pasek_l} g - \varepsilon \; < \; y \; < \; g + \varepsilon
\end{equation}

\noindent to musi istnie� takie \emph{otoczenie lewostronne} punktu $x = x_0$, czyli taki przedzia�

\begin{equation}
\label{igg:otoczenie_l} x_0 - h \;< \;x \;< \;x_0, \qquad \textsl{ gdzie } \quad h > 0
\end{equation}

\noindent �e ca�y wykres funkcji dla $x$ z przedzia�u (\ref{igg:otoczenie_l}) znajduje si� w pasku (\ref{igg:pasek_l}).

\includegraphics[scale=1.5]{img/funkcje-interpretacja_granicy-l.png}

\bigskip

Zapis
\[
\lim_{x \to x_0^+}
\]

\noindent geometrycznie oznacza, �e jakikolwiek we�miemy w�ski pasek

\begin{equation}
\label{igg:pasek_p} g - \varepsilon \; < \; y \; < \; g + \varepsilon
\end{equation}

\noindent to musi istnie� takie \emph{otoczenie prawostronne} punktu $x = x_0$, czyli taki przedzia�

\begin{equation}
\label{igg:otoczenie_p} x_0 \;< \;x \;< \;x_0 + h, \qquad \textsl{ gdzie } \quad h > 0
\end{equation}

\noindent �e ca�y wykres funkcji dla $x$ z przedzia�u (\ref{igg:otoczenie_p}) znajduje si� w pasku (\ref{igg:pasek_p}).

\includegraphics[scale=1.5]{img/funkcje-interpretacja_granicy-p.png}

\cite[Paragraf 5.2]{krysicki1}

\bigskip
%%%%%%%%%%%%%%%%%%%%%%%%%%%%%%%%%%%%
\subsection{Twierdzenia o granicach}

\begin{twierdzenie}[granice a dzia�ania]
Niech $\lim_{x \to x_0} f(x) = a$, $\lim_{x \to x_0} g(x) = b$ oraz $a, \:b, \:\lambda \: \in \mathbb{R}$. Wtedy

\begin{itemize}
\item $\lim_{x \to x_O} \left(f(x) \pm g(x)\right) \; = \; a \pm b$
\item $\lim_{x \to x_O} \lambda f(x) \; = \; \lambda a$
\item $\lim_{x \to x_O} \left(f(x) \cdot g(x)\right) \; = \; ab$
\item $\lim_{x \to x_O} \dfrac{(f(x)}{g(x)} \; = \; \dfrac{a}{b}$, je�li $b \neq 0$
\end{itemize}

\cite[Twierdzenie 5.37]{ptak}
\end{twierdzenie}

\bigskip

\begin{twierdzenie}[l'H\^{o}spitala]
TODO
\end{twierdzenie}

\bigskip
%%%%%%%%%%%%%%%%%%
\section{Ci�g�o��}

%%%%%%%%%%%%%%%%%%%%%%%%%%%%%%%%%
\subsection{Definicja Cauchy'ego}

\begin{definicja}[Cauchy'ego]
M�wimy, �e funkcja $f$ jest \textbf{\emph{ci�g�a w punkcie $x_0$}}\label{def:funkcja_ciagla_w_punkcie} wtedy i~tylko wtedy, gdy (TODO Df)

\begin{tabular*}{\textwidth}%
{@{\extracolsep{\stretch{1}}}lc}
& \\
$\forall \varepsilon > 0 \quad \exists \delta > 0 \quad \forall x \in Df$
& \vspace{0.3cm} \\
\multicolumn{2}{c}{
$|x - x_0| < \delta \quad \Rightarrow \quad |f(x) - f(x_0)| < \varepsilon$
\vspace{0.3cm}
}
\end{tabular*}

\includegraphics[scale=1.5]{img/funkcje-funkcja_ciagla.png}

\end{definicja}

\medskip
\cite[Rozdzia� 2.4]{zakowski1}


\bigskip
%%%%%%%%%%%%%%%%%%%%%%%%%%%%%%%%%%%%%%%%
\subsection{Definicja za pomoc� granicy}

\begin{twierdzenie}
Funkcj� $f(x)$ nazywamy \textbf{\emph{ci�g��}} w~punkcie $x_0$ wtedy i~tylko wtedy, gdy

\[ 
\exists \; \lim_{x \to x_0} f(x)
\]

oraz
\[
\lim_{x \to x_0} f(x) \quad = \quad f\left(x_0\right)
\]


\cite[Twierdzenie 5.36]{ptak} \cite[Paragraf 5.4]{krysicki1}
\end{twierdzenie}


\bigskip
%%%%%%%%%%%%%%%%%%%%%%%%%%%
\subsection{Funkcja ci�g�a}

\begin{definicja}
M�wimy, �e \textbf{\emph{funkcja}} $f$ jest \textbf{\emph{ci�g�a}}\label{def:funkcja_ciagla} wtedy i~tylko wtedy, gdy jest \textbf{ci�g�a w~ka�dym punkcie} (Def. \ref{def:funkcja_ciagla_w_punkcie}, str. \pageref{def:funkcja_ciagla_w_punkcie}) swej \textbf{dziedziny}.

\cite[Rozdzia� 2.4]{zakowski1}
\end{definicja}


\bigskip
%%%%%%%%%%%%%%%%%%%%%%%%%%%%%%%%%%%%%%
\subsection{Funkcja ci�g�a na zbiorze}

\begin{definicja}
M�wimy, �e \textbf{\emph{funkcja}} $f$ jest \textbf{\emph{ci�g�a na zbiorze}}\label{def:funkcja_ciagla_na_zbiorze} $A \subset D_f$ (TODO Df) wtedy i~tylko wtedy, gdy jest \textbf{ci�g�a w~ka�dym punkcie} (Def. \ref{def:funkcja_ciagla_w_punkcie}, str. \pageref{def:funkcja_ciagla_w_punkcie}) \textbf{zbioru} $A$.

\cite[Rozdzia� 2.4]{zakowski1}
\end{definicja}
\chapter{Rachunek r�niczkowy funkcji jednej zmiennej}

%%%%%%%%%%%%%%%%%%%%%%%%%%
\section{Pochodna funkcji}

\includegraphics[scale=1.5]{img/pochodna-pochodna_funkcji.png}

\begin{definicja}
Funkcja
\[
f \colon (a,b) \to \mathbb{R}
\]
ma \textbf{\emph{pochodn� w punkcie $x_0$}}\label{def:pochodna_funkcji} wtedy i tylko wtedy, gdy

\[
\exists \lim_{x \to x_0} \dfrac{f(x) - f(x_0)}{x - x_0}
\]

\bigskip

Pochodn� funkcji oznaczamy
\[
y', \quad f'(x_0), \quad \dfrac{dy}{dx}, \quad \dfrac{df(x_0)}{dx}, \quad \dot y
\]

\medskip

czyli

\[
f'(x_0) = \lim_{x \to x_0} \dfrac{f(x) - f(x_0)}{x - x_0}
\]

\medskip

lub u�ywaj�c $h = x - x_0$

\[
f'(x_0) = \lim_{h \to 0} \dfrac{f(x_0+h) - f(x_0)}{h}
\]

\cite[Definicja 6.1]{ptak}
\end{definicja}

\bigskip
%%%%%%%%%%%%%%%%%%%%%%%%%%%%%%%%%
\subsection{Pochodna wy�szego rz�du}

\begin{definicja}
Niech funkcja $f \colon (a,b) \to \mathbb{R}$ b�dzie \textbf{r�niczkowalna} w~$(a,b)$, a~$x_0 \in (a,b)$.

\medskip

M�wimy, �e 
\begin{itemize}
\item $f$ jest \textbf{\emph{dwukrotnie r�niczkowalna}} w punkcie $x_0$,
\end{itemize}

\noindent lub �e

\begin{itemize}
\item $f$ ma \textbf{\emph{pochodn� drugiego rz�du}}\label{def:pochodna_drugiego_rzedu} w punkcie $x_0$,
\end{itemize}

\medskip

\noindent je�li
\begin{center}
funkcja $f' \colon (a,b) \to \mathbb{R}$ jest \textbf{r�niczkowalna} w $x_0$,
\end{center}

\medskip

\noindent co oznacza, �e

\[
\exists \left(f'\right)'(x_0) \quad \colon \quad f''(x_0) = \left(f'\right)'(x_0)
\]

\medskip

\cite[Definicja 6.36]{ptak} \cite[Rozdzia� 2.9]{zakowski1}
\end{definicja}

\bigskip

\begin{definicja}
Niech funkcja $f \colon (a,b) \to \mathbb{R}$ b�dzie \textbf{($n - 1$)-krotnie r�niczkowalna} w~$(a,b)$, a~$x_0 \in (a,b)$.

\medskip

M�wimy, �e 

\begin{itemize}
\item $f$ jest \textbf{\emph{$n$-krotnie r�niczkowalna}} w punkcie $x_0$,
\end{itemize}

\noindent lub �e

\begin{itemize}
\item $f$ ma \textbf{\emph{pochodn� $n$-tego rz�du}}\label{def:pochodna_n-tego_rzedu} w punkcie $x_0$,
\end{itemize}

\medskip

\noindent je�li

\begin{center}
funkcja $f^{(n-1)} \colon (a,b) \to \mathbb{R}$ jest \textbf{r�niczkowalna} w $x_0$,
\end{center}

\medskip

\noindent co oznacza, �e

\[
\exists \left(f^{(n-1)}\right)'(x_0) \quad \colon \quad f^{(n)}(x_0) = \left(f^{(n-1)}\right)'(x_0)
\]

\medskip

\cite[Definicja 6.37]{ptak} \cite[Rozdzia� 2.9]{zakowski1}
\end{definicja}

\bigskip
%%%%%%%%%%%%%%%%%%%%%%%%%%%%%%%%%%%%
\subsection{Interpretacja geometryczna}

\includegraphics[scale=1.5]{img/pochodna-interpretacja_geometryczna.png}

Geometrycznie \textbf{\emph{pochodna}} funkcji $y = f(x)$ w punkcie $x_0$ r�wna si� \textbf{wp�czynnikowi k�towemu stycznej} (tangensowi k�ta $\alpha$, kt�ry prosta tworzy z dodatnim zwrotem osi $Ox$) do wykresu funkcji w tym punkcie.

\cite[Paragraf 6.1]{krysicki1}

\begin{twierdzenie}
Je�eli funkcja $f \colon (a,b) \to \mathbb{R}$ jest \textbf{r�niczkowalna} w $x_0 \in (a,b)$, to \textbf{styczna} do wykresu w punkcie $\left(x_0, f(x_0)\right)$ wyra�a si� wzorem:

\[
y - f(x_0) = f'(x_0)(x-x_0)
\]

\medskip

\cite[Wniosek 6.2]{ptak}
\end{twierdzenie}


%\bigskip
\newpage
%%%%%%%%%%%%%%%%%%%%%%%%%%%%%%
\subsection{Obliczanie pochodnej}

TODO marginesy

\[
\begin{array}{|rclc|l|}\hline
(c)' & = & 0 & \hspace{1cm} & c \textsl{ - sta�a} \\ \hline
(x)' & = & 1 & & \\ \hline
\left(x^a\right)' & = & a\;x^{a-1} & & x > 0, \quad a \in \mathbb{R}, \; a \neq 0 \\ \hline
\left(x^n\right)' & = & n\;x^{n-1} & & n \in \mathbb{N^+} \\ \hline
\left(\sqrt{x} \right)' & = & \dfrac{1}{2\sqrt{x}} & & x > 0 \\ \hline
\left(\dfrac{1}{x} \right)' & = & - \dfrac{1}{x^2} & & x \in \mathbb{R} \setminus \{0\} \\ \hline
\left(\sqrt[n]{x} \right)' & = & \dfrac{1}{n \; \sqrt[n]{x^{n-1}}} & & n \textsl{ - nieparzyste, } x \in \mathbb{R} \setminus \{0\} \\ \hline
\left(\sqrt[n]{x} \right)' & = & \dfrac{1}{n \; \sqrt[n]{x^{n-1}}} & & n \textsl{ - parzyste, } x > 0 \\ \hline
\left(\sin{x} \right)' & = & \cos{x} & &  \\ \hline
\left(\cos{x} \right)' & = & -\sin{x} & &  \\ \hline
\left(\tan{x} \right)' & = & \dfrac{1}{\cos^{2}{x}} & & x \in \mathbb{R} \setminus \left\{\frac{\pi}{2} + k\pi \colon k \in \mathbb{Z} \right\} \\ \hline
\left(\cot{x} \right)' & = & - \dfrac{1}{\sin^{2}{x}} & & x \in \mathbb{R} \setminus \left\{k\pi \colon k \in \mathbb{Z} \right\} \\ \hline
\left(\arcsin{x} \right)' & = & \dfrac{1}{\sqrt{1-x^2}} & & x \in (-1, 1) \\ \hline
\left(\arccos{x} \right)' & = & -\dfrac{1}{\sqrt{1-x^2}} & & x \in (-1, 1) \\ \hline
\left(\arctan{x} \right)' & = & \dfrac{1}{1 + x^2} & & x \in \left(-\frac{\pi}{2}, \frac{\pi}{2}\right) \\ \hline
\left(\arccot{x} \right)' & = & - \dfrac{1}{1 + x^2} & & x \in \left(0, \pi\right) \\ \hline
\left(\mathrm{e}^x \right)' & = & \mathrm{e}^x & &  \\ \hline
\left(\mathrm{a}^x \right)' & = & \mathrm{a}^x \; \ln{a} & & a > 0 \\ \hline
\left(\ln{x} \right)' & = & \dfrac{1}{x} & & x > 0 \\ \hline
\end{array}
\]

\cite[Paragraf 6.1]{krysicki1} \cite[Twierdzenie 6.13]{ptak} \cite[Rozdzia� 2.8, tabela 2.2a, 2.2b]{zakowski1}

\medskip

\begin{twierdzenie}[o dzia�aniach arytemtycznych na pochodnych]
Niech funkcje $f, g \colon (a,b) \to \mathbb{R}$ b�d� r�niczkowalne (Def. \ref{def:funkcja_rozniczkowalna}, str. \pageref{def:funkcja_rozniczkowalna}) w $x_0$, a $\lambda \in \mathbb{R}$. Wtedy

TODO odstepy

\[
\begin{array}{rcl}
\left(f \pm g\right)'\left(x_0\right) & = & f'\left(x_0\right) \pm g'\left(x_0\right) \\
\left(f \cdot g\right)'\left(x_0\right) & = & f'\left(x_0\right) g\left(x_0\right) + f\left(x_0\right) g'\left(x_0\right) \\
\left(\lambda \cdot f\right)'\left(x_0\right) & = & \lambda f'\left(x_0\right) \\
\left(\dfrac{f}{g}\right)'\left(x_0\right) & = & \dfrac{f'\left(x_0\right) g\left(x_0\right) - f\left(x_0\right) g'\left(x_0\right)}{g^2\left(x_0\right)}
\end{array}
\]
\cite[Twierdzenie 6.9]{ptak} \cite[Rozdzia� 2.8]{zakowski1}
\end{twierdzenie}

\bigskip

\begin{twierdzenie}[o pochodnej funkcji z�o�onej]
Niech b�d� dane dwie funkcje:

\[
\begin{array}{rlcl}
f \colon & (a,b) & \to & (c,d) \\
g \colon & (c,d) & \to & \mathbb{R}
\end{array}
\]

\medskip

Za��my, �e funkcja

\begin{itemize}
\item $f$ jest \textbf{r�niczkowalna} (Def. \ref{def:funkcja_rozniczkowalna}, str. \pageref{def:funkcja_rozniczkowalna}) w punkcie $x_0$
\item $g$ jest \textbf{r�niczkowalna} w punkcie $y_0 = f\left(x_0\right)$
\end{itemize}

\medskip

Wtedy funkcja $g \circ f\colon (a,b) \to \mathbb{R}$ jest \textbf{r�niczkowalna} w punkcie $x_0$

\[
\left(g \circ f\right)'\left(x_0\right) = g'\Big(f\left(x_0\right)\Big) \cdot f'\left(x_0\right)
\]

albo

\[
\bigg(g\Big(f\left(x_0\right)\Big)\bigg)' = g'\Big(f\left(x_0\right)\Big) \cdot f'\left(x_0\right)
\]

\medskip

\cite[Twierdzenie 6.10]{ptak} \cite[Rozdzia� 2.8]{zakowski1}
\end{twierdzenie}

\bigskip

\begin{twierdzenie}[o pochodnej funkcji odwrotnej]
Niech b�d� dane funkcje

\[
\begin{array}{rlcl}
f \colon & (a,b) & \to & (c,d) \\
f^{-1} \colon & (c,d) & \to & (a,b)
\end{array}
\]

\medskip

Za��my, �e funkcja

\begin{itemize}
\item $f$ jest odwracalna
\item $f^{-1}$ jest funkcj� odwrotn� (Def. \ref{def:relacja_odwrotna}, str. \pageref{def:relacja_odwrotna}) do $f$
\item $f$ jest \textbf{r�niczkowalna} (Def. \ref{def:funkcja_rozniczkowalna}, str. \pageref{def:funkcja_rozniczkowalna}) w punkcie $x_0$
\item $f^{-1}$ jest \textbf{ci�g�a} w punkcie $y_0 = f\left(x_0\right)$
\end{itemize}

\medskip

Wtedy funkcja $f^{-1}$ jest \textbf{r�niczkowalna} w punkcie $y_0$ oraz

\[
\left(f^{-1}\right)' \left(y_0\right) = \dfrac{1}{f'\left(x_0\right)}
\]

\medskip

\cite[Twierdzenie 6.11]{ptak} \cite[Rozdzia� 2.8]{zakowski1}
\end{twierdzenie}

\bigskip

\begin{definicja}
\textbf{\emph{Pochodn� logarytmiczn�}} funkcji $f$ nazywamy pochodn� jej logarytmu naturalnego

\[
\Big(\ln{f(x_0)}\Big)' = \dfrac{f'(x_0)}{f(x_0)}
\]

\medskip

Znaj�c ju� pochodn� logarytmiczn� funkcji mo�emy obliczy� jej pochodn�

\[
f'(x_0) = f(x_0) \cdot \Big(\ln{f(x_0)}\Big)'
\]

\medskip

\cite[Rozdzia� 2.8]{zakowski1}
\end{definicja}

\bigskip
%%%%%%%%%%%%%%%%%%%%%%%%%%%%%%%%%%
\subsection{Twierdzenia o pochodnych}

Niech $f \colon (a,b) \to \mathbb{R}$

\begin{twierdzenie}[r�niczkowalno�� a ci�g�o��]
Je�eli funkcja $f$ jest \textbf{r�niczkowalna} w $x_0 \in (a,b)$, to $f$ jest \textbf{ci�g�a} w $x_0$.
\end{twierdzenie}

\bigskip

\begin{twierdzenie}[o zwi�zku mi�dzy monotoniczno�ci� o pochodn�]
TODO
\end{twierdzenie}

%%%%%%%%%%%%%%%%%%%%%%%%%%%
\section{R�niczka funkcji}

\cite{zakowski1}[Rozdzia� 2.7]

%%%%%%%%%%%%%%%%%%%%%%%%%%%%%%%%%%%%%%%%%%%%%%%%%%%%%%%%%%%
\subsection{Twierdzenie o przedstawieniu przyrostu funkcji}

\begin{twierdzenie}[o przedstawieniu przyrostu funkcji]
Je�eli dziedzina funkcji $f$ zawiera pewne otoczenie $Q$ punktu $x_0$ oraz istnieje pochodna $f'(x_0)$, to dla ka�dego przyrostu $\Delta x$ takiego, �e $x_0 + \Delta x \in Q$, przyrost funkcji

\[
\Delta f = f\left(x_0 + \Delta x\right) - f\left(x_0\right)
\]

\medskip

mo�na przedstawi� nast�puj�co

\[
\Delta f = f'\left(x_0\right) \Delta x + \alpha \Delta x
\]

\medskip

przy czym $\alpha \to 0$, gdy $\Delta x$ d��y do zera w dowolny spos�b.

\medskip

\begin{proof}
Je�eli przyjmiemy

\[
\alpha = \left\{ \begin{array}{ccl}
\dfrac{\Delta f}{\Delta x} - f'(x_0) & \textsl{dla} & \Delta x \neq 0 \\
0 & \textsl{dla} & \Delta x = 0
\end{array} \right.
\]

\medskip

to dla ka�dego (dodatniego, ujemnego, b�d� r�wnego zeru) przyrostu $\Delta x$ takiego, �e $x_0 + \Delta x \in Q$, przedstawienie $\Delta f = f'\left(x_0\right) \Delta x + \alpha \Delta x$ jest prawdziwe.

Je�eli $\Delta x \to 0$, to poniewa� istnieje granica

\[
\lim_{\Delta x \to 0} \dfrac{\Delta f}{\Delta x} = f'\left(x_0\right)
\]

\medskip

wi�c wobec przyj�tego $\alpha$, $\alpha \to 0$, cnd.
\end{proof}
\end{twierdzenie}


\bigskip
%%%%%%%%%%%%%%%%%%%%%%%%%%%%%%%%%%%
\subsection{Funkcja r�niczkowalna}

\begin{definicja}
Funkcj� $f$ nazywamy \textbf{\emph{r�niczkowaln�}}\label{def:funkcja_rozniczkowalna} w punkcie $x_0$, je�eli jej przyrost 

\[
\Delta f = f\left(x_0 + \Delta x\right) - f\left(x_0\right)
\]

\medskip

mo�na dla ka�dego $\Delta x$ dostatecznie bliskiego zeru przedstawi� w postaci

\[
\Delta f = A \Delta x + o\left(\Delta x\right)
\]

\medskip

gdzie

\begin{itemize}
\item $A$ jest sta��,
\item a $o\left(\Delta x\right)$ jest niesko�czenie ma�� (Def. \ref{def:nieskonczenie_mala}, str. \pageref{def:nieskonczenie_mala}) rz�du wy�szego ni� $\Delta x$, gdy $\Delta x \to 0$.
\end{itemize}
\end{definicja}

\medskip

\emph{Wniosek}. Z twierdzenia o przedstawieniu przyrostu funkcji wynika, �e je�eli \textbf{istnieje} $f'(x_0)$, to funkcja $f$ jest w punkcie $x_0$ \textbf{r�niczkowalna}, przy czym $A = f'(x_0)$.

Na odwr�t, je�eli funkcja $f$ jest \textbf{r�niczkowalna} w punkcie $x_0$, to \textbf{istnieje} $f'(x_0) = A$.

\medskip

\emph{Wniosek}. Funkcja $f$ ma wi�c \textbf{pochodn�} w punkcie $x_0$ wtedy i tylko wtedy, gdy jest w tym punkcie \textbf{r�niczkowalna}, przy czym w�wczas

\[
\Delta f = f'\left(x_0\right) \Delta x + o\left(\Delta x\right)
\]

\medskip

dla ka�dego $\Delta x$ dostatecznie bliskiego zeru.

\bigskip
%%%%%%%%%%%%%%%%%%%%%%%%%%%%%%
\subsection{R�niczka funkcji}

\begin{definicja}
\textbf{\emph{R�niczk� funkcji}}\label{def:rozniczka_funkcji} $f$ w punkcie $x_0$ i dla przyrostu $\Delta x$ zmiennej niezale�nej $x$ nazywamy iloczyn

\[
f'(x_0) \Delta x
\]

\medskip

R�niczk� oznaczamy symbolem $df(x_0)$, b�d� te� kr�tko $df$ lub $dy$.

Mamy wi�c

\[
df(x_0) \; \stackrel{df}{=} \; f'(x_0) \Delta x
\]

lub kr�tko 

\[
dy \; \stackrel{df}{=} \; f'(x_0) \Delta x
\]
\end{definicja}


\bigskip
%%%%%%%%%%%%%%%%%%%%%%%%%%%%%%%%%%%%%%%%%%
\subsubsection{Interpretacja geometryczna}

\vspace{5cm}
TODO 2.24 str 106

\chapter{Badanie przebiegu zmienno�ci funkcji}

%%%%%%%%%%%%%%%%%%%%%%%%%%
\section{Ekstrema funkcji}

\cite{zakowski1}[Rozdzia� 2.13]

\medskip

\begin{definicja}
M�wimy, �e funkcja $f\colon \: (a,b) \to \mathbb{R}$ ma w punkcie $x_0 \in (a,b)$

\begin{itemize}
\item \textbf{\emph{maksimum lokalne}}\label{def:maksimum_lokalne}, je�li istnieje otoczenie punktu $x_0$ (Def. \ref{def:otoczenie_punktu}, str. \pageref{def:otoczenie_punktu}) takie, �e dla ka�dego $x$ z tego otoczenia mamy

\[
f(x_0) \geq f(x)
\]

\item \textbf{\emph{mocne maksimum lokalne}}\label{def:mocne_maksimum_lokalne}, je�li istnieje otoczenie punktu $x_0$ (Def. \ref{def:otoczenie_punktu}, str. \pageref{def:otoczenie_punktu}) takie, �e dla ka�dego $x$ z tego otoczenia oraz $x \neq x_0$ mamy

\[
f(x_0) > f(x)
\]

\item \textbf{\emph{minimum lokalne}}\label{def:minimum_lokalne}, je�li istnieje otoczenie punktu $x_0$ (Def. \ref{def:otoczenie_punktu}, str. \pageref{def:otoczenie_punktu}) takie, �e dla ka�dego $x$ z tego otoczenia mamy

\[
f(x_0) \leq f(x)
\]

\item \textbf{\emph{mocne minimum lokalne}}\label{def:mocne_minimum_lokalne}, je�li istnieje otoczenie punktu $x_0$ (Def. \ref{def:otoczenie_punktu}, str. \pageref{def:otoczenie_punktu}) takie, �e dla ka�dego $x$ z tego otoczenia oraz $x \neq x_0$ mamy

\[
f(x_0) < f(x)
\]
\end{itemize}

\vspace{5cm}
TODO rysunek 2.29, 2.30

\medskip

\cite{ptak}[Definicja 6.45]
\end{definicja}


\bigskip
%%%%%%%%%%%%%%%%%%%%%%%%%%%%%%%%%%%%%%%%%%%%%%%%%%
\subsection{Warunek konieczny istnienia ekstremum}

Niech $f\colon \: (a,b) \to \mathbb{R}$ b�dzie r�niczkowalna (Def. \ref{def:funkcja_rozniczkowalna}, str. \pageref{def:funkcja_rozniczkowlana}) w $x_0 \in (a,b)$.

\medskip

\begin{twierdzenie}[warunek konieczny istnienia ekstremum]
Je�li funkcja $f$ ma \textbf{maksimum} lokalne (\textbf{minimum} lokalne) w punkcie $x_0$, to 

\[
f'(x_0) = 0
\]

(Def. \ref{def:pochodna_funkcji}, str. \pageref{def:pochodna_funkcji}).

\medskip

\begin{proof}
Za��my, �e $f$ osi�ga \textbf{maksimum} lokalne w punkcie $x_0$.

\smallskip

Wtedy dla $x$ z otoczenia punktu $x_0$ (Def. \ref{def:otoczenie_punktu}, str. \pageref{def:otoczenie_punktu}) zachodzi nier�wno��:

\[
f(x) - f(x_0) \neq 0
\]

\medskip

Zatem dla $x_0 - \delta < x < x_0$ zachodzi nier�wno��

\[
\dfrac{f(x) - f(x_0)}{x - x_0} \geq 0
\]

\medskip

Poniewa� funkcja jest r�niczkowalna (Def. \ref{def:funkcja_rozniczkowalna}, str. \pageref{def:funkcja_rozniczkowlana}) w punkcie $x_0$, wi�c z twierdzenia o zachowaniu nier�wno�ci w granicy (Tw. \ref{tw:o_zachowaniu_nierownosci_w_granicy}, str. \pageref{tw:o_zachowaniu_nierownosci_w_granicy}) otrzymujemy:

\[
f'(x_0) = \lim_{x \to x_0} \dfrac{f(x) - f(x_0)}{x - x_0} \geq 0
\]

\medskip

Analogicznie, dla $x_0 < x < x_0 + \delta$ mamy

\[
\dfrac{f(x) - f(x_0)}{x - x_0} \leq 0
\]

\medskip

Rozumuj�c podobnie jak wy�ej, otrzymujemy

\[
f'(x_0) \leq 0
\]

\medskip

czyli

\[
f'(x_0) = 0
\]

\bigskip

Dow�d w przypadku \textbf{minimum} przebiega w taki sam spos�b.
\end{proof}

\medskip

\cite{ptak}[Twierdzenie 6.47]
\end{twierdzenie}

\vspace{5cm}
TODO rys. 2.32 str 132

\bigskip
%%%%%%%%%%%%%%%%%%%%%%%%%%%%%%%%%%%%%%%%%%%%%%%%%%%%%%%%
\subsection{Warunek wystarczaj�cy istnienia ekstremum I}

Niech $f\colon \: (a,b) \to \mathbb{R}$ b�dzie funkcj� klasy $C^{(2)}$ oraz $x_0 \in (a, b)$

\medskip

\begin{twierdzenie}[warunek wystarczaj�cy istnienia ekstremum lokalnego I]
Je�li $f'(x_0) = 0$ oraz

\begin{itemize}
\item $f''(x_0) < 0$, to $f$ ma mocne \textbf{maksimum} lokalne (Def. \ref{def:mocne_maksimum_lokalne}, str. \pageref{def:mocne_maksimum_lokalne}) w~punkcie $x_0$,
\item $f''(x_0) > 0$, to $f$ ma mocne \textbf{minimum} lokalne (Def. \ref{def:mocne_minimum_lokalne}, str. \pageref{def:mocne_minimum_lokalne}) w~punkcie $x_0$,
\end{itemize}

\begin{proof}
TODO sprawdzic, czy byl dowod na wykladzie
\end{proof}

\cite{ptak}[Twierdzenie 6.49]
\end{twierdzenie}

\bigskip
%%%%%%%%%%%%%%%%%%%%%%%%%%%%%%%%%%%%%%%%%%%%%%%%%%%%%%%%%
\subsection{Warunek wystarczaj�cy istnienia ekstremum II}

Niech $f\colon \: (a,b) \to \mathbb{R}$ b�dzie funkcj� klasy $C^{(1)}$ oraz $x_0 \in (a,b)$.

\medskip

\begin{twierdzenie}[warunek wystarczaj�cy istnienia ekstremum lokalnego II]
Je�li istnieje $\delta > 0$ takie, �e:

\begin{itemize}
\item 
\[
\begin{array}{lclrcll}
f'(x) > 0 & \textsl{dla} & x \in (& x_0 - \delta &,& x_0 &) \\
f'(x) < 0 & \textsl{dla} & x \in (& x_0 &,& x_0 + \delta &)
\end{array}
\]
(w skr�cie m�wimy, �e \textbf{pochodna} funkcji \textbf{zmienia znak} z \textbf{dodatniego} na \textbf{ujemny}), to $f$ ma mocne \textbf{maksimum} lokalne (Def. \ref{def:mocne_maksimum_lokalne}, str. \pageref{def:mocne_maksimum_lokalne}) w~punckie $x_0$.

\item 
\[
\begin{array}{lclrcll}
f'(x) < 0 & \textsl{dla} & x \in (& x_0 - \delta &,& x_0 &) \\
f'(x) > 0 & \textsl{dla} & x \in (& x_0 &,& x_0 + \delta &)
\end{array}
\]
(w skr�cie m�wimy, �e \textbf{pochodna} funkcji \textbf{zmienia znak} z \textbf{ujemnego} na \textbf{dodatni}), to $f$ ma mocne \textbf{minimum} lokalne (Def. \ref{def:mocne_minimum_lokalne}, str. \pageref{def:mocne_minimum_lokalne}) w~punckie $x_0$.

\end{itemize}

\medskip

\begin{proof}
TODO sprawdzic czy byl dowod na wykladzie
\end{proof}

\vspace{5cm}
TODO rys. 2.33 str 134

\medskip

\cite{ptak}[Twierdzenie 6.50]
\end{twierdzenie}


\bigskip
%%%%%%%%%%%%%%%%%%%%%%%%%%%%%
\section{Twierdzenie Rolle'a}

\bigskip
%%%%%%%%%%%%%%%%%%%%%%%%%%%%%%%%%%
\section{Twierdzenie Weierstrassa}

\bigskip
%%%%%%%%%%%%%%%%%%%%%%%%%%%%%%%%
\section{Twierdzenie Lagrange'a}

\bigskip
%%%%%%%%%%%%%%%%%%%%%%%%%%%%%
\section{Twierdzenie Taylora}

\bigskip
%%%%%%%%%%%%%%%%%%%%%%%%%%%%%%%%
\section{Twierdzenie Maclaurina}

\bigskip
%%%%%%%%%%%%%%%%%%%%%%%%%%%%%%%%%%%%%%%%%%%%%%%
\section{Wkl�s�o�� i wypuk�o�� wykresu funkcji}

\bigskip
%%%%%%%%%%%%%%%%%%%%%%%%%%
\section{Punkt przegi�cia}

\bigskip
%%%%%%%%%%%%%%%%%%%%%%%%%%%%%%%%%%%%%%%%%%%%%%%%%%%%%%%%%%
\subsection{Warunek konieczny istnienia punktu przegi�cia}

\bigskip
%%%%%%%%%%%%%%%%%%%%%%%%%%%%%%%%%%%%%%%%%%%%%%%%%%%%%%%%%%%%%%
\subsection{Warunek wystarczaj�cy istnienia punktu przegi�cia}

\bigskip
%%%%%%%%%%%%%%%%%%%
\section{Asymptoty}

\bigskip
%%%%%%%%%%%%%%%%%%%%
\subsection{Pionowa}

\bigskip
%%%%%%%%%%%%%%%%%%%%
\subsection{Pozioma}

\bigskip
%%%%%%%%%%%%%%%%%%%
\subsection{Uko�na}

\bigskip
%%%%%%%%%%%%%%%%%%%%%%%%%%%%%%%%%%%%%%%%%%%%%%%%%%%%%%
\section{Schemat badania przebiegu zmienno�ci funkcji}
\chapter{Rachunek ca�kowy funkcji jednej zmiennej}

%%%%%%%%%%%%%%%%%%%%%%%%%%%
\section{Funkcja pierwotna}

\begin{definicja}
\textbf{\emph{Funkcj� pierwotn�}}\label{def:funkcja_pierwotna}\index{funkcja!pierwotna} funkcji $f(x)$ w~przedziale $a < x < b$ nazywamy ka�d� tak� funkcj� $F(x)$, �e

\[
F'(x) \quad = \quad f(x)
\]

\medskip

dla ka�dego $x$ z~przedzia�u $a < x < b$.

\cite[Paragraf 15.1]{krysicki1} \cite[Rozdia� 3.4]{zakowski1}
\end{definicja}

\medskip

\begin{twierdzenie}
Je�eli $F(x)$ i $G(x)$ s� funkcjami pierwotnymi $f(x)$, to

\[
F(x) - G(x) \quad = \quad \textsl{const}
\]

(Dwie funkcje maj�ce w~danym przedziale t� sam� sko�czon� pochodn� \textbf{mog� si� r�ni� co najwy�ej o~sta��})

\cite[Paragraf 15.1]{krysicki1}
\end{twierdzenie}

\medskip

\begin{twierdzenie}[o istnieniu funkcji pierwotnej]
\label{tw:o_istnieniu_funkcji_pierwotnej}
Je�eli funkcja $f$ jest \textbf{ci�g�a} (jest klasy $C^0$) (Def. \ref{def:funkcja_ciagla_na_zbiorze}, str. \pageref{def:funkcja_ciagla_na_zbiorze}) na pewnym przedziale, to \textbf{posiada} na tym przedziale \textbf{funkcj� pierwotn�}.

\cite[Rozdzia� 3.4]{zakowski1}
\end{twierdzenie}


\bigskip

%%%%%%%%%%%%%%%%%%%%%%%%%%%%
\section{Ca�ka nieoznaczona}

\begin{definicja}
\textbf{\emph{Ca�k� nieoznaczon�}}\label{def:calka_nieoznaczona}\index{ca�ka!nieoznaczona} funkcji $f(x)$, oznaczon� symbolem

\[
\int f(x) \: \textnormal{d}x
\]

nazywamy wyra�enie

\[
F(x) + C
\]

gdzie
\begin{itemize}
\item $F(x)$ jest funkcj� pierwotn� funkcji f(x) (Def. \ref{def:funkcja_pierwotna}, str. \pageref{def:funkcja_pierwotna})
\item a $C$ jest dowoln� sta��.
\end{itemize}

Jest wi�c

\[
\int f(x) \: \textnormal{d}x \quad = \quad F(x) + C, \quad \textsl{gdzie} \quad F'(x) = f(x)
\]

\cite[Paragraf 15.1]{krysicki1} \cite[Rozdzia� 3.5]{zakowski1}
\end{definicja}

\medskip

Je�eli funkcja posiada na pewnym przedziale \textbf{funkcj� pierwotn�}, to m�wimy, �e \textbf{jest} ona na tym przedziale \textbf{ca�kowalna} (w~sensie Newtona).

\cite[Rozdzia� 3.4]{zakowski1}

\bigskip

Z powy�szego oraz twierdzenia [\ref{tw:o_istnieniu_funkcji_pierwotnej}, str. \pageref{tw:o_istnieniu_funkcji_pierwotnej}] wynika, �e

\smallskip

je�eli funkcja $f$ jest \textbf{ci�g�a} (Def. \ref{def:funkcja_ciagla_na_zbiorze}, str. \pageref{def:funkcja_ciagla_na_zbiorze}) na pewnym przedziale to jest \textbf{ca�kowalna} na tym przedziale.

\newpage

%%%%%%%%%%%%%%%%%%%%%%%%%%%%%%%%%%%%%%%%
\section{Obliczanie ca�ki nieoznaczonej}

%%%%%%%%%%%%%%%%%%%%%%%%%%%%%
\subsection{Podstawowe wzory}

\begin{center}
\begin{tabular}{|m{2cm}m{0,3cm}m{3cm}|m{3cm}|m{0,1cm}}
\cline{1-4}
 
$\int 0 \: \textnormal{d}x$ &
$=$ &
$C$ &
 &
 \\[10pt] \cline{1-4}
 
$\int x^a \: \textnormal{d}x$ &
$=$ &
$\dfrac{x^{a+1}}{a+1} + C$ &
$a \neq -1, \; x>0$ &
 \\[18pt] 
 
$\int \textnormal{d}x$ &
$=$ &
$x + C$ &
 &
 \\[10pt]
 
$\int\dfrac{\textnormal{d}x}{\sqrt{x}}$ &
$=$ &
$2\sqrt{x} + C$ &
$x>0$ &
 \\[15pt]
 
$\int \dfrac{\textnormal{d}x}{x^2}$ &
$=$ &
$-\dfrac{1}{x} + C$ &
$x \neq 0$ &
 \\[18pt] \cline{1-4}
 
$\int \dfrac{\textnormal{d}x}{x}$ &
$=$ &
$ln|x| + C$ &
$x \neq 0$ &
 \\[18pt] \cline{1-4}
 
$\int a^x \: \textnormal{d}x$ &
$=$ &
$\dfrac{a^x}{\ln a} + C$ &
$a>0, \: a \neq 1$ &
 \\[15pt]
 
$\int e^x \: \textnormal{d}x$ &
$=$ &
$e^x + C$ &
 &
 \\[10pt] \cline{1-4}
 
$\int \cos x \: \textnormal{d}x$ &
$=$ &
$\sin x + C$ &
 &
 \\[10pt] \cline{1-4}

$\int \sin x \: \textnormal{d}x$ &
$=$ &
$- \cos x + C$ &
 &
 \\[10pt] \cline{1-4}
  
$\int \dfrac{\textnormal{d}x}{\cos^2 x}$ &
$=$ &
$\tg x + C$ &
$\cos x \neq 0$ &
 \\[18pt] \cline{1-4}
  
$\int \dfrac{\textnormal{d}x}{\sin^2}$ &
$=$ &
$-\ctg x + C$ &
$\sin x \neq 0$ &
 \\[18pt] \cline{1-4}
 
$\int \dfrac{\textnormal{d}x}{\sqrt{1-x^2}}$ &
$=$ &
$\arcsin x + C$ &
$-1<x<1$ &
 \\[18pt] \cline{1-4}
 
$\int \dfrac{\textnormal{d}x}{\sqrt{1-x^2}}$ &
$=$ &
$- \arccos x + C$ &
$-1<x<1$ &
 \\[18pt] \cline{1-4}
 
   $\int \dfrac{\textnormal{d}x}{x^2 + 1}$ &
$=$ &
$\arc \tg x + C$ &
 &
 \\[18pt] \cline{1-4}
 
$\int \dfrac{\textnormal{d}x}{x^2 + 1}$ &
$=$ &
$-\arc \ctg x + C$ &
 &
 \\[18pt] \cline{1-4}
 
$\int \sinh x \: \textnormal{d}x$ &
$=$ &
$\cosh x + C$ &
 &
 \\[10pt] \cline{1-4}
 
$\int \cosh \: \textnormal{d}x$ &
$=$ &
$\sinh + C$ &
 &
 \\[10pt] \cline{1-4}
  
\end{tabular}
\end{center}

\cite[Paragraf 15.2]{krysicki1}

\bigskip

%%%%%%%%%%%%%%%%%%%%%%%%%%%%%%%%%%%%%%%%%%%
\subsection{W�asno�ci ca�ek nieoznaczonych}

\begin{twierdzenie}[o ca�ce sumy]\label{tw:o_calce_sumy}
Ca�ka sumy r�wna si� sumie ca�ek, tzn. (jest to tzw. addytywno�� ca�ki wzgl�dem funkcji podca�kowej)

\[
\int \Big(f(x) + g(x)\Big) \textnormal{d}x \quad = \quad \int f(x) \: \textnormal{d}x + \int g(x) \: \textnormal{d}x
\]

\cite[Paragraf 15.3, (15.3.1)]{krysicki1}
\end{twierdzenie}

\medskip

\begin{twierdzenie}\label{tw:o_wyniesieniu_stalego_czynnika_przed_znak_calki}
Sta�y czynnik wolno wynie�� przed znak ca�ki, tzn.

\[
\int A \: f(x) \: \textnormal{d}x \quad = \quad A \int f(x) \: \textnormal{d}x, \quad A \neq 0, \; A = \textsl{const}
\]

\cite[Paragraf 15.3, (15.3.2)]{krysicki1}
\end{twierdzenie}

\bigskip
%%%%%%%%%%%%%%%%%%%%%%%%%%%%%%%%%%%%
\subsection{Ca�kowanie przez cz�ci}

\begin{twierdzenie}[o ca�kowaniu przez cz�ci]\label{tw:o_calkowaniu_przez_czesci}
Je�eli funkcje $f(x)$ i~$g(x)$ maj� na pewnym przedziale ci�g�e pochodne $f'(x)$ i~$g'(x)$, to

\[
\int f'(x) \: g(x) \: \textnormal{d}x \quad = \quad f(x)\: g(x) - \int f(x) \: g'(x) \: \textnormal{d}x
\]

\medskip

na tym przedziale

\cite[Rozdzia� 3.6]{zakowski1}
\end{twierdzenie}

\bigskip
%%%%%%%%%%%%%%%%%%%%%%%%%%%%%%%%%%%%%%%%%%
\subsection{Ca�kowanie przez podstawienie}

\begin{twierdzenie}[o ca�kowaniu przez podstawienie]\label{tw:o_calkowaniu_przez_podstawienie}
Je�eli dla $a \: \leq \: x \: \leq \: b$

\begin{itemize}
\item $g(x) = u$ jest funkcj� maj�c� \textbf{ci�g�� pochodn�}
\item oraz $A \: \leq \: g(x) \: \leq \: B$,
\item a funkcja $f(u)$ jest \textbf{ci�g�a} w~przedziale $\langle A,B\rangle$
\end{itemize}

to

\[
\int f\Big(g(x)\Big) \: g'(x) \: \textnormal{d}x \quad = \quad \int f(u) \: du
\]

\medskip

przy czym po sca�kowaniu prawej strony nale�y w~otrzymanym wyniku podstawi� $u \: = \: g(x)$

\cite[Paragraf 15.3, (15.3.4)]{krysicki1}
\end{twierdzenie}

\bigskip
%%%%%%%%%%%%%%%%%%%%%%%%%%%%%%%%%%%%%
\subsection{Ca�ki funkcji wymiernych}

Ca�ka funkcji wymiernej (Def. \ref{def:funkcja_wymierna}, str. \pageref{def:funkcja_wymierna}), to ca�ka postaci

\[
\int \frac{P_n(x)}{Q_m(x)} \: \textnormal{d}x
\]

\medskip

\noindent Przy obliczaniu powy�szej ca�ki nale�y:

\begin{itemize}
\item je�eli funkcja wymierna jest \textbf{niew�a�ciwa}, czyli $m \geq n$ (Def. \ref{def:funkcja_wymierna_niewlasciwa}, str. \pageref{def:funkcja_wymierna_niewlasciwa}), to nale�y \textbf{podzieli� licznik przez mianownik} i~przedstawi� t� funkcj� jako \textbf{sum� wielomianu} oraz \textbf{funkcji wymiernej w�a�ciwej} (Tw. \ref{tw:funkcja_wymierna_suma_wielomianu_i_funkcji_wymiernej_wlasciwej}, str. \pageref{tw:funkcja_wymierna_suma_wielomianu_i_funkcji_wymiernej_wlasciwej}), czyli mamy

\[
\int \dfrac{P_p(x)}{Q_q(x)} \: \textnormal{d}x \quad = \quad \int \Big( W_w(x) \: + \: \dfrac{R_r(x)}{Q_q(x)} \Big) \: \textnormal{d}x
\]

\medskip

nast�pnie skorzysta� z~twierdzenia o~ca�ce sumy (Tw. \ref{tw:o_calce_sumy}, str. \pageref{tw:o_calce_sumy}), st�d mamy

\[
\int \dfrac{P_p(x)}{Q_q(x)} \: \textnormal{d}x \quad = \quad \int W_w(x) \: \textnormal{d}x \; + \; \int \dfrac{R_r(x)}{Q_q(x)} \: \textnormal{d}x
\]

\medskip

Pierwsz� ca�k� obliczamy korzystaj�c z~twierdzenia o~ca�ce sumy, z~twierdzenia o~wyniesieniu sta�ego czynnika przed znak ca�ki (Tw. \ref{tw:o_wyniesieniu_stalego_czynnika_przed_znak_calki}, str. \pageref{tw:o_wyniesieniu_stalego_czynnika_przed_znak_calki}) oraz podstawowych wzor�w na obliczanie ca�ki.

\smallskip

Drug� ca�k� obliczamy w~spos�b podany poni�ej.

\item je�eli funkcja wymierna jest \textbf{w�a�ciwa}, czyli $n < m$ (Def. \ref{def:funkcja_wymierna_wlasciwa}, str. \pageref{def:funkcja_wymierna_wlasciwa}), to przedstawiamy j� jako \textbf{sum� u�amk�w prostych} (Tw. \ref{tw:o_rozkladzie_na_ulamki_proste}, str. \pageref{tw:o_rozkladzie_na_ulamki_proste}).
\end{itemize}

\cite[Paragraf 16.1]{krysicki1}

\bigskip
%%%%%%%%%%%%%%%%%%%%%%%%%%%%%%%%%%%%%%%%%%%%%%%%%%%%%%%%%%%%%%
\subsubsection{Ca�kowanie u�amk�w prostych pierwszego rodzaju}

U�amki proste pierwszego rodzaju (Def. \ref{def:ulamek_prosty_pierwszego_rodzaju}, str. \pageref{def:ulamek_prosty_pierwszego_rodzaju}) ca�kujemy korzystaj�c z~twierdzenia o~ca�kowaniu przez podstawienie (Tw. \ref{tw:o_calkowaniu_przez_podstawienie}, str. \pageref{tw:o_calkowaniu_przez_podstawienie}) i~podstawienia

\[
u \quad = \quad x - a
\]

\medskip

st�d

\[
\int \frac{A}{(x - a)^n} \: \textnormal{d}x \quad = \quad A \int \frac{1}{u^n} \: du
\]

\medskip

i w ko�cu

\[
\int \frac{A}{(x - a)^n} \: \textnormal{d}x \quad = \quad \left\{ \begin{array}{cllc}
A & \ln |x-a| + C, & \textsl{gdy } r = 1 & \\[4pt]
\dfrac{A}{1-r} & (x-a)^{1-r} + C, &  \textsl{gdy TODO ?? ->} r \geq 2 &
\end{array} \right.
\]

\bigskip
%%%%%%%%%%%%%%%%%%%%%%%%%%%%%%%%%%%%%%%%%%%%%%%%%%%%%%%%%%%%%%
\subsubsection{Ca�kowanie u�amk�w prostych drugiego rodzaju}

U�amki proste drugiego rodzaju (Def. \ref{def:ulamek_prosty_drugiego_rodzaju}, str. \pageref{def:ulamek_prosty_drugiego_rodzaju}) ca�kujemy korzystaj�c, w~pierwszej kolejno�ci, z~twierdzenia o~ca�ce sumy (Tw. \ref{tw:o_calce_sumy}, str. \pageref{tw:o_calce_sumy}), z~twierdzenia o~wyniesieniu sta�ego czynnika przed znak ca�ki (Tw. \ref{tw:o_wyniesieniu_stalego_czynnika_przed_znak_calki}, str. \pageref{tw:o_wyniesieniu_stalego_czynnika_przed_znak_calki}), z~czego uzyskujemy dwie ca�ki

\begin{itemize}
\item pierwsz� postaci

\[
\int \frac{2x + p}{(x^2 +px + q)^n} \: \textnormal{d}x
\]

\medskip

gdzie licznik podca�kowej funkcji wymiernej ma posta� pochodnej funkcji mianownika b�d�cej w~pot�dze

\item drug� postaci

\[
\int \frac{1}{(x^2 +px + q)^n} \: \textnormal{d}x
\]

\end{itemize}

\bigskip

Aby otrzyma� te ca�ki licznik $Ax + B$ u�amka prostego przekszta�camy, w~og�lnym przypadku, nast�puj�co

\[
Ax + B \; = \; \frac{A}{2} \cdot 2x + B \; = \; \frac{A}{2} (2x + p - p) + B \; = \; \frac{A}{2} (2x + p) + B - \frac{Ap}{2}
\]

\bigskip

St�d do sca�kowania u�amka prostego drugiego rodzaju zastosujemy nast�puj�ce przekszta�cenie 

\[
\begin{array}{rcrlcc}
\int \dfrac{Ax + B}{(x^2 + px + q)^n} & = &  \dfrac{A}{2} & \int \dfrac{2x + p}{(x^2 +px + q)^n} \: \textnormal{d}x & + & \\[10pt]
& + & \left(B - \dfrac{Ap}{2}\right) & \int \dfrac{1}{(x^2 +px + q)^n} \: \textnormal{d}x & &
\end{array}
\]

\bigskip

Pierwsz� ca�k� 

\[
\int \frac{2x + p}{(x^2 +px + q)^n} \: \textnormal{d}x
\]

\medskip

obliczamy stosuj�c podstawienie 

\[
t \; = \; x^2 +px + q
\]

\medskip

st�d

\[
\textnormal{d}t \; = \; (2x + p) \: \textnormal{d}x
\]

\medskip

Otrzymujemy wi�c

\[
\int \frac{2x + p}{(x^2 +px + q)^n} \: \textnormal{d}x \quad = \quad \int \frac{1}{t^n} \: \textnormal{d}t
\]

\medskip

Dalsze ca�kowanie przebiega tak, jak w~przypadku u�amka prostego.

\bigskip

Drug� ca�k�

\[
\int \frac{1}{(x^2 +px + q)^n} \: \textnormal{d}x
\]

\medskip

przekszta�camy tak, aby otrzyma�

\[
\int \frac{1}{(t^2 + 1)^n} \: \textnormal{d}t
\]

\medskip


w~og�lnym przypadku stosuj�c podstawienie

\[
t \; = \; \frac{2x +p}{\sqrt{-\Delta}}
\]

\cite[Rozdzia� 3.8]{zakowski1}

\bigskip

Ca�k� 

\[
\int \frac{1}{(t^2 + 1)^n} \: \textnormal{d}t
\]

dla 

\begin{itemize}
\item $n = 1$ rozwi�zujemy korzystaj�c z podstawowych wzor�w ca�kowania

\[
\int \frac{1}{t^2 + 1} \: \textnormal{d}t \; = \; \arc \tg{t} + C
\]

\item $n > 1$ rozwi�zujemy korzystaj�c ze \textbf{wzoru redukuj�cego} (lub \textbf{rekurencyjnego})

\[
I_n = \frac{1}{2n-2} \cdot \frac{t}{(t^2 + 1)^{n-1}} + \frac{2n-3}{2n-2} \; I_{n-1}, \qquad \textsl{gdzie } I_n = \int \frac{1}{(t^2 + 1)^n} \: \textnormal{d}t
\]

\medskip

kt�rego wyprowadzenie znajduje si� poni�ej.

\bigskip

\textbf{Wyprowadzenie}

\medskip

Aby obliczy� ca�k�

\[
I_n = \int \frac{1}{(t^2 + 1)^n} \: \textnormal{d}t
\]

\medskip

zastosujemy przekszta�cenie

\[
\int \frac{1}{(t^2 + 1)^n} \: \textnormal{d}t \; = \; \int \frac{t^2 + 1 - t^2}{(t^2 + 1)^n} \: \textnormal{d}t \; =
\]

\[
 = \; \int \frac{1}{(t^2 + 1)^{n-1}} \: \textnormal{d}t \; - \; \int \frac{t^2}{(t^2 + 1)^n} \: \textnormal{d}t
\]

\medskip

czyli mamy

\[
I_n \; = \; I_{n-1} \; - \; \int \frac{t^2}{(t^2 + 1)^n} \: \textnormal{d}t
\]

\medskip

Drug� ca�k� zapisujemy jako

\[
\int \frac{t^2}{(t^2 + 1)^n} \: \textnormal{d}t \; = \; \int t \cdot \frac{t}{(t^2 + 1)^n} \: \textnormal{d}t
\]

\medskip

i~ca�kujemy przez cz�ci (Tw. \ref{tw:o_calkowaniu_przez_czesci}, str. \pageref{tw:o_calkowaniu_przez_czesci}). St�d otrzymujemy

\[
\int \frac{t^2}{(t^2 + 1)^n} \: \textnormal{d}t \; = \; \frac{-1}{2n -2} \cdot \frac{t}{(t^2 +1)^{n-1}} + \frac{1}{2n-2} \; I_{n-1}
\]

\medskip

Po podstawieniu do wzoru na $I_n$ otrzymujemy zapisany wy�ej \textbf{wz�r redukcyjny}.

\cite[Zadanie 16.21]{krysicki1}
\end{itemize}

\bigskip
%%%%%%%%%%%%%%%%%%%%%%%%%%%%%%%%%%%%%%%%%%%%%
\subsection{Ca�ki funkcji trygonometrycznych}

Ca�ki trygonometryczne obliczamy stosuj�c podstawienia charakterystyczne dla danych typ�w.

\begin{itemize}
\item 
\[
\int R(\sin x) \cos x \: \textnormal{d}x
\]

\begin{eqnarray*}
t 	& = & \sin x \\
\textnormal{d}t	& = & \cos x \: \textnormal{d}x
\end{eqnarray*}

\item

\[
\int R(\cos x) \sin x \: \textnormal{d}x
\]

\begin{eqnarray*}
t 	& = & \cos x \\
- \textnormal{d}t	& = & \sin x \: \textnormal{d}x
\end{eqnarray*}

\item

\[
\int R(\sin^2 x, \cos^2 x, \sin x \cos x) \: \textnormal{d}x
\]

\begin{eqnarray*}
t 							& = & \tg x \\
\arc \tg t					& = & x \\
\frac{1}{1 + t^2} \: \textnormal{d}t		& = & \textnormal{d}x
\end{eqnarray*}

\[
\sin^2 x \; = \; \frac{\sin^2 x}{1} \; = \; \frac{\sin^2 x}{\cos^2 x + \sin^2 x} \; = \; \frac{\tg^2 x}{1 + \tg^2 x} \; = \; \frac{t^2}{1 + t^2}
\]

\[
\cos^2 x \; = \; \frac{\cos^2 x}{1} \; = \; \frac{\cos^2 x}{\cos^2 x + \sin^2 x} \; = \; \frac{1}{1 + \tg^2 x} \; = \; \frac{1}{1 + t^2}
\]

\[
\sin x \cos x \; = \; \frac{\sin x \cos x}{1} \; = \; \frac{\sin x \cos x}{\cos^2 x + \sin^2 x} \; = \; \frac{\tg x}{1 + \tg^2 x} \; = \; \frac{t}{1 + t^2}
\]

\item

\[
\int R(\sin x, \cos x) \: \textnormal{d}x
\]

\begin{eqnarray*}
t 							& = & \tg \frac{x}{2} \\
\arc \tg t					& = & \frac{x}{2} \\
\frac{2}{1 + t^2} \: \textnormal{d}t		& = & \textnormal{d}x
\end{eqnarray*}

\[
\sin x \; = \; \frac{2 \sin \frac{x}{2} \cos \frac{x}{2}}{1} \; = \; \frac{2 \sin \frac{x}{2} \cos \frac{x}{2}}{\cos^2 \frac{x}{2} + \sin^2 \frac{x}{2}} \; = \; \frac{2 \tg \frac{x}{2}}{1 + \tg^2 \frac{x}{2}} \; = \; \frac{2t}{1 + t^2}
\]

\[
\cos x \; = \; \frac{\cos^2 \frac{x}{2} - \sin^2 \frac{x}{2}}{1} \; = \; \frac{\cos^2 \frac{x}{2} - \sin^2 \frac{x}{2}}{\cos^2 \frac{x}{2} + \sin^2 \frac{x}{2}} \; = \; \frac{1 - \tg^2 \frac{x}{2}}{1 + \tg^2 \frac{x}{2}} \; = \; \frac{1 - t^2}{1 + t^2}
\]
\end{itemize}

\bigskip
%%%%%%%%%%%%%%%%%%%%%%%%%%%%%%%%%%%%%%%%
\subsection{Ca�ki funkcji niewymiernych}

\subsubsection{Ca�ki funkcji zawieraj�cych pierwiastki z~wyra�enia liniowego}

\cite[Paragraf 17.1]{krysicki1}

\medskip

Je�eli funkcja podca�kowa jest funkcj� wymiern� pot�g zmiennej $x$~o~wyk�adnikach postaci $\dfrac{m}{n}$, gdzie $m$, $n$ s� liczbami naturalnymi wzgl�dem siebie pierwszymi, to wykonujemy podstawienie

\[
x \; = \; t^N
\]

\medskip

gdzie $N$ oznacza wsp�lny mianownik u�amk�w postaci $\dfrac{m}{n}$.

\bigskip
%%%%%%%%%%%%%%%%%%%%%%%%%%%%%%%%%%%%%%%%%%%%%%%%%%%%%%%%%%%%%%%%%%%%%%%%%%%%%%%%%%%%%%%%%%%%%%%%%%%%%%%%%%%%%%%%%
\subsubsection{Podstawienia Eulera - Ca�ki funkcji zawieraj�cych pierwiastek kwadratowy z~tr�jmianu kwadratowego}

Ca�k� postaci

\[
\int R\left(x, \sqrt{ax^2 + bx +c}\right) \: \textnormal{d}x
\]

\medskip

mo�emy obliczy� stosuj�c \textbf{podstawienie Eulera}. W~zale�no�ci od warunk�w stosujemy podstawienia:

\begin{enumerate}
\item gdy $a \; > \; 0$

\[
\sqrt{ax^2 + bx +c} \; = \; t \pm \sqrt{a} \: x
\]

\medskip

\item gdy $\Delta = b^2 -4ac \; > \; 0 \qquad \Big(ax^2 + bx +c \; = \; a(x-x_1)(x-x_2)\Big)$

\[
\sqrt{ax^2 + bx +c} \; = \; (x-x_{1})t
\]

\medskip

lub

\[
\sqrt{ax^2 + bx +c} \; = \; (x-x_{2})t
\]
\end{enumerate}

\medskip

R�wnania podnosimy obustronnie do kwadratu. Obliczamy $x$ i~wstawiamy do podstawienia. Otrzymujemy w~ten spos�b podstawienie dla $\sqrt{ax^2 + bx +c}$ zale�ne jedynie od $t$. R�wnanie z~wyliczonym $x$ r�niczkujemy.

\bigskip
%%%%%%%%%%%%%%%%%%%%%%%%%%%%%%%%%%%%%%%%%%%%%%%%%%%%%%%%%%%%%%%%%%%%%%%%%%%
\subsubsection{Ca�kowanie przy pomocy metody wsp�czynnik�w nieoznaczonych}

\cite[Paragraf 17.3, Zadanie 17.46, 17.47]{krysicki1}

\medskip

\textbf{Metod� wsp�czynnik�w nieoznaczonych} stosujemy przy obliczaniu ca�ek postaci

\[
\int \frac{W_n(x)}{\sqrt{ax^2 + bx +c}} \: \textnormal{d}x
\]

\medskip

gdzie $W_n(x)$ jest wielomianem (Def. \ref{def:wielomian}, str. \pageref{def:wielomian}) stopnia $n$ (Def. \ref{def:stopien_wielomianu}, str. \pageref{def:stopien_wielomianu}). Ca�ka ta r�wna si� wyra�eniu

\[
W_{n-1}(x) \: \sqrt{ax^2 + bx +c} \; + \; A \int \frac{1}{\sqrt{ax^2 + bx +c}} \: \textnormal{d}x
\]

\medskip

gdzie $W_{n-1}(x)$ jest wielomianem stopnia $n-1$, a $A$ - pewn� sta��.

\bigskip

\noindent Aby obliczy� \textbf{wsp�czynniki (nieoznaczone) wielomianu} $W_{n-1}$ oraz~$\mathbf{A}$

\begin{enumerate}
\item r�niczkujemy obie strony to�samo�ci


\begin{align*}
\int \frac{W_n(x)}{\sqrt{ax^2 + bx +c}} \: \textnormal{d}x \; \equiv \; W_{n-1}(x) \: \sqrt{ax^2 + bx +c} \; + \\
+ \; A \int \frac{1}{\sqrt{ax^2 + bx +c}} \: \textnormal{d}x
\end{align*}

\medskip

otrzymujemy

\begin{align*}
\frac{W_n(x)}{\sqrt{ax^2 + bx +c}} \; \equiv \; W_{n-1}'(x) \: \sqrt{ax^2 + bx +c} \; + \\
+ \; W_{n-1}(x) \: \frac{2ax + b}{2\sqrt{ax^2 + bx +c}} \; + \\
+ \; A \: \frac{1}{\sqrt{ax^2 + bx +c}}
\end{align*}

\item mno�ymy obie strony to�samo�ci przez $\sqrt{ax^2 + bx +c}$

\[
W_n(x) \; \equiv \; W_{n-1}'(x) \: (ax^2 + bx +c) \; + \; W_{n-1}(x) \: \frac{2ax + b}{2} \; + \; A
\]

\item korzystaj�c z~(Tw. \ref{tw:rownosc_wielomianow}, str. \pageref{tw:rownosc_wielomianow}) przyr�wnujemy kolejno odpowiednie wsp�czynniki.
\end{enumerate}

\medskip

Obliczone wsp�czynniki wstawiamy do pierwotnej to�samo�ci i~obliczamy ca�k�

\[
\int \frac{1}{\sqrt{ax^2 + bx +c}} \: \textnormal{d}x
\]

\medskip

Otrzymany wynik wstawiamy do to�samo�ci. W~ten spos�b otrzymujemy warto�� pierwotnej ca�ki.

\bigskip
%%%%%%%%%%%%%%%%%%%%%%%%%%%%%%%%%%%%%%%
\subsubsection{Inne rodzaje podstawie�}

\begin{itemize}
\item

\[
\int R(x, \sqrt{a^2 - x^2}) \: \textnormal{d}x
\]

\[
x \; = \; a \: \sin t \quad \vee \quad x = a \: \tgh t
\]

\item

\[
\int R(x, \sqrt{a^2 + x^2}) \: \textnormal{d}x
\]

\[
x \; = \; a \: \tg t \quad \vee \quad x = a \: \sinh t
\]

\item

\[
\int R(x, \sqrt{x^2 - a^2}) \: \textnormal{d}x
\]

\[
x \; = \; \frac{a}{\cos t} \quad \vee \quad x = a \: \cosh t
\]

\end{itemize}

\bigskip
%%%%%%%%%%%%%%%%%%%%%%%%%
\section{Ca�ka oznaczona}

Zak�adamy, �e funkcja $f(x)$ jest ograniczona w~przedziale domkni�tym $\langle a,b\rangle$ (Def. \ref{def:funkcja_ograniczona_w_przedziale}, str. \pageref{def:funkcja_ograniczona_w_przedziale})

\vspace{5cm}

TODO rysunek konstrukcji

\bigskip

Dokonajmy nast�puj�cej konstrukcji:

\begin{itemize}
\item dokonujemy $m$ r�nych podzia��w ($P_1$, $P_2$, $\ldots$, $P_m$) przedzia�u $\langle a,b\rangle$ na cz�ci (dzielimy w~r�ny spos�b przedzia� $\langle a,b\rangle$ na mniejsze przedzia�y)

\item przyjmujemy, �e w~wyniku \textbf{podzia�u $\mathbf{P_m}$ otrzymali�my przedzia�y}

\[
\langle x_{i-1} , x_{i} \rangle, \qquad \textsl{gdzie } n = 1, 2, \ldots n_m
\]

gdzie

\begin{itemize}
\item $n_m$ - liczba przedzia��w
\item $x_0 = a$
\item $x_{n_m} = b$
\item liczby $x_i$ tworz� ci�g silnie rosn�cy (Def. \ref{def:ciag_silnie_rosnacy}, str. \pageref{def:ciag_silnie_rosnacy})
\end{itemize}

\medskip

Przedzia�y te nazywamy \textbf{przedzia�ami cz�stkowymi podzia�u} $P_m$.

\item oznaczamy d�ugo�ci przedzia��w

\[
x_i - x_{i-1} \; = \; \Delta x_i
\]

\item oznaczamy \textbf{d�ugo�� najd�u�szego przedzia�u cz�stkowego} podzia�u $P_m$ przez $\mathbf{\delta_m}$

\[
\delta_m \; = \; \sup{\Delta x_i}
\]

\item Ci�g (Def. \ref{def:ciag}, str. \pageref{def:ciag}) podzia��w $\{P_m\}$ nazywamy \textbf{normalnym ci�giem podzia��w}, je�eli

\[
\lim_{n \to \infty} \delta_m = 0
\]

\item tworzymy \textbf{sum� $\mathbf{S_m}$ iloczyn�w}
\begin{itemize}
\item \textbf{warto�ci funkcji} $\mathbf{f(c_i)}$ w~dowolnym punkcie $c_i$ przedzia�u $\langle x_{i-1} , x_{i} \rangle$
\item oraz \textbf{d�ugo�ci} $\mathbf{\Delta x_i}$ tych przedzia��w
\end{itemize}

przy podziale $P_m$

\[
S_m \; = \; \sum_{i=1}^{n_m} \: f(c_i) \: \Delta x_i
\]

\end{itemize}

\begin{definicja}
Je�eli

\begin{itemize}
\item \textbf{ci�g} $\mathbf{\{S_m\}}$ dla $m \to \infty$ jest \textbf{zbie�ny} (Def. \ref{def:ciag_zbiezny}, str. \pageref{def:ciag_zbiezny})
\item \textbf{i}, dodatkowo, jest \textbf{zbie�ny do tej samej granicy} przy \textbf{ka�dym} normalnym normalnym ci�gu podzia��w $\{P_m\}$, niezale�nie od wyboru punkt�w $c_i$,
\end{itemize}

\noindent to funkcj� $f(x)$ nazywamy \textbf{funkcj� ca�kowaln� w~przedziale} $\langle a,b\rangle$, a~granic� ci�gu $S_m$ nazywamy \textbf{ca�k� oznaczon�}\index{ca�ka!oznaczona} funkcji $f(x)$ \textbf{w~granicach od $a$ do $b$} i~oznaczamy symbolem

\[
\int\limits_a^b f(x) \: \textnormal{d}x
\]


\end{definicja}

\cite[Paragraf 19.1]{krysicki1}

\bigskip
%%%%%%%%%%%%%%%%%%%%%%%%%%%%%%%%%%%%%%%%%%%%%%%%%%%%%%%%
\subsection{Interpretacja geometryczna ca�ki oznaczonej}

\vspace{5cm}

TODO rysunek

\bigskip

\begin{twierdzenie}
Je�eli w przedziale $\langle a,b\rangle$ jest

\[
f \geq 0
\]

\medskip

\noindent to \textbf{pole obszaru} ograniczonego

\begin{itemize}
\item �ukiem krzywej $y = f(x)$,
\item odcinkiem osi $Ox$
\item prost� $x=a$
\item prost� $x=b$
\end{itemize}

\noindent r�wna si� ca�ce oznaczonej

\[
\int\limits_a^b f(x) \: \textnormal{d}x
\]

\cite[Paragraf 19.2]{krysicki1}

\end{twierdzenie}

\medskip

\noindent\textbf{Wniosek}

Je�eli w~przedziale $\langle a,b\rangle$ jest

\[
f \leq 0
\]

\medskip

to pole r�wna si� 

\[
- \int\limits_a^b f(x) \: \textnormal{d}x
\]

\bigskip

\noindent\textbf{Wniosek}

Zawsze pole okre�lonego wy�ej obszaru mo�na wyrazi� ca�k� oznaczon�\label{wn:pole_obszaru_ogolnie}

\[
\int\limits_a^b |f(x)| \: \textnormal{d}x
\]

\cite[Paragraf 19.2]{krysicki1}

\bigskip
%%%%%%%%%%%%%%%%%%%%%%%%%%%%%%%%%%%%%%%
\subsection{W�asno�ci ca�ki oznaczonej}

\cite[Paragraf 19.3]{krysicki1}

\bigskip
%%%%%%%%%%%%%%%%%%%%%%%%%%%%%%%%%%%%%%%%%%%%%%%%%%%%%%%%%%%%%%%%
\subsubsection{Addytywno�� ca�ki wzgl�dem przedzia�u ca�kowania}

\begin{twierdzenie}
Je�eli $a \leq b \leq c$, to (TODO niekonieczne za�o�enie)

\[
\int\limits_a^c f(x) \: \textnormal{d}x \quad = \quad \int\limits_a^b f(x) \: \textnormal{d}x + \int\limits_b^c f(x) \: \textnormal{d}x
\]
\end{twierdzenie}

\bigskip
%%%%%%%%%%%%%%%%%%%%%%%%%%%%%%%%%%%%%%%%%%%%%%%%%%%%%%%%%%%%%%
\subsubsection{Addytywno�� ca�ki wzgl�dem funkcji podca�kowej}

\begin{twierdzenie}
Ca�ka sumy r�wna si� sumie ca�ek, tzn.

\[
\int\limits_a^b \Big(f(x)+g(x)\Big) \: \textnormal{d}x \quad = \quad \int\limits_a^b f(x) \: \textnormal{d}x + \int\limits_a^b g(x) \: \textnormal{d}x
\]
\end{twierdzenie}

\bigskip
%%%%%%%%%%%%%%%%%%%%%%%%%%%%%%%%%%%%%%%%%%%%%%%%%%%%%%%%%%%%%%%%%%%%%%%
\subsubsection{Wynoszenie sta�ego czynnika przed znak ca�ki oznaczonej}

\begin{twierdzenie}
Sta�y czynnik mo�na wynie�� przed znak ca�ki

\[
\int\limits_a^b A \: f(x) \: \textnormal{d}x \quad = \quad A \: \int\limits_a^b f(x) \: \textnormal{d}x
\]
\end{twierdzenie}

\bigskip
%%%%%%%%%%%%%%%%%%%%%%%%%%%%%%%%%%%%%%%%%%%%%%%%%%%%%%%%%%%%%%
\subsubsection{Parzysto�� i~nieparzysto�� funkcji podca�kowej}

\begin{twierdzenie}
Je�eli $f(x) \in C^0 \Big(\langle -a,a\rangle\Big)$ i~jest nieparzysta (Def. \ref{def:funkcja_nieparzysta}, str. \pageref{def:funkcja_nieparzysta}), to

\[
\int\limits_{-a}^a f(x) \: \textnormal{d}x \quad = \quad 0
\]
\end{twierdzenie}

\medskip

\begin{twierdzenie}
Je�eli $f(x) \in C^0 \Big(\langle -a,a\rangle\Big)$ i~jest parzysta (Def. \ref{def:funkcja_parzysta}, str. \pageref{def:funkcja_parzysta}), to

\[
\int\limits_{-a}^a f(x) \: \textnormal{d}x \quad = \quad 2 \: \int\limits_0^a f(x) \: \textnormal{d}x \;
\]
\end{twierdzenie}

\bigskip
%%%%%%%%%%%%%%%%%%%%%%%%%%%%%%%%%%%%%%%%%%%%%%%%%%%%%%%%%%%%%%%%%%
\subsubsection{Twierdzenie o warto�ci �redniej}

\begin{twierdzenie}
Je�eli $f(x) \in C^0 \Big(\langle a,b\rangle\Big)$, to istnieje $c \in (a,b)$ takie, �e

\[
\int\limits_a^b f(x) \: \textnormal{d}x \quad = \quad f(c)(b-a)
\]
\end{twierdzenie}

\bigskip
%%%%%%%%%%%%%%%%%%%%%%%%%%%%%%%%%%%%%%%%%%%%%%%%%%%%%%%%%%%%%
\subsubsection{Zwi�zek mi�dzy ca�k� oznaczon� a~nieoznaczon�}

\begin{twierdzenie}
Je�eli przez $F(x)$ oznaczymy funkcj� pierwotn� (Def. \ref{def:funkcja_pierwotna}, str. \pageref{def:funkcja_pierwotna}) funkcji $f(x) \in C^0 \Big(\langle a,b\rangle\Big)$, to

\[
\int\limits_a^b f(x) \: \textnormal{d}x \quad = \quad F(b) - F(a)
\]
\end{twierdzenie}

\medskip

\noindent\textbf{Oznaczenia}

$F(b) - F(a)$ oznacza� mo�na przez

\begin{itemize}
\item $\big[F(x)\big]_a^b$
\item $F(x)|_a^b$
\item $F(x)|_{x=a}^{x=b}$
\end{itemize}

\bigskip
%%%%%%%%%%%%%%%%%%%%%%%%%%%%%%%%%%%%%%%%%%%%%%%%%%%%%%%%%
\subsubsection{Ca�kowanie przez cz�ci ca�ek oznaczonych}

\begin{twierdzenie}
Je�eli $f(x), g(x) \in C^1 \Big(\langle a,b\rangle\Big)$, to

\[
\int\limits_a^b f'(x) \: g(x) \: \textnormal{d}x \quad = \quad \big[f(x) \: g(x)\big]_a^b - \int\limits_a^b f(x) \: g'(x) \: \textnormal{d}x
\]
\end{twierdzenie}

\bigskip
%%%%%%%%%%%%%%%%%%%%%%%%%%%%%%%%%%%%%%%%%%%%%%%%%%%%%%%%%%%%%%%
\subsubsection{Ca�kowanie przez podstawienie ca�ek oznaczonych}

\begin{twierdzenie}\label{tw:o_calkowaniu_przez_podstawienie_calek_oznaczonych}
Je�eli 
\begin{itemize}
\item $g'(x)$ jest funkcj� ci�g�� (Def. \ref{def:funkcja_ciagla}, str. \pageref{def:funkcja_ciagla}),
\item $g(x)$ jest funkcj� silnie rosn�c� (Def. \ref{def:funkcja_silnie_rosnaca}, str. \pageref{def:funkcja_silnie_rosnaca}) w~przedziale $\langle a,b\rangle$
\item $f(t)$ jest ci�g�a w~przedziale $\Big\langle g(a), g(b)\Big\rangle$ (Def. \ref{def:funkcja_ciagla_na_zbiorze}, str. \pageref{def:funkcja_ciagla_na_zbiorze})
\end{itemize}

\medskip

to

\[
\int\limits_a^b f\Big(g(x)\Big) \: g'(x) \: \textnormal{d}x \quad = \quad \int\limits_{g(a)}^{g(b)} f(t) \: \textnormal{d}t
\]

\medskip

Podstawienie

\[
t \; = \; g(x)
\]

\end{twierdzenie}


\bigskip
%%%%%%%%%%%%%%%%%%%%%%%%%%%%%%%%%%%%%%%%%%
\subsection{Zastosowanie ca�ki oznaczonej}

\bigskip
%%%%%%%%%%%%%%%%%%%%%%%%%%%%%%%%%%%%%%%%%%%%%%%%%
\subsubsection{Obliczanie pola obszaru p�askiego}

TODO

\bigskip
%%%%%%%%%%%%%%%%%%%%%%%%%%%%%%%%%%%%%%%%%%%%%%%%%%%%%%%%%%%%%%%%%%%%%%%%%%%%%%%%%%%%%%%%%%%%%%%%%%%%%%%%%%%%%%%%%%
\subsubsection{Obliczanie pola obszaru p�askiego, gdy linia ograniczaj�ca okre�lona jest w~postaci parametrycznej}

\cite[Paragraf 20.1]{krysicki1}

\medskip

Je�eli linia ograniczaj�ca okre�lona jest w~\textbf{postaci parametrycznej}

\[
\begin{array}{rcl}
x & = & g(t)\\[4pt]
y & = & h(t)
\end{array}
\]

\medskip

\noindent gdzie w~przedziale $\langle t_1, t_2\rangle$

\begin{itemize}
\item funkcje $g(t)$ i~$h(t)$ s� \textbf{ci�g�e} (Def. \ref{def:funkcja_ciagla_na_zbiorze}, str. \pageref{def:funkcja_ciagla_na_zbiorze}) 
\item funkcja $g(t)$ jest \textbf{rosn�ca} (Def. \ref{def:funkcja_rosnaca}, str. \pageref{def:funkcja_rosnaca})
\item funkcja $g(t)$ ma \textbf{pochodn�} (Def. \ref{def:pochodna_funkcji}, str. \pageref{def:pochodna_funkcji})  \textbf{ci�g��} 
\end{itemize}

\bigskip

\noindent to, korzystaj�c z~(Tw. \ref{tw:o_calkowaniu_przez_podstawienie_calek_oznaczonych}, str. \pageref{tw:o_calkowaniu_przez_podstawienie_calek_oznaczonych}) oraz wniosk�w z~interpretacji geometrycznej ca�ki oznaczonej (Wn. \ref{wn:pole_obszaru_ogolnie}, str. \pageref{wn:pole_obszaru_ogolnie}), \textbf{pole obszaru} ograniczonego:

\begin{itemize}
\item dan� lini� ograniczaj�c� ($x \; = \; g(t)$, $y \; = \; h(t)$)
\item odcinkiem osi $Ox$
\item prost� $x \; = \; x_1$ ($x_1 \; = \; g(t_1)$)
\item prost� $x \; = \; x_2$ ($x_2 \; = \; g(t_2)$)
\end{itemize}

\medskip

\noindent wyra�a si� wzorem

\[
P \quad = \quad \int\limits_{x_1}^{x_2} |y| \: \textnormal{d}x \quad = \quad \int\limits_{t_1}^{t_2} |h(t)| \: g'(t) \: \textnormal{d}t
\]

\bigskip

\noindent W~przypadku, gdy

\begin{itemize}
\item za�o�enia s� \textbf{takie same} jak powy�ej
\item a funkcja $g(t)$ jest \textbf{malej�ca}
\end{itemize}

\medskip

\noindent to \textbf{pole obszaru} wyra�a si� 

\[
P \quad = \quad \int\limits_{x_1}^{x_2} |y| \: \textnormal{d}x \quad = \quad -\int\limits_{t_1}^{t_2} |h(t)| \: g'(t) \: \textnormal{d}t
\]

\bigskip
%%%%%%%%%%%%%%%%%%%%%%%%%%%%%%%%%%%%%%%%%%%%%%%%%%%%%%%%%%%%%%%%%%%%%%%%%%%%%%%%%%%%%%%%%%%%%%%%%%%%%%%%%%%%%%%%%%
\subsubsection{Obliczanie pola obszaru p�askiego, gdy linia ograniczaj�ca okre�lona jest we wsp�rz�dnych biegunowych}

\cite[Paragraf 20.1]{krysicki1}

\medskip

Je�eli linia ograniczaj�ca dana jest we wsp�rz�dnych biegunowych (Def. \ref{def:uklad_wspolrzednych_biegunowych}, str. \pageref{def:uklad_wspolrzednych_biegunowych})

\[
r \; = \; f(\theta)
\]

\medskip

\noindent gdzie 

\begin{itemize}
\item $f(\theta)$ jest funkcj� \textbf{nieujemn�} (Def. \ref{def:funkcja_nieujemna}, str. \pageref{def:funkcja_nieujemna}) \textbf{ci�g��} (Def. \ref{def:funkcja_ciagla_na_zbiorze}, str. \pageref{def:funkcja_ciagla_na_zbiorze}) w~przedziale $\langle \alpha, \beta\rangle$
\item a przy tym $\beta - \alpha \: \in \: (0, 2\pi)$
\end{itemize}

\medskip

\noindent to \textbf{pole obszaru} ograniczonego

\begin{itemize}
\item lini� ograniczaj�c� $r \; = \; f(\theta)$
\item promieniem wodz�cym (Def. \ref{def:promien_wodzacy}, str. \pageref{def:promien_wodzacy}) o~amplitudzie (Def. \ref{def:amplituda_promiena_wodzacego}, str. \pageref{def:amplituda_promiena_wodzacego}) $\alpha$
\item promieniem wodz�cym o~amplitudzie $\beta$
\end{itemize}

\medskip

\noindent wyra�a si� wzorem (TODO sk�d $r^2$? - wyprowadzenie)

\[
P \quad = \quad \frac{1}{2} \: \int\limits_{\alpha}^{\beta} r^2 \: d\theta \quad = \quad \frac{1}{2} \: \int\limits_{\alpha}^{\beta} \Big(f(\theta)\Big)^2 \: d\theta
\]

\bigskip
%%%%%%%%%%%%%%%%%%%%%%%%%%%%%%%%%%%%%%%%
\subsubsection{Obliczanie d�ugo�ci �uku}

\cite[Paragraf 20.2]{krysicki1}

\medskip

Je�eli krzywa dana jest:

\begin{itemize}
\item r�wnaniem postaci

\[
y \; = \; f(x)
\]

przy czym 
\begin{itemize}
\item funkcja $f(x)$ ma w~przedziale $\langle a,b\rangle$ \textbf{pochodn�} (Def. \ref{def:pochodna_funkcji}, str. \pageref{def:pochodna_funkcji}) \textbf{ci�g��} (Def. \ref{def:funkcja_ciagla_na_zbiorze}, str. \pageref{def:funkcja_ciagla_na_zbiorze}),
\end{itemize}

to \textbf{d�ugo�� �uku} wyra�a si� wzorem

\[
L \quad = \quad \int\limits_{a}^{b} \sqrt{1 + \left(\frac{\textnormal{d}y}{\textnormal{d}x}\right)^2} \: \textnormal{d}x
\]

\medskip

a \textbf{r�niczka �uku} ma posta�

\[
dL \quad = \quad \sqrt{1 + \left(\frac{\textnormal{d}y}{\textnormal{d}x}\right)^2} \: \textnormal{d}x
\]

\bigskip

\item parametrycznie 

\[
\begin{array}{rcl}
x & = & g(t)\\[4pt]
y & = & h(t)
\end{array}
\]

przy czym

\begin{itemize}
\item funkcje $g(t)$ i~$h(t)$ maj� w~przedziale $\langle t_1,t_2\rangle$ \textbf{pochodne ci�g�e}
\item �uk \textbf{nie ma} cz�ci wielokrotnych
\end{itemize}

to \textbf{d�ugo�� �uku} wyra�a si� wzorem 

\[
L \quad = \quad \int\limits_{t_1}^{t_2} \sqrt{\left(\frac{\textnormal{d}x}{\textnormal{d}t}\right)^2 + \left(\frac{\textnormal{d}y}{\textnormal{d}t}\right)^2} \: \textnormal{d}t
\]

\medskip

a \textbf{r�niczka �uku} ma posta�

\[
dL \quad = \quad \sqrt{\left(\frac{\textnormal{d}x}{\textnormal{d}t}\right)^2 + \left(\frac{\textnormal{d}y}{\textnormal{d}t}\right)^2} \: \textnormal{d}t
\]

\bigskip

\item r�wnaniem we wsp�rz�dnych biegunowych (Def. \ref{def:uklad_wspolrzednych_biegunowych}, str. \pageref{def:uklad_wspolrzednych_biegunowych})

\[
r \; = \; f(\theta)
\]

przy czym

\begin{itemize}
\item funkcje $f(\theta)$ ma w~przedziale $\langle \alpha,\beta\rangle$ \textbf{pochodn� ci�g��}
\item �uk \textbf{nie ma} cz�ci wielokrotnych
\end{itemize}

to \textbf{d�ugo�� �uku} wyra�a si� wzorem 

\[
L \quad = \quad \int\limits_{\alpha}^{\beta} \sqrt{r^2 + \left(\frac{dr}{d\theta}\right)^2} \: d\theta
\]

\medskip

a \textbf{r�niczka �uku} ma posta�

\[
dL \quad = \quad \sqrt{r^2 + \left(\frac{dr}{d\theta}\right)^2} \: d\theta
\]

\end{itemize}

TODO wyprowadzenia

\bigskip
%%%%%%%%%%%%%%%%%%%%%%%%%%%%%%%%%%%%%%%%%%%%%%%%%%%%
\subsubsection{Obliczanie obj�to�ci i~pola powierzchni bry�y obrotowej}

\cite[Paragraf 20.3]{krysicki1} \cite[Rozdzia� 3.10]{zakowski1}

\medskip

Dla \textbf{bry�y obrotowej} (Def. \ref{def:bryla_obrotowa}, str. \pageref{def:bryla_obrotowa}) powsta�ej przez obr�t wok� osi~$Ox$ krzywej danej

\begin{itemize}
\item r�wnaniem
\[
y \; = \; f(x)
\]

\noindent gdzie $f(x)$ jest w~przedziale $\langle a,b\rangle$

\begin{itemize}
\item ci�g�a
\item nieujemna
\end{itemize}

\noindent \textbf{Obj�to��} wyra�a si� wzorem

\[
V \quad = \quad \pi \: \int\limits_{a}^{b} y^2 \: \textnormal{d}x
\]

\medskip

\noindent a \textbf{pole powierzchni} - wzorem

\[
S \quad = \quad 2\pi \: \int\limits_{a}^{b} y \: \sqrt{1 + \left(\frac{\textnormal{d}y}{\textnormal{d}x}\right)^2} \: \textnormal{d}x
\]

\bigskip

\item w postaci parametrycznej

\[
\begin{array}{rcl}
x & = & g(t)\\[4pt]
y & = & h(t)
\end{array}
\]

\noindent przy czym w~przedziale $\langle t_1, t_2\rangle$

\begin{itemize}
\item funkcje $g(t)$ i~$h(t)$ maj� \textbf{pochodne} (Def. \ref{def:pochodna_funkcji}, str. \pageref{def:pochodna_funkcji}) \textbf{ci�g�e} (Def. \ref{def:funkcja_ciagla_na_zbiorze}, str. \pageref{def:funkcja_ciagla_na_zbiorze}) 
\item funkcja $g(t)$ jest \textbf{silnie rosn�ca} (Def. \ref{def:funkcja_silnie_rosnaca}, str. \pageref{def:funkcja_silnie_rosnaca}) lub \textbf{silnie malej�ca} (Def. \ref{def:funkcja_silnie_malejaca}, str. \pageref{def:funkcja_silnie_malejaca})
\item funkcja $h(t)$ jest \textbf{nieujemna} (Def. \ref{def:funkcja_nieujemna}, str. \pageref{def:funkcja_nieujemna})
\end{itemize}

\medskip

\noindent \textbf{Obj�to��} wyra�a si� wzorem

\[
V \quad = \quad \pi \: \int\limits_{t_1}^{t_2} y^2 \: \frac{\textnormal{d}x}{\textnormal{d}t}\: \textnormal{d}x
\]

\medskip

\noindent a \textbf{pole powierzchni} - wzorem

\[
S \quad = \quad 2\pi \: \int\limits_{t_1}^{t_2} y \: \sqrt{\left(\frac{\textnormal{d}x}{\textnormal{d}t}\right)^2 + \left(\frac{\textnormal{d}y}{\textnormal{d}t}\right)^2} \: \textnormal{d}t
\]
\end{itemize}

TODO wyprowadzenie


\bigskip
%%%%%%%%%%%%%%%%%%%%%%%%%%%
\section{Ca�ka niew�a�ciwa}

\begin{definicja}
Je�eli funkcja $f(x)$ okre�lona na przedziale $\langle a,b)$, gdzie

\begin{itemize}
\item $b \in \mathbb{R}$
\item lub $b = +\infty$
\end{itemize}

\medskip

\noindent jest w~tym przedziale \textbf{ci�g�a} (Def. \ref{def:funkcja_ciagla_na_zbiorze}, str. \pageref{def:funkcja_ciagla_na_zbiorze}) oraz \textbf{istnieje granica} (Def. \ref{def:granica_funkcji}, str. \pageref{def:granica_funkcji})

\[
\lim_{\alpha \to b^-} \int\limits_{a}^{\alpha} f(x) \: \textnormal{d}x
\]

\medskip

\noindent i~jest to liczba rzeczywista, to granic� t� nazywamy \textbf{ca�k� niew�a�ciw� funkcji}\label{def:calka_niewlasciwa} $f(x)$ i~oznaczamy

\[
\int\limits_{a}^{b} f(x) \: \textnormal{d}x
\]

\medskip

\noindent Zatem

\[
\int\limits_{a}^{b} f(x) \: \textnormal{d}x \quad = \quad \lim_{\alpha \to b^-} \int\limits_{a}^{\alpha} f(x) \: \textnormal{d}x
\]

\cite[Definicja 8.27]{ptak} \cite[Paragraf 21.1]{krysicki1}
\end{definicja}

\bigskip

\begin{definicja}
Je�eli funkcja $f(x)$ okre�lona na przedziale $(a,b\rangle$, gdzie

\begin{itemize}
\item $a \in \mathbb{R}$
\item lub $a = -\infty$
\end{itemize}

\medskip

\noindent to otrzymujemy w~analogiczny spos�b do (Def. \ref{def:calka_niewlasciwa}, str. \pageref{def:calka_niewlasciwa})

\[
\int\limits_{a}^{b} f(x) \: \textnormal{d}x \quad = \quad \lim_{\alpha \to a^+} \int\limits_{\alpha}^{b} f(x) \: \textnormal{d}x
\]

\cite[Definicja 8.27]{ptak} \cite[Paragraf 21.1]{krysicki1}
\end{definicja}

\bigskip
%%%%%%%%%%%%%%%%%%%%%%%%%%%%%%%%%%%%%%%%%
\section{Metody ca�kowania przybli�onego}

\cite[Rozdzia� 3.11]{zakowski1} \cite[Rozdzia� XXII]{krysicki1}

\medskip

W tym podrozdziale przedstawione s� konstrukcje metod obliczania \textbf{przybli�onej warto�ci ca�ki funkcji} ci�g�ej (Def. \ref{def:funkcja_ciagla_na_zbiorze}, str. \pageref{def:funkcja_ciagla_na_zbiorze}) $f(x)$ w~przedziale $\langle a,b\rangle$

\[
\int\limits_{a}^{b} f(x) \: \textnormal{d}x
\]

\bigskip
%%%%%%%%%%%%%%%%%%%%%%%%%%%
\subsection{Metoda prostok�t�w}

\begin{itemize}
\item \textbf{dzielimy przedzia�} $\langle a,b\rangle$ punktami

\[
x_k \; = \; a + \lambda \: k \qquad k = 1, 2, \ldots, n-1
\]

na $n$ podprzedzia��w o~jednakowej d�ugo�ci $\lambda$

\[
\lambda \; = \; \frac{b - a}{n}
\]

\item pocz�tek $a$ i~koniec $b$ przedzia�u oznaczamy

\[
\begin{array}{rcl}
a & = & x_0\\[4pt]
b & = & x_n
\end{array}
\]

\item w~ka�dym z~podprzedzia��w $\langle x_{k-1}, x_k\rangle$ \textbf{wybieramy �rodek} przedzia�u $\xi_k$

\[
\xi_k \; = \; \frac{x_{k-1} + x_k}{2}
\]

\item dla ka�dego �rodka $\xi_k$ podprzedzia�u $\langle x_{k-1}, x_k\rangle$ \textbf{obliczamy warto�� funkcji} $f(x)$ w~tym punkcie

\[
y_k \; = \; f(\xi_k)
\]

\item \textbf{warto�� przybli�on� ca�ki} obliczamy przez \textbf{sumowanie p�l prostok�t�w} o~bokach $\lambda$ i~$y_k$

\[
\int\limits_{a}^{b} f(x) \: \textnormal{d}x \quad \approx \quad \lambda \: \sum_{k=1}^{n} y_k
\]
\end{itemize}

\bigskip
%%%%%%%%%%%%%%%%%%%%%%%%%%%
\subsection{Metoda trapez�w}

\begin{itemize}
\item \textbf{dzielimy przedzia�} $\langle a,b\rangle$ punktami

\[
x_k \; = \; a + \lambda \: k \qquad k = 1, 2, \ldots, n-1
\]

na $n$ podprzedzia��w o~jednakowej d�ugo�ci $\lambda$

\[
\lambda \; = \; \frac{b - a}{n}
\]

\item pocz�tek $a$ i~koniec $b$ przedzia�u oznaczamy

\[
\begin{array}{rcl}
a & = & x_0\\[4pt]
b & = & x_n
\end{array}
\]

\item dla ka�dego punktu $x_k$ \textbf{obliczamy warto�� funkcji} $f(x)$ w~tym punkcie

\[
y_k \; = \; f(x_k) \qquad k = 0, 1, \ldots, n
\]

\item \textbf{warto�� przybli�on� ca�ki} obliczamy przez \textbf{sumowanie p�l trapez�w prostok�tnych} o~podstawach $y_{k-1}$ i~$y_k$ oraz wysoko�ci $\lambda$

\[
\int\limits_{a}^{b} f(x) \: \textnormal{d}x \quad \approx \quad \frac{\lambda}{2} \: \sum_{k=1}^{n} \Big(y_{k-1} +y_k\Big)
\]
\end{itemize}

\bigskip
%%%%%%%%%%%%%%%%%%%%%%%%%%%
\subsection{Metoda Simpsona}

\begin{itemize}
\item \textbf{dzielimy przedzia�} $\langle a,b\rangle$ punktami

\[
x_k \; = \; a + \lambda \: k \qquad k = 1, 2, \ldots, 2n-1
\]

na parzyst� $2n$ podprzedzia��w o~jednakowej d�ugo�ci $\lambda$

\[
\lambda \; = \; \frac{b - a}{2n}
\]

\item pocz�tek $a$ i~koniec $b$ przedzia�u oznaczamy

\[
\begin{array}{rcl}
a & = & x_0\\[4pt]
b & = & x_{2n}
\end{array}
\]

\item w ka�dym punkcie $x_k$ ($k = 0,1,2,\ldots, 2n$) \textbf{obliczamy warto�� funkcji} podca�kowej $f(x)$

\[
y_k \; = \; f(x_k)
\]

\item dla ka�dego podprzedzia�u $\langle x_{2i-2}, x_{2i}\rangle$ ($i = 1,2, \ldots, n$) \textbf{ca�k�}

\[
\int\limits_{x_{2i-2}}^{x_{2i}} f(x) \: \textnormal{d}x \qquad i = 1,2, \ldots, n
\]

\textbf{zast�pujemy} ca�k� z~funkcji

\[
h_i(x) \; = \; a_i \: x^2 + b_i \: x + c_i
\]

tak dobranej, aby spe�nione by�y warunki

\[
\begin{array}{rcl}
h_i(x_{2i-2}) & = & y_{2i-2}\\[4pt]
h_i(x_{2i-1}) & = & y_{2i-1}\\[4pt]
h_i(x_{2i}) & = & y_{2i}
\end{array}
\]

czyli mamy

\[
\int\limits_{x_{2i-2}}^{x_{2i}} f(x) \: \textnormal{d}x \quad = \quad \int\limits_{x_{2i-2}}^{x_{2i}} h_i(x) \: \textnormal{d}x
\]

\item \textbf{warto�� przybli�on� ca�ki} obliczamy przez \textbf{sumowanie warto�ci ca�ek} okre�lonych powy�ej

\[
\int\limits_{a}^{b} f(x) \: \textnormal{d}x \quad \approx \quad \sum_{i=1}^{n} \: \int\limits_{x_{2i-2}}^{x_{2i}} h_i(x) \: \textnormal{d}x
\]
\end{itemize}
\chapter{Funkcje wielu zmiennych}

\section{Granica funkcji}

\subsection{Granice iterowane}

\section{Ci�g�o��}
\chapter{Rachunek r�niczkowy funkcji wielu zmiennych}

\section{Pochodna funkcji}

\subsection{Pochodna cz�stkowa}

\subsection{Pochodna kierunkowa}

\subsection{Pochodne cz�stkowe wy�szych rz�d�w}

\section{R�niczka funkcji}

\subsection{Twierdzenie o przedstawieniu przyrostu funkcji}

\subsection{Funkcja r�niczkowalna}

\subsection{R�niczka funkcji}

\subsection{R�niczka zupe�na}

\cite[Paragraf 11.5, Paragraf 11.6]{krysicki2}

\medskip

Niech
\begin{itemize}
\item funkcje $P(x,y)$ i~$Q(x,y)$ s� klasy $C^0$ w~obszarze $D$
\item funkcja $F(x,y)$ jest klasy $C^1$ w~obszarze $D$
\end{itemize}

\medskip

\begin{definicja}
Wyra�enie

\[
P(x,y) \: \textnormal{d}x + Q(x,y) \: \textnormal{d}y
\]

\medskip

\noindent jest \textbf{\emph{r�niczk� zupe�n�}}\label{def:rozniczka_zupelna} funkcji $F(x,y)$ je�eli zachodz� zwi�zki

\[
\frac{\partial F}{\partial x} = P(x,y), \qquad \frac{\partial F}{\partial y} = Q(x,y)
\]

\medskip

\noindent w~ka�dym punkcie obszaru $D$.
\end{definicja}

\bigskip
%%%%%%%%%%%%%%%%%%%%%%%%%%%%%%%%%%%%%%%%%%%%%%%%%
\subsubsection{Warunek konieczny i~wystarczaj�cy}

Niech funkcje $P(x,y)$ i~$Q(x,y)$ s� klasy $C^1$ w~obszarze jednosp�jnym $D$ (Def. \ref{def:obszar_jednospojny}, str. \pageref{def:obszar_jednospojny}).

\medskip

\begin{twierdzenie}
Warunkiem koniecznym i~wystarczaj�cym na to, aby wyra�enie 

\[
P(x,y) \: \textnormal{d}x + Q(x,y) \: \textnormal{d}y
\]

\medskip

\noindent by�o \textbf{\emph{r�niczk� zupe�n�}} w~tym obszarze, jest spe�nienie r�wno�ci

\[
\frac{\partial P}{\partial y} = \frac{\partial Q}{\partial x}
\]

\medskip

\noindent w~ka�dym punkcie obszaru $D$.
\end{twierdzenie}

\chapter{Badanie przebiegu zmienno�ci funkcji wielu zmiennych}

\section{Ekstrema funkcji}

\subsection{Warunek konieczny istnienia ekstremum}

\subsection{Warunek wystarczaj�cy istnienia ekstremum}

Jacobian?
\chapter{Rachunek ca�kowy funkcji wielu zmiennych}
\chapter{Elementy teorii pola}

\section{Operator nabla, hamiltona}

\section{Operator gradientu}

\section{Operator dywergencji}

\section{Operator rotacji}

\section{Laplasjan}
\chapter{Funkcja zespolona}

\section{Warunek Cauchy-Riemann'a}

\section{Warunek wystarczaj�cy istnienia pochodnej}
\chapter{R�wnania r�niczkowe zwyczajne}

%%%%%%%%%%%%%%%%%%%%%%%%%%%%%%%%%%%%%%%%%%%
\subsection{R�wnanie r�niczkowe zwyczajne}

\begin{definicja}
\textbf{\emph{R�wnaniem r�niczkowym zwyczajnym}}\label{def:rownanie_rozniczkowe_zupelne} nazywamy r�wnanie zawieraj�ce

\begin{itemize}
\item zmienn� niezale�n� $x$,
\item nieznan� funkcj� $y$ (Def. \ref{def:funkcja}, str. \pageref{def:funkcja}),
\item oraz jej pochodne $y', y'', \ldots, y^{(n)}$ (Def. \ref{def:pochodna_funkcji}, str. \pageref{def:pochodna_funkcji}).
\end{itemize}

\[
F\left(x, \: y', \: y'', \: \ldots, \: y^{(n)}\right) \; = \; 0
\]

\smallskip

\cite[Definicja 1.1]{niedoba}
\end{definicja}

\bigskip
%%%%%%%%%%%%%%%%%%%%%%%%%%%%%%%%%%%%%%%%
\subsection{Rz�d r�wnania r�niczkowego}

\begin{definicja}
\textbf{\emph{Rz�dem r�wnania r�niczkowego}} zwyczajnego nazywamy liczb� r�wn� \textbf{rz�dowi najwy�szej pochodnej} (Def. \ref{def:pochodna_n-tego_rzedu}, str. \pageref{def:pochodna_n-tego_rzedu}) wyst�puj�cej w~r�wnaniu.

\smallskip

\cite[Definicja 1.2]{niedoba}
\end{definicja}

\bigskip
%%%%%%%%%%%%%%%%%%%%%%%%%%%%%%%%%%%%%%%%%%
\subsection{Problem pocz�tkowy Cauchy'ego}

\begin{definicja}
\textbf{\emph{Problemem pocz�tkowym Cauchy'ego}} dla r�wnania r�niczkowego zwyczajnego nazywamy zagadnienie:

\medskip

Znale�� rozwi�zanie r�wnania r�niczkowego zwyczajnego

\[
F\left(x, \: y', \: y'', \: \ldots, \: y^{(n)}\right) \; = \; 0
\]

\medskip

\noindent spe�niaj�ce warunek pocz�tkowy

\[
\left\{
\begin{array}{rcl}
y(x_0) & = & y_0\\
y'(x_0) & = & y_1\\
\vdots & & \\
y^{(n-1)}(x_0) & = & y_{n-1}
\end{array}
\right.
\]

\medskip

\noindent gdzie

\begin{itemize}
\item $x_0 \in (a,b)$
\item $y_0, y_1, \ldots, y_{n_1}$ s� zadanymi liczbami.
\end{itemize}

\smallskip

\cite[Definicja 1.4]{niedoba}
\end{definicja}

\bigskip
%%%%%%%%%%%%%%%%%%%%%%%%%%%%%%%%%%%%%%%%%%%%%%%%%%%%%%
\subsection{Ca�ka szczeg�lna (rozwi�zanie szczeg�lne)}

\begin{definicja}
\textbf{\emph{Ca�k� szczeg�ln�}}\label{def:calka_szczegolna} (\textbf{\emph{rozwi�zaniem szczeg�lnym}}) r�wnania r�niczkowego zwyczajnego nazywamy r�wnanie spe�niaj�ce warunek pocz�tkowy problemu pocz�tkowego Cauchy'ego.

\smallskip

\cite[Rozdzia� 1.1]{zakowski4}
\end{definicja}

\subsection{Ca�ka og�lna (rozwi�zanie og�lne)}

\begin{definicja}
\textbf{\emph{Ca�k� og�ln�}}\label{def:calka_ogolna} (\textbf{\emph{rozwi�zaniem og�lnym}}) r�wnania r�niczkowego zwyczajnego nazywamy zbi�r wszystkich ca�ek szczeg�lnych r�wnania.

\smallskip

\cite[Definicja 1.7]{niedoba}
\end{definicja}

\section{R�wnania r�niczkowe zwyczajne rz�du I-go}

\subsection{R�wnania o zmiennych rozdzielonych}

\begin{definicja}
R�wnanie postaci

\[
X(x) \: \textnormal{d}x \; + \; Y(y) \: \textnormal{d}y \; = \; 0
\]

\medskip 

\noindent nazywamy \textbf{\emph{r�wnaniem o~zmiennych rozdzielonych}}.

\smallskip

\cite[Rozdzia� 1.3.1]{niedoba}
\end{definicja}

\medskip

Ca�k� og�ln� (Def. \ref{def:calka_ogolna}, str. \pageref{def:calka_ogolna}) tego r�wnania jest 

\[
\int X(x) \: \textnormal{d}x \; + \; \int Y(y) \: \textnormal{d}y \; = \; 0
\]

\noindent lub

\[
\int\limits_{x_0}^x X(x) \: \textnormal{d}x \; + \; \int\limits_{y_0}^y Y(y) \: \textnormal{d}y \; = \; C
\]

\bigskip
%%%%%%%%%%%%%%%%%%%%%%%%%%%%%%%%%%%%%%%%%%%%%%%%%%%%%%%%%%%%%%%%%%%%%%%%%%%
\subsubsection{R�wnania sprowadzalne do r�wnania o~zmiennych rozdzielonych}

Niech $f \colon \mathbb{R} \to \mathbb{R}$ b�dzie ci�g�a (Def. \ref{def:funkcja_ciagla}, str. \pageref{def:funkcja_ciagla}).

\medskip

\begin{itemize}
\item W r�wnaniu postaci

\[
\frac{\textnormal{d}y}{\textnormal{d}x} \; = \; f\left(\frac{y}{x}\right)
\]

\medskip

\noindent wprowadzamy now� zmienn� zale�n�

\[
u \; = \; \frac{y}{x}
\]

\medskip

\noindent sk�d

\[
y' \; = \; u + x u'
\]

\medskip

Po wstawieniu do r�wnania i~rozdzieleniu zmiennych mamy

\[
\frac{\textnormal{d}u}{f(u) - u} \; = \; \frac{\textnormal{d}x}{x} \quad \vee \quad f(u) \; = \; u \quad \vee \quad x \; = \; 0
\]

\item W r�wnaniu

\[
\frac{\textnormal{d}y}{\textnormal{d}x} \; = \; f(ax + by + c)
\]

\medskip

\noindent wprowadzamy now� zmienn� zale�n�

\[
u \; = \; ax + by + c
\]

\medskip

\noindent i~dalej post�pujemy jak w pierwszym przypadku.

\item W r�wnaniu

\[
\frac{\textnormal{d}y}{\textnormal{d}x} \; = \; f\left(\frac{a_1x + b_1y + c_1}{a_2x + b_2y + c_2}\right)
\]

\medskip

\noindent przy za�o�eniu, �e

\[
\det \left[
\begin{array}{cc}
a_1 & b_1\\
a_2 & b_2
\end{array}
\right]
\; \neq \; 0
\]

\medskip

(Def. \ref{def:wyznacznik}, str. \pageref{def:wyznacznik})

\medskip

\noindent wprowadzamy nowe zmienne:
\begin{itemize}
\item niezale�na $\xi$
\item zale�n� $\eta$
\end{itemize}

\[
\left\{
\begin{array}{rcl}
x & = & \xi + \alpha\\
y & = & \eta + \beta
\end{array}
\right.
\]

\medskip

\noindent gdzie $\alpha$ i~$\beta$ spe�niaj� uk�ad r�wna� (Def. \ref{def:uklad_rownan}, str. \pageref{def:uklad_rownan})

\[
\left\{
\begin{array}{rcl}
a_1\alpha + b_1\beta + c_1 & = & 0\\
a_2\alpha + b_2\beta + c_2 & = & 0
\end{array}
\right.
\]

R�wnanie przyjmuje posta�

\[
\frac{\textnormal{d}\eta}{\textnormal{d}\xi} \; = \; f\left(\frac{a_1\xi + b_1\eta}{a_2\xi + b_2\eta}\right)
\]

\end{itemize}

\smallskip

\cite[Rozdzia� 1.3.2]{niedoba}

\bigskip
%%%%%%%%%%%%%%%%%%%%%%%%%%%%%
\subsection{R�wnania liniowe}

\label{def:rownanie_rozniczkowe_liniowe}

\begin{definicja}
R�wnanie postaci

\[
y' \; + \; p(x) \: y \; = \; q(x)
\]

\medskip

\noindent nazywamy \textbf{\emph{r�wnaniem liniowym niejednorodnym}}\label{def:rownanie_rozniczkowe_liniowe_niejednorodne} (Def. \ref{def:rownanie_niejednorodne}, str. \pageref{def:rownanie_niejednorodne}).

\smallskip

\cite[Rozdzia� 1.3.3]{niedoba}
\end{definicja}

\medskip

\begin{definicja}
R�wnanie postaci

\[
y' \; + \; p(x) \: y \; = \; 0
\]

\medskip

\noindent nazywamy \textbf{\emph{r�wnaniem liniowym jednorodnym}} (Def. \ref{def:rownanie_jednorodne}, str. \pageref{def:rownanie_jednorodne}).

\smallskip

\cite[Rozdzia� 1.3.3]{niedoba}
\end{definicja}

\medskip

\textbf{Ca�k� og�ln�} (Def. \ref{def:calka_ogolna}, str. \pageref{def:calka_ogolna}) r�wnania liniowego \textbf{jednorodnego} obliczamy, korzystaj�c z~r�wnania o~zmiennych rozdzielonych i~otrzymujemy

\[
y_1 \; = \; C\: e^{-P(x)}
\]

\medskip

\noindent gdzie $P(x)$ jest funkcj� pierwotn� (Def. \ref{def:funkcja_pierwotna}, str. \pageref{def:funkcja_pierwotna}) do $p(x)$.

\medskip

\textbf{Ca�k� szczeg�ln�} (Def. \ref{def:calka_szczegolna}, str. \pageref{def:calka_szczegolna}) r�wnania liniowego \textbf{niejednorodnego} obliczamy, korzystaj�c z~\textbf{metody uzmienniania sta�ej}.

\medskip

Przewidujemy, �e funkcja

\[
y_1 \; = \; C(x)\: e^{-P(x)} \qquad C \in C^1 \langle a,b\rangle
\]

\medskip

\noindent jest rozwi�zaniem r�wnania liniowego jednorodnego.

\medskip

Obliczamy $y_1'$

\[
y_1' \; = \; C'(x) \: e^{-P(x)} - C(x) \: e^{-P(x)} \: p(x)
\]

\medskip

Wstawiamy $y_1$ oraz $y_1'$ do r�wnania liniowego niejednorodnego i~otrzymujemy

\[
C'(x) \: e^{-P(x)} \; = \; q(x)
\]

\medskip

Sk�d

\[
C(x) \; = \; \int q(x) \: e^{P(x)} \textnormal{d}x
\]

\medskip

\textbf{Ca�ka og�lna} r�wnania liniowego \textbf{niejednorodnego} jest \textbf{sum�}

\begin{itemize}
\item ca�ki \textbf{og�lnej} (Def. \ref{def:calka_ogolna}, str. \pageref{def:calka_ogolna}) r�wnania liniowego \textbf{jednorodnego}
\item i ca�ki \textbf{szczeg�lnej} (Def. \ref{def:calka_szczegolna}, str. \pageref{def:calka_szczegolna}) r�wnania liniowego \textbf{niejednorodnego}
\end{itemize}

\medskip
St�d

\[
y_0 \; = \; e^{-P(x)} \: \left[C + \int q(x) \: e^{P(x)} \textnormal{d}x\right]
\]

\bigskip
%%%%%%%%%%%%%%%%%%%%%%%%%%%%%%%%%%
\subsection{R�wnanie Bernoulliego}

Niech

\begin{itemize}
\item $p, q \in C\langle a,b\rangle$
\item $r \in \mathbb{R}$
\end{itemize}

\begin{definicja}
R�wnanie postaci

\[
y' \; + \; p(x) \: y \; = \; q(x) \: y^r
\]

\medskip

\noindent nazywamy \textbf{\emph{r�wnaniem Bernoulliego}}.

\smallskip

\cite[Rozdzia� 1.3.4]{niedoba}
\end{definicja}

\medskip

Dla $r \in \{0,1\}$ powy�sze r�wnanie jest \textbf{r�wnaniem liniowym} (Def. \ref{def:rownanie_rozniczkowe_liniowe}, str. \pageref{def:rownanie_rozniczkowe_liniowe}).

\medskip

R�wnanie Bernoulliego rozwi�zujemy dziel�c obie strony r�wnania przez~$y^r$, a~nast�pnie wprowadzamy now� zmienn� zale�n� $z = y^{1-r}$. Obliczamy~$z'$

\[
z' \; = \; (1-r) \: y^{-r} \: y'
\]

\medskip

St�d

\[
y' \; = \; \frac{y^r \: z'}{1-r}
\]

\medskip

Po wstawieniu otrzymujemy nast�puj�ce r�wnanie

\[
\frac{1}{1-r} \: z' \; + \; p(x) \: z \; = \; q(x)
\]

\medskip

Jest to \textbf{r�wnanie liniowe niejednorodne} (Def. \ref{def:rownanie_rozniczkowe_liniowe_niejednorodne}, str. \pageref{def:rownanie_rozniczkowe_liniowe_niejednorodne}).

\bigskip
%%%%%%%%%%%%%%%%%%%%%%%%%%%%%
\subsection{R�wnanie zupe�ne}

\cite[Paragraf 11.5, Paragraf 11.6]{krysicki2}

\medskip

Niech wyra�enie

\[
P(x,y) \: \textnormal{d}x + Q(x,y) \: \textnormal{d}y
\]

\medskip

\noindent jest \textbf{r�niczk� zupe�n�} (Def. \ref{def:rozniczka_zupelna}, str. \pageref{def:rozniczka_zupelna}) pewnej funkcji dw�ch zmiennych $F(x,y)$ (Def. \ref{def:funkcja_wielu_zmiennych}, str. \pageref{def:funkcja_wielu_zmiennych}) okre�lonej w~obszarze $D$.

\medskip

\begin{definicja}
\textbf{\emph{R�wnaniem r�niczkowym zupe�nym}} nazywamy r�wnanie r�niczkowe rz�du I-go postaci

\[
P(x,y) \; + \; Q(x,y) \: \frac{\textnormal{d}y}{\textnormal{d}x} \; = \; 0
\]
\end{definicja}

\bigskip
%%%%%%%%%%%%%%%%%%%%%%%%%%%%%%%%%%%%%%%%%%%%%%
\subsubsection{Rozwi�zania r�wnania zupe�nego}

Niech

\begin{itemize}
\item $C$ jest sta�� dowoln� ca�kowania
\item punkty $(x,y)$ nale�� do obszaru $D$
\item wyra�enie 

\[
P(x,y) \: \textnormal{d}x + Q(x,y) \: \textnormal{d}y
\]

\medskip

\noindent jest przy podanych za�o�eniach \textbf{r�niczk� zupe�n�} (Def. \ref{def:rozniczka_zupelna}, str. \pageref{def:rozniczka_zupelna}) funkcji $F(x,y)$
\end{itemize}

\begin{twierdzenie}
Je�eli spe�nione s� za�o�enia, to r�wnanie

\[
F(x,y) \; = \; C
\]

\medskip

\noindent okre�la \textbf{wszystkie rozwi�zania r�wnania zupe�nego}

\[
P(x,y) \; + \; Q(x,y) \: \frac{\textnormal{d}y}{\textnormal{d}x} \; = \; 0
\]
\end{twierdzenie}

\bigskip
%%%%%%%%%%%%%%%%%%%%%%%%%%%%%%%%%%%%%%%%%%%%%%%%%%%%%%%%%%%
\subsection{R�wnanie r�niczkowe sprowadzalne do zupe�nego}

\subsubsection{R�wnanie r�niczkowe nie spe�niaj�ce za�o�enia zupe�nego}

Wr��my do r�wnania 

\[
P(x,y) + Q(x,y) \: \frac{\textnormal{d}y}{\textnormal{d}x} = 0
\]

\medskip

\noindent i przyjmijmy, �e

\begin{itemize}
\item funkcje $P(x,y)$ i~$Q(x,y)$ s� klasy $C^1$ w~obszarze jednosp�jnym (Def. \ref{def:obszar_jednospojny}, str. \pageref{def:obszar_jednospojny}) $D$
\item oraz, �e \textbf{niespe�niony} jest \textbf{warunek}

\[
\frac{\partial P}{\partial y} = \frac{\partial Q}{\partial x}
\]

\medskip

\noindent w~ka�dym punkcie obszaru $D$, czyli \textbf{niespe�nione} jest \textbf{za�o�enie} dotycz�ce r�niczki zupe�nej, wi�c

\[
P(x,y) \: \textnormal{d}x + Q(x,y) \: \textnormal{d}y
\]

\medskip

\noindent \textbf{nie jest r�niczk� zupe�n�} funkcji $F(x,y)$.

\end{itemize}

\bigskip

W tym przypadku r�wnanie postaci

\[
P(x,y) + Q(x,y) \: \frac{\textnormal{d}y}{\textnormal{d}x} = 0
\]

\medskip

\noindent \textbf{nie jest} r�wnaniem r�niczkowym \textbf{zupe�nym}.

\bigskip
%%%%%%%%%%%%%%%%%%%%%%%%%%%%%%%%%
\subsubsection{Czynnik ca�kuj�cy}

Mo�na wykaza�, �e dla ka�dego r�wnania r�niczkowego postaci

\[
P(x,y) + Q(x,y) \: \frac{\textnormal{d}y}{\textnormal{d}x} = 0
\]

\medskip

\noindent \textbf{istnieje niesko�czenie wiele funkcji}

\[
\mu(x,y) \not\equiv 0
\]

\medskip

\noindent w~rozpatrywanym obszarze, �e je�eli \textbf{pomno�ymy} przez t� \textbf{funkcj� obie strony r�wnania} 

\[
P(x,y) + Q(x,y) \: \frac{\textnormal{d}y}{\textnormal{d}x} = 0
\]

\medskip

\noindent to otrzymane w ten spos�b r�wnanie

\[
\mu(x,y) \: P(x,y) + \mu(x,y) \: Q(x,y) \: \frac{\textnormal{d}y}{\textnormal{d}x} = 0
\]

\medskip

\noindent \textbf{jest} r�wnaniem \textbf{zupe�nym}, tzn. spe�niony jest warunek

\[
\frac{\partial P}{\partial y} = \frac{\partial Q}{\partial x}
\]

\medskip

\noindent dla otrzymanego r�wnania, tzn.

\[
\frac{\partial \left(\mu \: P\right)}{\partial y} = \frac{\partial \left(\mu \: Q\right)}{\partial x} 
\]

\bigskip
%%%%%%%%%%%%%%%%%%%%%%%%%%%%%%%%%%%%%%%%%%%%%%%%
\subsubsection{Znajdowanie czynnika ca�kuj�cego}

Znalezienie czynnik�w ca�kuj�cych w og�lnym przypadku prowadzi do rozwi�zania r�wnania r�niczkowego 

\[
\frac{\partial \left(\mu \: P\right)}{\partial y} = \frac{\partial \left(\mu \: Q\right)}{\partial x} 
\]

\medskip

\noindent rz�du I-go o~pochodnych cz�stkowych, z~funkcj� niewiadom� $\mu(x,y)$, kt�re jest na og� trudniejsze do rozwi�zania ni� r�wnanie

\[
P(x,y) + Q(x,y) \: \frac{\textnormal{d}y}{\textnormal{d}x} = 0
\]

\bigskip

W pewnych specjalnych przypadkach jednak �atwo jest znale�� czynnik ca�kuj�cy.

\medskip

Poni�ej przedstawione jest kilka najprostszych.

\begin{itemize}
\item \textbf{Znajdowanie czynnika ca�kuj�cego} - przypadek \textbf{I}

Niech
\begin{itemize}
\item funkcje $P(x,y)$ i~$Q(x,y)$ s� klasy $C^1$ w~obszarze jednosp�jnym $D$
\item $Q(x,y) \neq 0$
\item wyra�enie

\[
\frac{1}{Q(x,y)} \left(\frac{\partial P}{\partial y} - \frac{\partial Q}{\partial x} \right)
\]

\medskip

jest \textbf{funkcj� tylko jednej zmiennej} $x$
\end{itemize}

\medskip

Je�eli spe�nione s� za�o�enia, to istnieje \textbf{czynnik ca�kuj�cy $\mu(x)$} r�wnania 

\[
P(x,y) + Q(x,y) \: \frac{\textnormal{d}y}{\textnormal{d}x} = 0
\]

\medskip

\noindent kt�ry jest \textbf{funkcj� tylko zmiennej $x$}, okre�lony r�wno�ci�

\[
\mu(x) = \exp \left\{\int \frac{1}{Q} \left(\frac{\partial P}{\partial y} - \frac{\partial Q}{\partial x} \right) \textnormal{d}x \right\}
\]

\bigskip

\item \textbf{Znajdowanie czynnika ca�kuj�cego} - przypadek \textbf{II}

Niech

\begin{itemize}
\item funkcje $P(x,y)$ i~$Q(x,y)$ s� klasy $C^1$ w~obszarze jednosp�jnym $D$
\item $P(x,y) \neq 0$
\item wyra�enie

\medskip

\[
\frac{1}{P(x,y)} \left(\frac{\partial Q}{\partial x} - \frac{\partial P}{\partial y} \right)
\]

jest \textbf{funkcj� tylko jednej zmiennej} $y$
\end{itemize}

\medskip

Je�eli spe�nione s� za�o�enia, to istnieje \textbf{czynnik ca�kuj�cy $\mu(y)$} r�wnania

\[
P(x,y) + Q(x,y) \: \frac{\textnormal{d}y}{\textnormal{d}x} = 0
\]

\medskip

\noindent kt�ry jest \textbf{funkcj� tylko zmiennej $y$}, okre�lony r�wno�ci�

\[
\mu(y) = \exp \left\{\int \frac{1}{P} \left(\frac{\partial Q}{\partial x} - \frac{\partial P}{\partial y} \right) \textnormal{d}y \right\}
\]

\bigskip

\item \textbf{Znajdowanie czynnika ca�kuj�cego} - przypadek \textbf{III}

Niech

\begin{itemize}
\item funkcje $P(x,y)$ i~$Q(x,y)$ s� klasy $C^1$ w~obszarze jednosp�jnym $D$
\item istniej� takie dwie funkcje $f(x)$ i~$g(y)$ spe�niaj�ce to�samo�ciowo r�wno��

\[
\frac{\partial P}{\partial y} - \frac{\partial Q}{\partial x} \quad \equiv \quad Q(x,y) \: f(x) - P(x,y) \: g(y)
\]
\end{itemize}

\medskip

Je�eli spe�nione s� za�o�enia, to istnieje \textbf{czynnik ca�kuj�cy $\mu(x,y)$} b�d�cy \textbf{iloczynem dw�ch funkcji $\varphi(x)$ i~$\psi(x)$} okre�lonych r�wno�ciami

\begin{eqnarray*}
\varphi(x) & = & \exp\left(\int f(x) \: \textnormal{d}x \right) \\
\psi(y) & = & \exp\left(\int g(y) \: \textnormal{d}y \right)
\end{eqnarray*}

\end{itemize}


\bigskip
%%%%%%%%%%%%%%%%%%%%%%%%%%%%%%%%%%%%%%%%%%%%%%%%%%%%
\section{R�wnania r�niczkowe zwyczajne rz�du II-go}

\subsection{R�wnania rz�du II-go sprowadzalne do rz�du I-go}

\cite[Rozdzia� 1.7]{zakowski4}

\medskip

Rozwi�zywanie niekt�rych r�wna� r�niczkowych II-go rz�du mo�na sprowadzi� za pomoc� podstawie� do rozwi�zywania r�wna� I-go rz�du.

\begin{itemize}
\item R�wnanie typu $F(x, \: y', \: y'') \; = \; 0$.

Wprowadzamy now� zmienn� zale�n�

\[
u \; = \; y'
\]

St�d

\[
u' \; = \; y''
\]

Po podstawieniu otrzymujemy r�wnanie r�niczkowe I-go rz�du

\[
F(x, \: u, \: u') \; = \; 0
\]
\item R�wnanie typu $F(y, \: y', \: y'') \; = \; 0$.

Wprowadzamy now� zmienn� zale�n�

\[
u(y) \; = \; y'
\]

St�d

\[
u'y' \; = \; y''
\]

\[
u'u \; = \; y''
\]

Po podstawieniu otrzymujemy r�wnanie r�niczkowe I-go rz�du

\[
F(y, \: u, \: u'u) \; = \; 0
\]
\end{itemize}

\bigskip
%%%%%%%%%%%%%%%%%%%%%%%%%%%%%
\subsection{R�wnania liniowe}

\cite[Rozdzia� 1.8]{zakowski4} \cite[Rozdzia� 3.1]{niedoba} - r�wnania liniowe rz�du $n$

\medskip

\label{def:rownanie_rozniczkowe_liniowe_II-go_rzedu}

\begin{definicja}
R�wnanie postaci

\[
y'' \; + \; p(x) \: y' \; + \; q(x) \: y \; = \; f(x)
\]

\medskip

\noindent nazywamy \textbf{\emph{r�wnaniem liniowym niejednorodnym II-go rz�du}}\label{def:rownanie_rozniczkowe_liniowe_niejednorodne_II-go_rzedu} (Def. \ref{def:rownanie_niejednorodne}, str. \pageref{def:rownanie_niejednorodne}).
\end{definicja}

\medskip

\begin{definicja}
R�wnanie postaci

\[
y'' \; + \; p(x) \: y' \; + \; q(x) \: y \; = \; 0
\]

\medskip

\noindent nazywamy \textbf{\emph{r�wnaniem liniowym jednorodnym II-go rz�du}} (Def. \ref{def:rownanie_jednorodne}, str. \pageref{def:rownanie_jednorodne}).
\end{definicja}


\bigskip
%%%%%%%%%%%%%%%%%%%%%%%%%%%%%%%%%%%%%%%%%%%%%%%%%%%%%%%
\subsection{R�wnania liniowe o~sta�ych wsp�czynnikach}

Rozwa�my problem pocz�tkowy

\[
y'' \; + \; a_1 \: y' \; + \; a_2 \: y \; = \; f(x)
\]

\[
\left\{\begin{array}{rcl}
y(x_0) & = & y_0\\
y'(x_0) & = & y_1
\end{array}
\right.
\]

\medskip

\noindent gdzie 
\begin{itemize}
\item $x_0 \in (a,b)$
\item $a_k \in \mathbb{R}$
\end{itemize}

\medskip

Zaczniemy od wyznaczenia ca�ki og�lnej (Def. \ref{def:calka_ogolna}, str. \pageref{def:calka_ogolna}) r�wnania liniowego jednorodnego (Def. \ref{def:rownanie_jednorodne}, str. \pageref{def:rownanie_jednorodne}) stowarzyszonego z~r�wnaniem niejednorodnym

\[
y'' \; + \; a_1 \: y' \; + \; a_2 \: y \; = \; 0
\]

\medskip

Wprowadzamy nowe zmienne 

\[
\left\{\begin{array}{rcl}
t_1(x) & = & y(x)\\
t_2(x) & = & y'(x)
\end{array}
\right.
\]

\medskip

Problem pocz�tkowy przyjmuje posta�

\[
\left\{\begin{array}{rcl}
t'_1(x) & = & y'(x)\\
		& = & t_2(x)\\[1.2em]
t'_2(x) & = & y''(x)\\
		& = & - \; a_1 \: y' \; - \; a_2 \: y\\
		& = & - \; a_1 \: t_2(x) \; - \; a_2 \: t_1(x)
\end{array}
\right.
\]

\[
\left\{\begin{array}{rcl}
t_1(x_0) & = & y_0\\
t_2(x_0) & = & y_1
\end{array}
\right.
\]

\medskip

Zapisujemy uk�ad r�wna� (Def. \ref{def:uklad_rownan}, str. \pageref{def:uklad_rownan}) w~postaci macierzowej (Def. \ref{def:macierz}, str. \pageref{def:macierz})

\[
\left[
\begin{array}{cc}
0 & 1\\
-a_2 & -a_1
\end{array}
\right]
\left[
\begin{array}{c}
t_1 \\
t_2
\end{array}
\right]
=
\left[
\begin{array}{c}
t'_1 \\
t'_2
\end{array}
\right]
\]

\medskip

Wyznaczamy r�wnanie charakterystyczne (Def. \ref{def:rownanie_charakterystyczne}, str. \pageref{def:rownanie_charakterystyczne}) macierzy wsp�czynnik�w tego uk�adu r�wna�

\begin{eqnarray*}
\det\left(A - \lambda I\right) & = & \det\left(
\left[
\begin{array}{cc}
0 & 1\\
-a_2 & -a_1
\end{array}
\right]
- \lambda
\left[
\begin{array}{cc}
1 & 0\\
0 & 1
\end{array}
\right]
\right)\\[0.5em]
& = & \det
\left[
\begin{array}{cc}
-\lambda & 1\\
-a_2 & -a_1 - \lambda
\end{array}
\right]\\[0.5em]
& = & \lambda^2 \; + \; a_1 \: \lambda \; + \; a_2
\end{eqnarray*}

\medskip

Pierwiastki tego r�wnania nazywamy r�wnie� pierwiastkami charakterystycznymi r�wnania niejednorodnego

\medskip

Je�eli $\Delta \; = \; a_1^2 \; - \; 4 \: a_2$ r�wnania charakterystycznego jest

\begin{itemize}
\item $>0$, to rozwi�zaniem r�wnania jednorodnego odpowiadaj�cym warto�ci w�asnej $\lambda_1$ (Def. \ref{def:wartosc_wlasna}, str. \pageref{def:wartosc_wlasna}) jest funkcja

\[
y_1 \; = \; e^{\lambda_1 \: x} \: C_1 
\]

\medskip

\noindent a rozwi�zaniem r�wnania jednorodnego odpowiadaj�cym warto�ci w�asnej $\lambda_2$ (Def. \ref{def:wartosc_wlasna}, str. \pageref{def:wartosc_wlasna}) jest funkcja

\[
y_2 \; = \; e^{\lambda_2 \: x} \: C_2
\]

\medskip

\textbf{Ca�k� og�ln� r�wnania jednorodnego} jest funkcja

\[
y \; = \; C_1 \: e^{\lambda_1 \: x} \; + \; C_2 \: e^{\lambda_2 \: x} 
\]

\item $= 0$, to rozwi�zaniem r�wnania jednorodnego odpowiadaj�cym warto�ci w�asnej $\lambda_0$jest funkcja

\[
y_0 \; = \; e^{\lambda_0 \: x} \: \left(C_1 \: + \: C_2 \: x\right)
\]

\medskip

Poniewa� wyst�puje jeden pierwiastek charakterystyczny, to \textbf{ca�ka og�lna r�wnania liniowego jednorodnego} jest r�wna rozwi�zaniu r�wnania jednorodnego odpowiadaj�cym warto�ci w�asnej $\lambda_0$, czyli

\[
y \; = \; \left(C_1 \: + \: C_2 \: x\right) \: e^{\lambda_0 \: x}
\]
\item $< 0$, to rozwi�zaniem r�wnania jednorodnego odpowiadaj�cym warto�ci w�asnej $\lambda \; = \; \alpha \: + \: i \: \beta$ (Def. \ref{def:liczba_zespolona}, str. \pageref{def:liczba_zespolona}) jest funkcja

\[
y_1 \; = \; e^{(\alpha \: + \: i \: \beta)}
\]

\medskip

\noindent a rozwi�zaniem r�wnania jednorodnego odpowiadaj�cym warto�ci w�asnej $\overline{\lambda} \; = \; \alpha \: - \: i \: \beta$ sprz�onej z $\lambda$ (Def. \ref{def:liczba_zespolona_sprzezona}, str. \pageref{def:liczba_zespolona_sprzezona}) jest funkcja

\[
y_2 \; = \; e^{(\alpha \: - \: i \: \beta)}
\]

\medskip

Korzystaj�c ze wzoru Eulera (Def. \ref{def:wzor_eulera}, str. \pageref{def:wzor_eulera}) otrzymujemy

\[
y_1 \; = \; e^{\alpha \: x} \: \cos{\beta \: x} \: + \: i \: e^{\alpha \: x} \: \sin{\beta \: x}
\]

\medskip

Na mocy twierdzenia

\begin{twierdzenie}
Je�eli funkcja zespolona (Def. \ref{def:funkcja_zespolona}, str. \pageref{def:funkcja_zespolona}) zmiennej rzeczywistej $x$

\[
w(x) \; = \; u(x) \: + \: i \: v(x)
\]

\medskip

\noindent jest \textbf{ca�k� r�wnania}

\[
y'' \; + \; p(x) \: y' \; + \; q(x) \: y \; = \; 0
\]

\medskip

\noindent z rzeczywistymi wsp�czynnikami $p(x)$ i~$q(x)$ w~przedziale $(a,b)$, to jej

\begin{itemize}
\item cz�� rzeczywista $u(x)$
\item i~cz�� urojona $v(x)$ 
\end{itemize}

\medskip

\noindent \textbf{s�} tak�e \textbf{ca�kami} tego \textbf{r�wnania} w~przedziale $(a,b)$

\smallskip

\cite[Rozdzia� 1.8, Tw. 3]{zakowski4}
\end{twierdzenie}

\medskip

\noindent funkcje 

\[
y_{11} \; = \; e^{\alpha \: x} \: \cos{\beta \: x}
\]

\[
y_{12} \; = \; e^{\alpha \: x} \: \sin{\beta \: x}
\]

\medskip

\noindent s� tak�e ca�kami r�wnania jednorodnego.

\medskip

\textbf{Ca�ka og�lna r�wnania jednorodnego} wyra�a si� wzorem

\[
y \; = \; e^{\alpha \: x} \: (C_1 \: \sin{\beta \: x} \: + \:  C_2 \: \cos{\beta \: x})
\]
\end{itemize}

\medskip

\subsection{R�wnanie Eulera}
\chapter{Szeregi liczbowe}

\section{Szereg liczbowy}

\section{Zbie�no�� szeregu}

\section{Szereg harmoniczny}

\section{Szereg Dirichleta}

\section{Szereg naprzemienny}

\section{Kryteria zbie�no�ci szereg�w}

\subsection{Kryterium por�wnawcze}

\subsection{Kryterium D'Alamberta}

\subsection{Kryterium Cauchy'ego}

\subsection{Kryterium Leibnitza}

\subsection{Kryterium ca�kowe}
\chapter{Szeregi funkcyjne}

\section{Ci�g funkcyjny}

\section{Szereg funkcyjny}

\section{Zbie�no�� szeregu}

\section{Kryterium Weierstrassa}

% Dodatki
\appendix

\include{licencja}

\backmatter

% Indeks
\printindex

% Bibliografia
\begin{thebibliography}{99}
\bibitem{rut} Jerzy Rutkowski: \emph{Algebra Abstrakcyjna w zadaniach}, PWN 2005
\bibitem{krysicki1} W. Krysicki, L. W�odarski: \emph{Analiza matematyczna w zadaniach 1}, PWN 2004
\bibitem{ptak} Marek Ptak: \emph{Matematyka dla student�w kierunk�w technicznych i przyrodniczych}, Wydawnictwo AR w Krakowie 2006
\bibitem{przybylo} Sylwester Przyby�o, Andrzej Szlachtowski: \emph{algebra i wielowymiarowa geometria analityczna w zadaniach} WNT 2005
\bibitem{kostrykin1} Aleksiej I. Kostrykin: \emph{Wst�p do algebry. Podstawy algebry. 1}. PWN 2004
\bibitem{kostrykin2} Aleksiej I. Kostrykin: \emph{Wst�p do algebry. Podstawy algebry. 2}. PWN 2004
\bibitem{furdzik} Zbigniew Furdzik, Janina Maj-Kluskowa, Alicja Kulczycka, Magdalena S�kowska: \emph{Nowoczesna matematyka dla in�ynier�w. Cz�� I. Algebra}. AGH 1998
\bibitem{zakowski1} W. �akowski, G. Decewicz: \emph{Matematyka, cz. I}. WNT 1968, 1991
\bibitem{trajdos} T. Trajdos: \emph{Matematyka, cz. III}. WNT 1971, 1993
\bibitem{onyszkiewicz} Wiktor Marek, Janusz Onyszkiewicz: \emph{Elementy logiki i teorii mnogo�ci w zadaniach}. PWN 2005
\bibitem{niedoba} Janina Niedoba, Wies�aw Niedoba: \emph{R�wnania r�niczkowe zwyczajne i~cz�stkowe. Zadania z~matematyki}. AGH 2001
\bibitem{zakowski4} W. �akowski, W. Leksi�ski: \emph{Matematyka, cz. IV}. WNT 1971, 1995
\bibitem{krysicki2} W. Krysicki, L. W�odarski: \emph{Analiza matematyczna w zadaniach 2}, PWN 2000
\end{thebibliography}

\end{document}
