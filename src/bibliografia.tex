\begin{thebibliography}{99}
\bibitem{rut} Jerzy Rutkowski: \emph{Algebra Abstrakcyjna w zadaniach}, PWN 2005
\bibitem{krysicki1} W. Krysicki, L. W�odarski: \emph{Analiza matematyczna w zadaniach 1}, PWN 2004
\bibitem{ptak} Marek Ptak: \emph{Matematyka dla student�w kierunk�w technicznych i przyrodniczych}, Wydawnictwo AR w Krakowie 2006
\bibitem{przybylo} Sylwester Przyby�o, Andrzej Szlachtowski: \emph{algebra i wielowymiarowa geometria analityczna w zadaniach} WNT 2005
\bibitem{kostrykin1} Aleksiej I. Kostrykin: \emph{Wst�p do algebry. Podstawy algebry. 1}. PWN 2004
\bibitem{kostrykin2} Aleksiej I. Kostrykin: \emph{Wst�p do algebry. Podstawy algebry. 2}. PWN 2004
\bibitem{furdzik} Zbigniew Furdzik, Janina Maj-Kluskowa, Alicja Kulczycka, Magdalena S�kowska: \emph{Nowoczesna matematyka dla in�ynier�w. Cz�� I. Algebra}. AGH 1998
\bibitem{zakowski1} W. �akowski, G. Decewicz: \emph{Matematyka, cz. I}. WNT 1968, 1991
\bibitem{trajdos} T. Trajdos: \emph{Matematyka, cz. III}. WNT 1971, 1993
\bibitem{onyszkiewicz} Wiktor Marek, Janusz Onyszkiewicz: \emph{Elementy logiki i teorii mnogo�ci w zadaniach}. PWN 2005
\bibitem{niedoba} Janina Niedoba, Wies�aw Niedoba: \emph{R�wnania r�niczkowe zwyczajne i~cz�stkowe. Zadania z~matematyki}. AGH 2001
\bibitem{zakowski4} W. �akowski, W. Leksi�ski: \emph{Matematyka, cz. IV}. WNT 1971, 1995
\bibitem{krysicki2} W. Krysicki, L. W�odarski: \emph{Analiza matematyczna w zadaniach 2}, PWN 2000
\end{thebibliography}